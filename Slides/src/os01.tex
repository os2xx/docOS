%%%%%%%%%%%%%%%%%%%%%%%%%%%%%%%%%%%%%%%%%%%%%%%%%%%%%%%%%%%%%%%%%%%%%%%%
% Beamer Presentation - LaTeX - Template Version 1.0 (10/11/12)
% This template has been downloaded from: http://www.LaTeXTemplates.com
% License: % CC BY-NC-SA 3.0 (http://creativecommons.org/)
% Modified by Rahmat M. Samik-Ibrahim

% REV419: Wed 24 Jul 2024 18:00
% REV395: Sun 11 Sep 2022 22:00
% REV386: Thu 28 Jul 2022 10:00
% REV369: Mon 14 Feb 2022 16:00
% REV363: Mon 20 Dec 2021 15:00
% STARTX: Wed 14 Sep 2016 10:00
%%%%%%%%%%%%%%%%%%%%%%%%%%%%%%%%%%%%%%%%%%%%%%%%%%%%%%%%%%%%%%%%%%%%%%%%%

% PACKAGES AND THEMES 
\documentclass[aspectratio=169, xcolor=table, notheorems, hyperref={pdfpagelabels=false}]{beamer}
%%%%%%%%%%%%%%%%%%%%%%%%%%%%%%%%%%%%%%%%%%%%%%%%%%%%%%%%%%%%%%%%%%%%%%%%
% Beamer Presentation - LaTeX - Template Version 1.0 (10/11/12)
% This template has been downloaded from: http://www.LaTeXTemplates.com
% License: % CC BY-NC-SA 3.0 (http://creativecommons.org/)
% Modified by Bin Kadal, Sdn Bhd.
% REV023: Tue 30 Jan 2024 16:00
% REV006: Mon 22 Jan 2018 19:00
% STARTX: Thu 25 Aug 2016 14:00
%%%%%%%%%%%%%%%%%%%%%%%%%%%%%%%%%%%%%%%%%%%%%%%%%%%%%%%%%%%%%%%%%%%%%%%%%

%% ZCZC NNNN
\newtheorem{example}{Example}

%%%%%%%%%%%%%%%%%%%%%%%%%%%%%%%%%%%%%%%%%%%%%%%%%%%%%%%%%%%%%%%%%%%%%%%%%

\let\Tiny=\tiny
\mode<presentation> {
% The Beamer class comes with a number of default slide themes
% which change the colors and layouts of slides. Below this is a list
% of all the themes, uncomment each in turn to see what they look like.
%\usetheme{Boadilla}
\usetheme{Madrid}
% ZCZC %%%%%%%%%%%%%%%%%%%%%%%%%%%%%%%%%%%%%%%%%%%%%%%%%%%%%%%%%%%%%%%%%%
% \usetheme{default} \usetheme{AnnArbor} \usetheme{Antibes} \usetheme{Bergen}
% \usetheme{Berkeley} \usetheme{Berlin} \usetheme{CambridgeUS} 
% \usetheme{Copenhagen} \usetheme{Darmstadt} \usetheme{Dresden}
% \usetheme{Frankfurt} \usetheme{Goettingen} \usetheme{Hannover}
% \usetheme{Ilmenau} \usetheme{JuanLesPins} \usetheme{Luebeck}
% \usetheme{Malmoe} \usetheme{Marburg} \usetheme{Montpellier}
% \usetheme{PaloAlto} \usetheme{Pittsburgh} \usetheme{Rochester}
% \usetheme{Singapore} \usetheme{Szeged} \usetheme{Warsaw}
% NNNN %%%%%%%%%%%%%%%%%%%%%%%%%%%%%%%%%%%%%%%%%%%%%%%%%%%%%%%%%%%%%%%%%%
% As well as themes, the Beamer class has a number of color themes
% for any slide theme. Uncomment each of these in turn to see how it
% changes the colors of your current slide theme.
%\usecolortheme{orchid}
%\usecolortheme{rose}
%\usecolortheme{seagull}
%\usecolortheme{seahorse}
\usecolortheme{whale}
% ZCZC %%%%%%%%%%%%%%%%%%%%%%%%%%%%%%%%%%%%%%%%%%%%%%%%%%%%%%%%%%%%%%%%%%
%\usecolortheme{albatross} \usecolortheme{beaver} \usecolortheme{beetle}
%\usecolortheme{crane} \usecolortheme{dolphin} \usecolortheme{dove}
%\usecolortheme{fly} \usecolortheme{lily} \usecolortheme{wolverine}
% NNNN %%%%%%%%%%%%%%%%%%%%%%%%%%%%%%%%%%%%%%%%%%%%%%%%%%%%%%%%%%%%%%%%%%
% To remove the footer line in all slides uncomment this line
%\setbeamertemplate{footline} 
% To replace the footer line in all slides uncomment this line
%\setbeamertemplate{footline}[page number] 
% To remove the navigation symbols from the bottom uncomment this line
\setbeamertemplate{navigation symbols}{} 
}

\usepackage{array}       % ZCZC
\usepackage{amssymb}     % ZCZC
\usepackage{bold-extra}  % ZCZC
\usepackage{booktabs}    % Allows \toprule, \midrule and \bottomrule in tables
\usepackage{caption}
\usepackage[T1]{fontenc} % ZCZC << >>
\usepackage{graphicx}    % Allows including images
\usepackage{listings}    % listing
\usepackage{lmodern}     % ZCZC
\usepackage{perpage}     % reset footnote per page
\usepackage{geometry}    % ZCZC
\usepackage{adjustbox}   % ZCZC
\usepackage{multirow}    % ZCZC

% \definecolor{links}{HTML}{2A1B81}
\definecolor{links}{HTML}{0011FF}
\hypersetup{colorlinks,linkcolor=,urlcolor=links}

% \usepackage{xcolor}
% \usepackage[colorlinks = true,
%             linkcolor = blue,
%             urlcolor  = blue,
%             citecolor = blue,
%             anchorcolor = blue]{hyperref}

\captionsetup[table]{name=Tabel}
\makeatletter
\def\input@path{{src/}}
\makeatother
\graphicspath{{src/}}      % src directory
\MakePerPage{footnote}     % reset page

% NNNN %%%%%%%%%%%%%%%%%%%%%%%%%%%%%%%%%%%%%%%%%%%%%%%%%%%%%%%%%%%%%%%%%%

%% % XXXXXXXXXXXXXXXXXXXXXXXXXXXXXXXXXXXXXXXXXXXXXXXXXXXXXXXXXXXXXXXXXXXXXXXXXX
%% % The short title appears at the bottom of every slide, 
%% % the full title is only on the title page
%% \title[Judul Pendek]{Judul Panjang dan Lengkap} 
%% \author{Cecak bin Kadal}
%% \institute[UILA]
%% {
%% University of Indonesia at Lenteng Agung \\ 
%% \medskip
%% \textit{cecak@binKadal.com}
%% }
%% \date{REV00 24-Aug-2016}
%% % \date{\today}
%% 

%% % XXXXXXXXXXXXXXXXXXXXXXXXXXXXXXXXXXXXXXXXXXXXXXXXXXXXXXXXXXXXXXXXXXXXXXXXXX
%% \begin{document}
%% \section{Judul}
%% \begin{frame}
%% \titlepage
%% \end{frame}
%% 
%% % XXXXXXXXXXXXXXXXXXXXXXXXXXXXXXXXXXXXXXXXXXXXXXXXXXXXXXXXXXXXXXXXXXXXXXXXXX
%% \section{Agenda}
%% \begin{frame}
%% \frametitle{Agenda}
%% % Throughout your presentation, if you choose to use \section{} and 
%% % \subsection{} commands, these will automatically be printed on 
%% % this slide as an overview of your presentation
%% \tableofcontents 
%% \end{frame}
%% 
%% % XXXXXXXXXXXXXXXXXXXXXXXXXXXXXXXXXXXXXXXXXXXXXXXXXXXXXXXXXXXXXXXXXXXXXXXXXX
%% \section{UUD dan Pancasila}
%% \subsection{UUD}
%% \begin{frame}
%% \frametitle{Pembukaan}
%% Bahwa sesungguhnya kemerdekaan itu ialah hak segala bangsa dan oleh 
%% sebab itu, maka penjajahan diatas dunia harus dihapuskan karena 
%% tidak sesuai dengan perikemanusiaan dan perikeadilan.
%% \\~\\
%% Atas berkat rahmat Allah Yang Maha Kuasa dan dengan didorongkan oleh 
%% keinginan luhur, supaya berkehidupan kebangsaan yang bebas, maka 
%% rakyat Indonesia menyatakan dengan ini kemerdekaannya.
%% \end{frame}
%% 
%% % XXXXXXXXXXXXXXXXXXXXXXXXXXXXXXXXXXXXXXXXXXXXXXXXXXXXXXXXXXXXXXXXXXXXXXXXXX
%% \begin{frame}
%% \frametitle{Alenia Ketiga}
%% Kemudian daripada itu untuk membentuk suatu pemerintah negara Indonesia 
%% yang melindungi segenap bangsa Indonesia dan seluruh tumpah darah Indonesia 
%% dan untuk memajukan kesejahteraan umum, mencerdaskan kehidupan bangsa, dan 
%% ikut melaksanakan ketertiban dunia yang berdasarkan kemerdekaan, perdamaian 
%% abadi dan keadilan sosial, maka disusunlah kemerdekaan kebangsaan Indonesia 
%% itu dalam suatu Undang-Undang Dasar negara Indonesia, yang terbentuk dalam 
%% suatu susunan negara Republik Indonesia yang berkedaulatan rakyat dengan 
%% berdasar kepada:
%% \begin{itemize}
%% \item Ketuhanan Yang Maha Esa,
%% \item kemanusiaan yang adil dan beradab,
%% \item persatuan Indonesia,
%% \item dan kerakyatan yang dipimpin oleh hikmat kebijaksanaan 
%%       dalam permusyawaratan/ perwakilan,
%% \item serta dengan mewujudkan suatu keadilan sosial bagi seluruh rakyat 
%%       Indonesia.
%% \end{itemize}
%% \end{frame}
%% 
%% % XXXXXXXXXXXXXXXXXXXXXXXXXXXXXXXXXXXXXXXXXXXXXXXXXXXXXXXXXXXXXXXXXXXXXXXXXX
%% \subsection{Pancasila}
%% \begin{frame}
%% \frametitle{Tujuh Kunci Pokok}
%% \begin{block}{Pertama - Kedua - Ketiga}
%% Indonesia ialah negara berdasarkan hukum.
%% Sistem konstitusional.
%% Kekuasaan negara tertinggi di tangan MPR.
%% \end{block}
%% 
%% \begin{block}{Keempat - Kelima}
%% Presiden adalah penyelenggara pemerintahan tertinggi di bawah MPR.
%% Adanya pengawasan DPR.
%% \end{block}
%% 
%% \begin{block}{Keenam}
%% Menteri negara adalah pembantu presiden dan tidak bertanggung jawab 
%% kepada DPR.
%% \end{block}
%% 
%% \begin{block}{Ketujuh}
%% Kekuasaan kepala negara tidak tak tebatas.
%% \end{block}
%% 
%% \end{frame}
%% 
%% % XXXXXXXXXXXXXXXXXXXXXXXXXXXXXXXXXXXXXXXXXXXXXXXXXXXXXXXXXXXXXXXXXXXXXXXXXX
%% \section{Rupa-rupa}
%% \subsection{Kolom}
%% \begin{frame}
%% \frametitle{Kolom}
%% % The "c" option specifies centered vertical alignment 
%% % while the "t" option is used for top vertical alignment
%% \begin{columns}[c] 
%% % Left column and width
%% \column{.45\textwidth} 
%% \textbf{Heading}
%% \begin{enumerate}
%% \item Satu-satu
%% \item Dua-dua
%% \item Tiga-tiga
%% \item Satu-dua-tiga
%% \end{enumerate}
%% 
%% % Right column and width
%% \column{.5\textwidth}
%% Satu-satu~\dots{} aku sayang ibu!
%% Dua-dua~\ldots{} juga sayang ayah!
%% Tiga-tiga~\ldots{} sayang adik kakak!
%% Satu-dua-tiga~\ldots{} sayang semuanya!
%% 
%% \end{columns}
%% \end{frame}
%% 
%% % XXXXXXXXXXXXXXXXXXXXXXXXXXXXXXXXXXXXXXXXXXXXXXXXXXXXXXXXXXXXXXXXXXXXXXXXXX
%% \subsection{Tabel}
%% \begin{frame}
%% \frametitle{Tabel}
%% \begin{table}
%% \begin{tabular}{l l l}
%% \toprule
%% \textbf{Nama} & \textbf{NPM} & \textbf{Tanggal Lahir}\\
%% \midrule
%% Cecak bin Kadal & 1234567890 & 1 Jan 2015 \\
%% Aneh bin Ajaib  & 0987654321 & 31 Des 2014 \\
%% \bottomrule
%% \end{tabular}
%% \caption{Keterangan Tabel}
%% \end{table}
%% \end{frame}
%% 
%% % XXXXXXXXXXXXXXXXXXXXXXXXXXXXXXXXXXXXXXXXXXXXXXXXXXXXXXXXXXXXXXXXXXXXXXXXXX
%% \subsection{Teori}
%% \begin{frame}
%% \frametitle{Teori}
%% \begin{theorem}[Teori Satu Batu]
%% $E = mc^2$
%% \end{theorem}
%% \end{frame}
%% 
%% % XXXXXXXXXXXXXXXXXXXXXXXXXXXXXXXXXXXXXXXXXXXXXXXXXXXXXXXXXXXXXXXXXXXXXXXXXX
%% \subsection{Verbatim}
%% % Need to use the fragile option when verbatim is used in the slide
%% \begin{frame}[fragile] 
%% \frametitle{Verbatim}
%% \begin{example}[Teori Satu Batu]
%% \begin{verbatim}
%% \begin{theorem}[Teori Satu Batu]
%% $E = mc^2$
%% \end{theorem}
%% \end{verbatim}
%% \end{example}
%% \end{frame}
%% 
%% % XXXXXXXXXXXXXXXXXXXXXXXXXXXXXXXXXXXXXXXXXXXXXXXXXXXXXXXXXXXXXXXXXXXXXXXXXX
%% \subsection{Gambar}
%% \begin{frame}
%% \frametitle{Gambar}
%% \begin{figure}
%% \includegraphics[width=0.5\linewidth]{2}
%% \caption{Ini Gambar JPG}
%% \end{figure}
%% \end{frame}
%% 
%% % XXXXXXXXXXXXXXXXXXXXXXXXXXXXXXXXXXXXXXXXXXXXXXXXXXXXXXXXXXXXXXXXXXXXXXXXXX
%% \subsection{Rujukan}
%% % Need to use the fragile option when verbatim is used in the slide
%% \begin{frame}[fragile] 
%% \frametitle{Rujukan dan Kutipan}
%% Contoh penggunaan \verb|\cite| ketika mengutip\cite{p1}.
%% Perhatian: Beamer tidak mengerti \verb|\BibTeX|~\ldots
%% \footnotesize{
%%   \begin{thebibliography}{99} 
%%   \bibitem[Smith, 2012]{p1} John Smith (2012)
%%      \newblock Katak dalam Tempurung
%%      \newblock \emph{Jurnal Kelapa dan Amfibi} 12(3), 45 -- 678.
%%   \end{thebibliography}
%% }
%% \end{frame}
%% 
%% % XXXXXXXXXXXXXXXXXXXXXXXXXXXXXXXXXXXXXXXXXXXXXXXXXXXXXXXXXXXXXXXXXXXXXXXXXX
%% \subsection{Selesai}
%% \begin{frame}
%% \Huge{\centerline{Selesai}}
%% \end{frame}
%% 
%% % XXXXXXXXXXXXXXXXXXXXXXXXXXXXXXXXXXXXXXXXXXXXXXXXXXXXXXXXXXXXXXXXXXXXXXXXXX
%% \end{document}

\newcommand{\revision}{%
REV424: Tue 03 Sep 2024 20:00
}
% w! tmptmp
% REV424: Tue 03 Sep 2024 20:00
% REV419: Wed 24 Jul 2024 17:00
% REV409: Tue 08 Aug 2023 12:00
% REV399: Fri 03 Feb 2023 20:00
% REV339: Sat 04 Sep 2021 12:00
% STARTS: Wed 24 Aug 2016 19:00
%%%%%%%%%%%%%%%%%%%%%%%%%%%%%%%%%%%%%
\newcommand{\kopikopi}{\textcopyright{}2016-2024 CBKadal + VauLSMorg}



% XXXXXXXXXXXXXXXXXXXXXXXXXXXXXXXXXXXXXXXXXXXXXXXXXXXXXXXXXXXXXXXXXXXXXXXXXX
% The short title appears at the bottom of every slide, 
% the full title is only on the title page
% \date{\today}
\title[\kopikopi]
{CSGE602055 Operating Systems \\ 
CSF2600505 Sistem Operasi \\
Week 01:
Overview 2, Virtualization \& Scripting}
\author{C. BinKadal}
\institute[SdnBhd]
{
Sendirian Berhad\\
\medskip
\url{https://docOS.vlsm.org/Slides/os01.pdf}
\\ \texttt{Always check for the latest revision!}
}
\date{\revision}

% XXXXXXXXXXXXXXXXXXXXXXXXXXXXXXXXXXXXXXXXXXXXXXXXXXXXXXXXXXXXXXXXXXXXXXXXXX
\begin{document}

\lstset{
basicstyle=\ttfamily\tiny, % \tiny \small \footnotesize 
breakatwhitespace=true,
language=C,
columns=fullflexible,
keepspaces=true,
breaklines=true,
tabsize=3, 
showstringspaces=false,
extendedchars=true}

\section{Start}
\begin{frame}
\titlepage
\end{frame}

% XXXXXXXXXXXXXXXXXXXXXXXXXXXXXXXXXXXXXXXXXXXXXXXXXXXXXXXXXXXXXXXXXXXXXXXXXX

%%%%%%%%%%%%%%%%%%%%%%%%%%%%%%%%%%%%%%%%%%%%%%%%%%%%%%%%%%%%%%%%%%%%%%%%%
% REV418: Tue 30 Jan 2024 22:00
% REV406: Sat 05 Aug 2023 14:00
% REV399: Thu 02 Feb 2023 00:00
% REV369: Mon 14 Feb 2022 09:00
% REV328: Sat 14 Aug 2021 06:00
% STARTX: Wed 14 Sep 2016 10:00
%%%%%%%%%%%%%%%%%%%%%%%%%%%%%%%%%%%%%%%%%%%%%%%%%%%%%%%%%%%%%%%%%%%%%%%%%

\begin{frame}[fragile]
\section{OS241 Schedule}
\frametitle{OS241\footnote{%
) This information will be on \textbf{EVERY} page two (2) of this course material.}): 
Operating Systems Schedule 2023 - 2}

\vspace{5pt}

\scalebox{0.99}{%
\begin{tabular}{|c|c|l|l|}
\hline
\textbf{Week} & 
\textbf{Topic}\footnote{%
) For schedule, see \url{https://os.vlsm.org/\#idx02}}) & \textbf{OSC10}\footnote{%
    ) Silberschatz et. al.: \textbf{Operating System Concepts}, $10^{th}$ Edition, 2018.}) \\
\hline
Week 00  & Overview (1), Assignment of Week 00           & Ch. 1, 2      \\
Week 01  & Overview (2), Virtualization \& Scripting     & Ch. 1, 2, 18. \\
Week 02  & Security, Protection, Privacy, \& C-language. & Ch. 16, 17.   \\
Week 03  & File System \& FUSE  & Ch. 13, 14, 15.                        \\
Week 04  & Addressing, Shared Lib, \& Pointer & Ch. 9. \\
Week 05  & Virtual Memory & Ch. 10. \\
\hline
Week 06  & Concurrency: Processes \& Threads & Ch. 3, 4. \\
Week 07  & Synchronization \& Deadlock & Ch. 6, 7, 8. \\
Week 08  & Scheduling + W06/W07 & Ch. 5. \\
Week 09  & Storage, Firmware, Bootloader, \& Systemd & Ch. 11. \\
Week 10  & I/O \& Programming & Ch. 12. \\%
% MidTerm  & 00 XXX 2020 (XX:XX-XX:XX) & MidTerm (UTS) & \cellcolor{red!44} TBA! \\
% Reserved & 00 XXX - 00 XXX 2020 & Q \& A & \\
% Final    & 00 XXX 2020 XX:XX & First Part Final  (UAS tahap I)  & \cellcolor{red!44} This schedule is   \\
% Extra    & NA & No Extra assignment & \cellcolor{red!44} subject to change. \\
\hline
\end{tabular}
}
\end{frame}

\begin{frame}[fragile]
\frametitle{\textbf{STARTING POINT} --- 
{
\definecolor{links}{HTML}{FDEE00}
\hypersetup{colorlinks,linkcolor=,urlcolor=links}
\url{https://os.vlsm.org/}
}
}
\begin{itemize}
\item[$\square$] \textbf{Text Book} ---
                 Any recent/decent OS book. Eg. (\textbf{OSC10}) Silberschatz et. al.: 
                 \textbf{Operating System Concepts}, $10^{th}$ Edition, 2018.
                 (See \url{https://codex.cs.yale.edu/avi/os-book/OS10/}).
\item[$\square$] \textbf{Resources ({\footnotesize \url{https://os.vlsm.org/\#idx03}})}
\begin{itemize}
\item[$\square$] \href{https://scele.cs.ui.ac.id/course/view.php?id=3743}{\textbf{SCELE}} ---
\url{https://scele.cs.ui.ac.id/course/view.php?id=3743}.\\
The enrollment key is \textbf{XXX}.
\item[$\square$] \textbf{Download Slides and Demos from GitHub.com} --- (\url{https://github.com/os2xx/docOS/})\\
                 {\scriptsize%
                 \href{https://docOS.vlsm.org/Slides/os00.pdf}{\texttt{os00.pdf} (W00)},
                 \href{https://docOS.vlsm.org/Slides/os01.pdf}{\texttt{os01.pdf} (W01)},
                 \href{https://docOS.vlsm.org/Slides/os02.pdf}{\texttt{os02.pdf} (W02)},
                 \href{https://docOS.vlsm.org/Slides/os03.pdf}{\texttt{os03.pdf} (W03)},
                 \href{https://docOS.vlsm.org/Slides/os04.pdf}{\texttt{os04.pdf} (W04)},
                 \href{https://docOS.vlsm.org/Slides/os05.pdf}{\texttt{os05.pdf} (W05)},\\
                 \href{https://docOS.vlsm.org/Slides/os06.pdf}{\texttt{os06.pdf} (W06)},
                 \href{https://docOS.vlsm.org/Slides/os07.pdf}{\texttt{os07.pdf} (W07)},
                 \href{https://docOS.vlsm.org/Slides/os08.pdf}{\texttt{os08.pdf} (W08)},
                 \href{https://docOS.vlsm.org/Slides/os09.pdf}{\texttt{os09.pdf} (W09)},
                 \href{https://docOS.vlsm.org/Slides/os10.pdf}{\texttt{os10.pdf} (W10)}.
                 }
\item[$\square$] \textbf{Problems}\\
                 {\scriptsize% 
                 \href{https://rms46.vlsm.org/2/195.pdf}{\texttt{195.pdf} (W00)},
                 \href{https://rms46.vlsm.org/2/196.pdf}{\texttt{196.pdf} (W01)},
                 \href{https://rms46.vlsm.org/2/197.pdf}{\texttt{197.pdf} (W02)},
                 \href{https://rms46.vlsm.org/2/198.pdf}{\texttt{198.pdf} (W03)},
                 \href{https://rms46.vlsm.org/2/199.pdf}{\texttt{199.pdf} (W04)},
                 \href{https://rms46.vlsm.org/2/200.pdf}{\texttt{200.pdf} (W05)},\\
                 \href{https://rms46.vlsm.org/2/201.pdf}{\texttt{201.pdf} (W06)},
                 \href{https://rms46.vlsm.org/2/202.pdf}{\texttt{202.pdf} (W07)},
                 \href{https://rms46.vlsm.org/2/203.pdf}{\texttt{203.pdf} (W08)},
                 \href{https://rms46.vlsm.org/2/204.pdf}{\texttt{204.pdf} (W09)},
                 \href{https://rms46.vlsm.org/2/205.pdf}{\texttt{205.pdf} (W10)}.}
\item[$\square$] \textbf{LFS} --- \url{http://www.linuxfromscratch.org/lfs/view/stable/}
\item[$\square$] \textbf{OSP4DISS} --- \url{https://osp4diss.vlsm.org/}
\item[$\square$] \textbf{This is How Me Do It!} --- \url{https://doit.vlsm.org/}
\begin{itemize}
\item[$\square$] PS: "Me" rhymes better than "I", duh!
\end{itemize}
\end{itemize}
\end{itemize}
\end{frame}



% XXXXXXXXXXXXXXXXXXXXXXXXXXXXXXXXXXXXXXXXXXXXXXXXXXXXXXXXXXXXXXXXXXXXXXXXXX
% Throughout your presentation, if you choose to use \section{} and 
% \subsection{} commands, these will automatically be printed on 
% this slide as an overview of your presentation
\section{Agenda}
\begin{frame}{Outline}
  \frametitle{Agenda}
  \tableofcontents[sections={1-}]
\end{frame}
% \begin{frame}
%    \frametitle{Agenda (2)}
%    \tableofcontents[sections={12-}]
% \end{frame}

% XXXXXXXXXXXXXXXXXXXXXXXXXXXXXXXXXXXXXXXXXXXXXXXXXXXXXXXXXXXXXXXXXXXXXXXXXX

\input{os01-BRP.tex}

% XXXXXXXXXXXXXXXXXXXXXXXXXXXXXXXXXXXXXXXXXXXXXXXXXXXXXXXXXXXXXXXXXXXXXXX
\section{OSC10 (Silberschatz) Chapter 18: Virtual Machines}
\begin{frame}[fragile]
\frametitle{OSC10 (Silberschatz) Chapter 18: Virtual Machines}
\begin{itemize}
  \item OSC10 Chapter 18
  \begin{itemize}
    \item Overview
    \item History
    \item Benefits and Features
    \item Building Blocks
    \item Types of Virtual Machines and Their Implementations
    \item Virtualization and Operating-System Components
    \item Examples
  \end{itemize}
  \vfill \null
\end{itemize}
\vfill \null
\end{frame}

% XXXXXXXXXXXXXXXXXXXXXXXXXXXXXXXXXXXXXXXXXXXXXXXXXXXXXXXXXXXXXXXXXXXXXXX
\section{What defines an Operating System? (The Three Layers Model)}
\begin{frame}[fragile]
\frametitle{What defines an Operating System? (The Three Layers Model)}
\begin{multicols}{2}
\begin{table}
\scalebox{0.8}{%
\begin{tabular}{| c |}
\hline \\ [1pt]
Business Goal \\
\vline \\ [1pt]
Application \\
\vline \\ [1pt]
\hline
OS API \\
\vline \\ [1pt]
OS Managers and Utilities \\
\vline \\ [1pt]
OS Drivers \\
\hline
\vline \\ [1pt]
(Hypervisor) \\
\vline \\ [1pt]
Hardware \\ [1pt]
\hline
\end{tabular}}
\end{table}
  \vfill \null
\columnbreak
  \begin{itemize}
    \item The Three Layers Model
  \begin{itemize}
    \item An Operating System is between your Application and your Hardware (or Hypervisor).
  \begin{itemize}
    \item OS API: Application Programming Interface
    \item OS Resources Managers and Utilities: Process, Scheduler, Dispatcher,
             (Virtual) Memory, Disk, I/O, Network, Security, Protection, etc.
    \item OS Device Drivers: controls devices
  \end{itemize}
    \item Remember that your future "\textbf{Business Goal}" may not directly relate to an Operating System at all!
  \end{itemize}
  \end{itemize}
  \vfill \null
\end{multicols}
\end{frame}

% XXXXXXXXXXXXXXXXXXXXXXXXXXXXXXXXXXXXXXXXXXXXXXXXXXXXXXXXXXXXXXXXXXXXXXX

\begin{frame}
\frametitle{Week 01: Review 2}

\begin{itemize}
\item Intellectual Property Rights (IPR)
\item Richard Stallman: Introduction to Free Software
\begin{itemize}
\item YouTube: \url{https://youtu.be/Ag1AKIl_2GM} 
      (\href{https://www.fsf.org/blogs/rms/20140407-geneva-tedx-talk-free-software-free-society}{+article}).
\item See also \url{https://rms46.vlsm.org/1/70.pdf}
\end{itemize}
\item Operating System Services
\item User Operating System Interface
\item System Calls
\item Types of System Calls
\item System Programs
\item Operating System Design and Implementation
\item Operating System Structure
\item \href{https://osp4diss.vlsm.org/Welcome2GNULinux.html}{Introduction to GNU/Linux}.
\item \href{https://osp4diss.vlsm.org/osp-115.html}{More Operating Systems}.
\end{itemize}
\end{frame}

% XXXXXXXXXXXXXXXXXXXXXXXXXXXXXXXXXXXXXXXXXXXXXXXXXXXXXXXXXXXXXXXXXXXXXXX
\begin{frame}
\frametitle{Intelectual Property Rights (IPR)}
\begin{itemize}
\item Trade Secret (Rahasia Dagang) --- UU no. 30/2000.
\item Industrial Design (Desain Industri) --- UU no. 31/2000.
\item Integrated Circuit Layout Design (Desain Tata Letak Sirkuit Terpadu) --- UU no. 32/2000.
\item Patent (Paten) --- UU no. 14/2001.
\item Copyright (Hak Cipta) --- UU no. 19/2002.
\item The problem of Intellectual Property Rights (IPR).
\item Software IPR.
\item Software Licenses: GNU GPL, EULA, Public Domain, Apache, Microsoft Public License.
\end{itemize}
\end{frame}

% XXXXXXXXXXXXXXXXXXXXXXXXXXXXXXXXXXXXXXXXXXXXXXXXXXXXXXXXXXXXXXXXXXXXXXXXXX
\begin{frame}
\frametitle{Is this a Software Patent or Not?}
\begin{figure}
\includegraphics[width=0.75\linewidth]{os01-avw}
\end{figure}
\end{frame}

% XXXXXXXXXXXXXXXXXXXXXXXXXXXXXXXXXXXXXXXXXXXXXXXXXXXXXXXXXXXXXXXXXXXXXXXXXX
\begin{frame}
\frametitle{The Codec Mess}
\begin{figure}
\includegraphics[width=0.75\linewidth]{os01-avx}
\end{figure}
\end{frame}

% XXXXXXXXXXXXXXXXXXXXXXXXXXXXXXXXXXXXXXXXXXXXXXXXXXXXXXXXXXXXXXXXXXXXXXXXXX
\begin{frame}
\frametitle{Alliance for Open Media}
\begin{figure}
\includegraphics[width=0.99\linewidth]{os01-avy}
\end{figure}
{\small Source (per 24-Aug-2023): \url{https://aomedia.org/membership/members/}}
\end{frame}

% XXXXXXXXXXXXXXXXXXXXXXXXXXXXXXXXXXXXXXXXXXXXXXXXXXXXXXXXXXXXXXXXXXXXXXXX
\section{Free Software}
\begin{frame}
\frametitle{Free Software}
\begin{itemize}
\item Free Software Definition (FSF)
\begin{enumerate}
\setcounter{enumi}{-1}
\item The freedom to run the program as you wish, for any purpose (freedom 0).
\item The freedom to study how the program works and change it does your computing 
      as you wish (freedom 1). Access to the source code is a precondition for this.
\item The freedom to redistribute copies so you can help your neighbor (freedom 2).
\item The freedom to distribute copies of your modified versions to others (freedom 3).
      By doing this, you can give the whole community a chance to benefit from your changes.
      Access to the source code is a precondition for this.
\end{enumerate}
\item Free Software vs. Open Source Software.
\item Copyleft Software.
\end{itemize}
\end{frame}

% XXXXXXXXXXXXXXXXXXXXXXXXXXXXXXXXXXXXXXXXXXXXXXXXXXXXXXXXXXXXXXXXXXXXXXXX
\section{Software Licenses}
\begin{frame}
\frametitle{Software Licenses}
\begin{itemize}
\item 3-clause BSD license and 2-clause BSD license (BSD-X-Clause) 
\item Apache License 2.0 (Apache-2.0)
\item Artistic License 2.0 (ArtisticLicense2)
\item Common Development and Distribution License (CDDL-1.0)
\item Eclipse Public License (EPL-1.0)
\item Educational Community License 2.0 (ECL2.0)
\item Expat License (Expat) aka.  MIT license (MIT) 
\item GNU Affero General Public License v3 (AGPL-3.0)
\item GNU All-Permissive License (GNUAllPermissive)
\item GNU General Public License (GPL)
\item GNU Lesser General Public License (LGPL)
\item Microsoft Public License (MS-PL)
\item Mozilla Public License 2.0 (MPL-2.0)
\item ''Public Domain'' (PublicDomain)
\item X11 License (X11License)
\end{itemize}
\end{frame}

% XXXXXXXXXXXXXXXXXXXXXXXXXXXXXXXXXXXXXXXXXXXXXXXXXXXXXXXXXXXXXXXXXXXXXXXX
\section{Virtualization \& Cloud Computing}
\begin{frame}
\frametitle{Virtualization \& Cloud Computing}
\begin{itemize}
\item Virtual Machine
\begin{itemize}
\item Host \& Guest
\item Hypervisor (Virtual Machine Manager)
\begin{itemize}
\item Type 0, 1, 2 Hypervisor
\item ParaVirtualization
\item Programming-environment Virtualization
\item Emulators
\end{itemize}
\item Application Containment (OS-Level)
\begin{itemize}
\item Containers: LXC, Solaris Containers, Docker.
\item Zones: Solaris Containers
\item Virtual Private Servers: OpenVZ
\item Virtual Kernels: DragonFly BSD
\item Jails: FreeBSD Jail/ Chroot Jail
\end{itemize}
\item Kubernetes (K8s): A (open source) system for managing CONTAINERIZED applications.
\end{itemize}
\item Cloud Computing
\begin{itemize}
\item SAAS: Software As A Service.
\item PAAS: Platform As A Service.
\item IAAS: Infrastructure As A Service.
\end{itemize}
\end{itemize}
\end{frame}

% XXXXXXXXXXXXXXXXXXXXXXXXXXXXXXXXXXXXXXXXXXXXXXXXXXXXXXXXXXXXXXXXXXXXXXXXXX
\section{Potpourri}
\begin{frame}
\frametitle{Potpourri}
\begin{itemize}
\item Mobile/Distributed/Client-Server/Peer-to-Peer Computing.
\item Real-Time Computing: Hard Real-Time vs. Soft Real-Time.
\item Operating System Comparison: 
Android, 
*BSD,
GNU/Linux, 
iOS, 
Mac OS, 
Windows.
\item Operating System Services: UI (GUI, CLI); Program Executing; I/O Operations; 
      File Systems Manipulation; Communication; Error Detection; Resource Allocation;
      Accounting; Protection \& Security.
\item System Calls: Process Control; File Management; Device Management; Information
      Maintenance; Communications; Protection.
\item Application Programming Interface (API)
\item Standard C Library.
\item System Programs.
\item Microkernel System Structure.
\item Loadable Kernel Modules.
\item Virtualization and Cloud System.
\end{itemize}
\end{frame}

% XXXXXXXXXXXXXXXXXXXXXXXXXXXXXXXXXXXXXXXXXXXXXXXXXXXXXXXXXXXXXXXXXXXXXXXX
\section{Some Essential Command Lines}
\begin{frame}
\frametitle{Some Essential Command Lines (1)}
\begin{tabular}{l l}
\hline
man    & manual. E.g., \texttt{''man man''}                                       \\
passwd & changes passwords.                                                       \\
ls     & list directory contents.  E.g., \texttt{''ls -al''}                      \\
cd     & change the working directory. E.g., \texttt{''cd /tmp''}                 \\
cp     & copy file(s). E.g., \texttt{''cp SOURCE DEST''}                          \\
rm     & remove file(s).  E.g., \texttt{''rm AFILE''}                             \\
mv     & move files(s).        E.g., \texttt{''mv FROMFILE TOFILE''}              \\
mkdir  & make directories(s).        E.g., \texttt{''mkdir ADIRECTORY''}          \\
rmdir  & remove directories(s).        E.g., \texttt{''rmdir ADIRECTORY''}        \\
cat    & read file(s)      E.g., \texttt{''cat AFILE''}                           \\
more   & read file(s) per screen      E.g., \texttt{''more AFILE''}               \\
ln     & make a link of a file. E.g., \texttt{''ln -s file sfile''}               \\
grep   & search string ''aword'' inside file.  E.g., \texttt{''grep aword file''} \\
sort   & sort lines of text files. E.g., \texttt{''sort file1.txt''}              \\
top    & display systems task.  E.g., \texttt{''top''}                            \\
\hline
\end{tabular}
\end{frame}

% XXXXXXXXXXXXXXXXXXXXXXXXXXXXXXXXXXXXXXXXXXXXXXXXXXXXXXXXXXXXXXXXXXXXXXXX
\begin{frame}
\frametitle{Some Essential Command Lines (2)}
\begin{tabular}{l l}
\hline
find   & E.g., \texttt{''find / -name minix3.iso -print''}. Find from ''/''.      \\
chmod  & E.g. \texttt{''chmod 755 file''}. Change file with access mode 755.    \\
chown  & E.g. \texttt{''chown user file''}. Change owner file to user.          \\
chgrp  & E.g. \texttt{''chgrp other file''}. Change group file to other.        \\
tar    & tape archive file. E.g.                                                \\
       & \texttt{''tar cf /tmp/tfile.tar  dir/''}. Archive ''dir/'' into tfile.tar. \\
       & \texttt{''tar tf /tmp/tfile.tar''}. List tfile.tar.                   \\
       & \texttt{''tar xf /tmp/tfile.tar''}. Extract tfile.tar.                \\
%      E.g. \texttt{''''}  \\
%      E.g. \texttt{''''}  \\
%      E.g. \texttt{''''}  \\
%      E.g. \texttt{''''}  \\
%      E.g. \texttt{''''}  \\
date   & print or set the system date and time. E.g. \texttt{''date +\%Y''}     \\
tee    & read from standard input and write to standard output and files.      \\
       & E.g. \texttt{''ls -al | tee listing.txt''}                             \\
diff   & compare files line by line. E.g. \texttt{''diff file1.txt file2.txt''} \\
wc     & print newline, word, and byte counts for each file.  E.g. \texttt{''wc file.txt''}  \\
\hline
\end{tabular}

\begin{itemize}
\item See also:
\begin{itemize}
\item \href{https://www.youtube.com/watch?v=CpTfQ-q6MPU}{10 Linux Terminal Commands for Beginners}
\item \href{https://linuxopsys.com/topics/basic-linux-commands}{Basic Linux Commands You Should Know}
\end{itemize}
\end{itemize}
\end{frame}

% XXXXXXXXXXXXXXXXXXXXXXXXXXXXXXXXXXXXXXXXXXXXXXXXXXXXXXXXXXXXXXXXXXXXXXXX
\section{The ''vi'' editor}
\begin{frame}
\frametitle{The ''vi'' editor}
\begin{itemize}
\item VI Basics\\[9pt]
\scalebox{0.75}{%
\begin{tabular}{ll||ll}
  & Basics                               &             & More Commands \\
\hline
i         & insert mode                  & d\^{}       & delete from \^{} (beginning) to the cursor   \\
a         & append mode                  & d\${}       & delete from the cursor to \${} (end)         \\
$<$ESC$>$ & escape mode                  & dd          & delete the whole line                        \\
q!        & quit                         & 5dd         & delete 5 lines                               \\
wq!       & write and quit               & yy          & yank (copy) the line                         \\
ZZ        & write and quit               & p           & put (paste) the line                         \\
h j k l   & move [left, down, up, right] & J           & joint current and next line                  \\
r         & replace a character          & :r file.txt & read (insert) file.txt                       \\
d         & delete  a character          & :w! file.txt & write into file.txt \\
u         & undo                         & :1,8 w! file.txt & write line 1 to 8 into file.txt \\
\hline
\end{tabular}
}\\[9pt]
\item See also:
\begin{itemize}
\item Basic vi Commands \url{https://www.cs.colostate.edu/helpdocs/vi.html}
\item Vim Basics in 8 Minutes \url{https://youtu.be/ggSyF1SVFr4}
\end{itemize}
\end{itemize}

\end{frame}

% XXXXXXXXXXXXXXXXXXXXXXXXXXXXXXXXXXXXXXXXXXXXXXXXXXXXXXXXXXXXXXXXXXXXXXXX
\section{More man, awk, regex, sed}
\begin{frame}[fragile]
\frametitle{More man, awk, regex, sed (1/7)}
\begin{itemize}
\item \textbf{RTFM: Read The Fine Manual!}
\begin{itemize}
\item Linux Man Pages: A Quick Tutorial (\url{https://youtu.be/uJnrh9hAQR0})
\end{itemize}
\item awk
\begin{itemize}
\item GNU AWK (\url{https://www.gnu.org/software/gawk/manual/gawk.pdf})
\item Learning Awk Is Essential For Linux Users (\url{https://youtu.be/9YOZmI-zWok})
\begin{itemize}
\item \texttt{awk '\{print ''Hello awk!''\}' file.txt} --- print ''Hello awk!'' for every file.txt line.
\item \texttt{awk '\{print \$0\}' file.txt} --- print every file.txt line.
\item \texttt{awk '\{print \$1\}' file.txt} --- print first field of every file.txt line.
\item \texttt{awk '\{print \$2\}' file.txt} --- print second field of every file.txt line.
\end{itemize}
\end{itemize}
\item regex
\begin{itemize}
\item Regular Expressions (\url{https://youtu.be/bgBWp9EIlMM})
\item to search patterns
\item BRE (Basic Regular Expression) vs ERE (Extended Regular Expression)
\item Flavors: Grep, Java, JavaScript, PHP, POSIX, Python, sed, XML, \ldots
\end{itemize}
\end{itemize}
\end{frame}

% ZCZC
% XXXXXXXXXXXXXXXXXXXXXXXXXXXXXXXXXXXXXXXXXXXXXXXXXXXXXXXXXXXXXXXXXXXXXXXX
\begin{frame}[fragile]
\frametitle{More man, awk, regex, sed (2/7)}
\begin{itemize}
\item $\ll$\texttt{\^{}\${}}$\gg$ --- matches a beginning-of-line + end-of-line (empty line).
\begin{itemize}
\item $\ll$\texttt{\^{}}$\gg$ --- matches a beginning-of-line (meaningless).
\item $\ll$\texttt{\^{}hello\${}}$\gg$ --- matches just ''hello'' in a line.
\end{itemize}
\item $\ll${\tiny{}\textbullet}$\gg$ --- matches any character.
\begin{itemize}
\item $\ll$\texttt{hell.}$\gg$ --- matches ''hellA'', ''hella'', ''hellB'', ''hellb'', \ldots
\end{itemize}
\item $\ll$\texttt{[AB]}$\gg$ --- matches ''A'' or ''B'' only.
\begin{itemize}
\item $\ll$\texttt{[0-3]}$\gg$ --- matches ''0'', ''1'', ''2'', or ''3'' only.
\item $\ll$\texttt{[\^{}4-9]}$\gg$ --- not match ''4'', ''5'', ''6'', ''7'', ''8'', or ''9''.
\end{itemize}
\item $\ll$\texttt{?}$\gg$ --- matches preceding zero or one time.
\begin{itemize}
\item $\ll$\texttt{a?b}$\gg$ --- matches ''b'' or ''ab'' only.
\end{itemize}
\item $\ll$\texttt{*}$\gg$ --- matches preceding zero or more times.
\begin{itemize}
\item $\ll$\texttt{a*b}$\gg$ --- matches ''b'' or ''ab'' or ''aab'' or \ldots
\item $\ll$\texttt{A.*Z}$\gg$ --- matches ''AZ'' or ''AaZ'' or ''AabZ'' or \ldots
\end{itemize}
\item $\ll$\texttt{+}$\gg$ --- matches preceding one or more times.
\begin{itemize}
\item $\ll$\texttt{a+b}$\gg$ --- matches ''ab'', ''aab'', ''aaab'', \ldots
\end{itemize}
\end{itemize}
\end{frame}

% XXXXXXXXXXXXXXXXXXXXXXXXXXXXXXXXXXXXXXXXXXXXXXXXXXXXXXXXXXXXXXXXXXXXXXXX
\begin{frame}[fragile]
\frametitle{More man, awk, regex, sed (3/7)}
\begin{itemize}
\item $\ll$\texttt{\{\}}$\gg$ --- matches numbers in \{\}.
\begin{itemize}
\item $\ll$\texttt{a\{2\}}$\gg$ --- matches ''aa''.
\item $\ll$\texttt{a\{2,5\}}$\gg$ --- matches ''aa'', ''aaa'', ''aaaa'', and ''aaaaa''.
\item $\ll$\texttt{a\{2,\}}$\gg$ --- matches ''aa'', ''aaa'', ''aaaa'', ''aaaaa'', \ldots
\end{itemize}
\item $\ll$\texttt{\textbackslash}$\gg$ --- escape character.
\item $\ll$\texttt{\textbackslash}0$\gg$ --- NULL.
\item $\ll$\texttt{\textbackslash}b$\gg$ --- word boundary.
\item $\ll$\texttt{\textbackslash}B$\gg$ --- non-word boundary.
\item $\ll$\texttt{\textbackslash}d$\gg$ --- any digit. E.g. $\ll$\texttt{\textbackslash}d\{1,3\}$\gg$
      = \texttt{0 -- 999}.
\item $\ll$\texttt{\textbackslash}D$\gg$ --- any non-digit.
\item $\ll$\texttt{\textbackslash}n$\gg$ --- new line.
\item $\ll$\texttt{\textbackslash}t$\gg$ --- tab.
\item $\ll$\texttt{\textbackslash}s$\gg$ --- white space character.
\item $\ll$\texttt{\textbackslash}S$\gg$ --- non white space character.
\end{itemize}
\end{frame}

% XXXXXXXXXXXXXXXXXXXXXXXXXXXXXXXXXXXXXXXXXXXXXXXXXXXXXXXXXXXXXXXXXXXXXXXX
\begin{frame}[fragile]
\frametitle{More man, awk, regex, sed (4/7)}
\begin{itemize}
\item $\ll$\texttt{(\ldots)}$\gg$ --- group.
\begin{itemize}
\item $\ll$\texttt{(?:\ldots)}$\gg$ --- passive group.
\item $\ll$\texttt{(regex)$\mid$(regex)}$\gg$ --- matches left regex or right regex.
\item $\ll$\texttt{(a\textbar{}b}$\gg$ --- matches either a or b.
\item $\ll$\texttt{\^{}(From\textbar{}To):}$\gg$ --- matches either $\ll$\texttt{\^{}From:}$\gg$ or
      $\ll$\texttt{\^{}To:}$\gg$.
\end{itemize}
\item $\ll$\texttt{[0-9]\{10\}}$\gg$ --- 10 digits.
\item $\ll$\texttt{0[0-9]|1[0-9]|2[0-3]):[0-5][0-9]}$\gg$ --- 00:00--23:59.
\item $\ll$\texttt{([0-9]|0[0-9]|1[0-9]|2[0-3]):[0-5][0-9]}$\gg$ --- (0)0:00--23:59.
\end{itemize}
\end{frame}


% XXXXXXXXXXXXXXXXXXXXXXXXXXXXXXXXXXXXXXXXXXXXXXXXXXXXXXXXXXXXXXXXXXXXXXXX
\begin{frame}[fragile]
\frametitle{More man, awk, regex, sed (5/7)}
\begin{itemize}
\item $\ll$\texttt{[:alnum:]}$\gg$ --- alpha-numerics.
\item $\ll$\texttt{[:alpha:]}$\gg$ --- alphabets
\item $\ll$\texttt{[:blank:]}$\gg$ --- spaces and tabs.
\item $\ll$\texttt{[:digit:]}$\gg$ --- digits.
\item $\ll$\texttt{[:lower:]}$\gg$ --- lower case.
\item $\ll$\texttt{[:space:]}$\gg$ --- spaces.
\item $\ll$\texttt{[:upper:]}$\gg$ --- upper case.
\item $\ll$\texttt{[:xdigit:]}$\gg$ --- hexadecimal digits.
\item $\ll$\texttt{[:punct:]}$\gg$ --- punctuation.
\item $\ll$\texttt{[:cntrl:]}$\gg$ --- control characters.
\item $\ll$\texttt{[:graph:]}$\gg$ --- printed characters.
\item $\ll$\texttt{[:print:]}$\gg$ --- printed and spaces.
\item $\ll$\texttt{[:word:]}$\gg$ --- alpha-numerics and underscore.
\end{itemize}
\end{frame}

% XXXXXXXXXXXXXXXXXXXXXXXXXXXXXXXXXXXXXXXXXXXXXXXXXXXXXXXXXXX
\begin{frame}[fragile]
\frametitle{More man, awk, regex, sed (6/7)}
\begin{lstlisting}[basicstyle=\ttfamily\tiny]

# Filename: ZA-thisfile1.txt

1. This is no. 1.
2. This is no. 22.
3. This is no. 333.  
4. This is no. 4 4 4 4.
5. This is Joko.
6. This is Joko Joko.
7. This is joko.
8. This is Bowo.
9. This is bowo.
sed    'G'    ZA-thisfile1.txt
sed    'G;G'  ZA-thisfile1.txt
sed -n '4,6p' ZA-thisfile1.txt
sed -n '4,6p' ZA-thisfile1.txt > ZA-thisfile2.txt
sed -n '/[0-9]\{2\}/p' ZA-thisfile1.txt
sed    '4,6d' ZA-thisfile1.txt
sed    '$d' ZA-thisfile1.txt
sed    '5,/HABATS/d'   ZA-thisfile1.txt
sed    's/Joko/Bowo/'   ZA-thisfile1.txt
sed    's/Joko/Bowo/2' ZA-thisfile1.txt
sed    's/Joko/Bowo/g' ZA-thisfile1.txt
sed    's/Bowo\|bowo/Joko/g' ZA-thisfile1.txt
awk    '{print "Hello awk!"}' ZA-thisfile1.txt
awk    '{print $0}' ZA-thisfile1.txt
awk    '{print $1}' ZA-thisfile1.txt
awk    '{print $2}' ZA-thisfile1.txt
HABATS: This is the last line, dude!

\end{lstlisting}
\end{frame}


% NNNN
% XXXXXXXXXXXXXXXXXXXXXXXXXXXXXXXXXXXXXXXXXXXXXXXXXXXXXXXXXXXXXXXXXXXXXXXX
\begin{frame}[fragile]
\frametitle{More man, awk, regex, sed (7/7)}
\begin{itemize}
\item \texttt{sed 'G'    file.txt} --- double space.
\item \texttt{sed 'G;G'  file.txt} --- triple space.
\item \texttt{sed -n '4,6p'  file.txt} --- show only line 4 to 6.
\item \texttt{sed -n '4,6p'  file.txt > newfile.txt} --- write line 4 to 6 to newfile.txt.
\item \texttt{sed '/[0-9]\textbackslash\{2\textbackslash\}/p' file.txt} --- show only lines with two digits. 
\item \texttt{sed '4,6d'     file.txt} --- show all except line 4 to 6.
\item \texttt{sed '\$d'      file.txt} --- show all except last line.
\item \texttt{sed '5,/HABATS/d'} --- show all except from line 5 to a line with HABATS.
\item \texttt{sed 's/Joko/Bowo/'  file.txt} --- replace Joko with Bowo.
\item \texttt{sed 's/Joko/Bowo/2' file.txt} --- replace the second Joko with Bowo.
\item \texttt{sed 's/Joko/Bowo/g' file.txt} --- replace every Joko with Bowo.
\item \texttt{sed 's/Bowo\textbackslash{}|bowo/Joko/g' file.txt} -- replace every Bowo or bowo with Joko.
\end{itemize}
\end{frame}

% 12 XXXXXXXXXXXXXXXXXXXXXXXXXXXXXXXXXXXXXXXXXXXXXXXXXXXXXXXXXXXXXXXXXXXXXXX
% XXXXXXXXXXXXXXXXXXXXXXXXXXXXXXXXXXXXXXXXXXXXXXXXXXXXXXXXXXXXXXXXXXXXXXXXXX
\section{The End}
\begin{frame}
\frametitle{The End}
\begin{itemize}
\item[$\square$] This is the end of the presentation.
\item[$\boxtimes$] This is the end of the presentation.
\item This is the end of the presentation.
\end{itemize}
\end{frame}

% XXXXXXXXXXXXXXXXXXXXXXXXXXXXXXXXXXXXXXXXXXXXXXXXXXXXXXXXXXXXXXXXXXXXXXXXXX
\end{document}

