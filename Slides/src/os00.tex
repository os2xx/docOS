%%%%%%%%%%%%%%%%%%%%%%%%%%%%%%%%%%%%%%%%%%%%%%%%%%%%%%%%%%%%%%%%%%%%%%%%
% Beamer Presentation - LaTeX - Template Version 1.0 (10/11/12)
% This template has been downloaded from: http://www.LaTeXTemplates.com
% License: % CC BY-NC-SA 3.0 (http://creativecommons.org/)
% Modified by Rahmat M. Samik-Ibrahim

% REV425: Thu 12 Sep 2024 08:00
% REV419: Wed 24 Jul 2024 18:00
% REV399: Thu 02 Feb 2023 19:00
% REV379: Tue 17 May 2022 05:00
% REV366: Sat 05 Feb 2022 23:00
% STARTX: Wed 14 Sep 2016 10:00
%%%%%%%%%%%%%%%%%%%%%%%%%%%%%%%%%%%%%%%%%%%%%%%%%%%%%%%%%%%%%%%%%%%%%%%%%

% PACKAGES AND THEMES 
\documentclass[aspectratio=169, xcolor=table, notheorems, hyperref={pdfpagelabels=false}]{beamer}
%%%%%%%%%%%%%%%%%%%%%%%%%%%%%%%%%%%%%%%%%%%%%%%%%%%%%%%%%%%%%%%%%%%%%%%%
% Beamer Presentation - LaTeX - Template Version 1.0 (10/11/12)
% This template has been downloaded from: http://www.LaTeXTemplates.com
% License: % CC BY-NC-SA 3.0 (http://creativecommons.org/)
% Modified by Bin Kadal, Sdn Bhd.
% REV023: Tue 30 Jan 2024 16:00
% REV006: Mon 22 Jan 2018 19:00
% STARTX: Thu 25 Aug 2016 14:00
%%%%%%%%%%%%%%%%%%%%%%%%%%%%%%%%%%%%%%%%%%%%%%%%%%%%%%%%%%%%%%%%%%%%%%%%%

%% ZCZC NNNN
\newtheorem{example}{Example}

%%%%%%%%%%%%%%%%%%%%%%%%%%%%%%%%%%%%%%%%%%%%%%%%%%%%%%%%%%%%%%%%%%%%%%%%%

\let\Tiny=\tiny
\mode<presentation> {
% The Beamer class comes with a number of default slide themes
% which change the colors and layouts of slides. Below this is a list
% of all the themes, uncomment each in turn to see what they look like.
%\usetheme{Boadilla}
\usetheme{Madrid}
% ZCZC %%%%%%%%%%%%%%%%%%%%%%%%%%%%%%%%%%%%%%%%%%%%%%%%%%%%%%%%%%%%%%%%%%
% \usetheme{default} \usetheme{AnnArbor} \usetheme{Antibes} \usetheme{Bergen}
% \usetheme{Berkeley} \usetheme{Berlin} \usetheme{CambridgeUS} 
% \usetheme{Copenhagen} \usetheme{Darmstadt} \usetheme{Dresden}
% \usetheme{Frankfurt} \usetheme{Goettingen} \usetheme{Hannover}
% \usetheme{Ilmenau} \usetheme{JuanLesPins} \usetheme{Luebeck}
% \usetheme{Malmoe} \usetheme{Marburg} \usetheme{Montpellier}
% \usetheme{PaloAlto} \usetheme{Pittsburgh} \usetheme{Rochester}
% \usetheme{Singapore} \usetheme{Szeged} \usetheme{Warsaw}
% NNNN %%%%%%%%%%%%%%%%%%%%%%%%%%%%%%%%%%%%%%%%%%%%%%%%%%%%%%%%%%%%%%%%%%
% As well as themes, the Beamer class has a number of color themes
% for any slide theme. Uncomment each of these in turn to see how it
% changes the colors of your current slide theme.
%\usecolortheme{orchid}
%\usecolortheme{rose}
%\usecolortheme{seagull}
%\usecolortheme{seahorse}
\usecolortheme{whale}
% ZCZC %%%%%%%%%%%%%%%%%%%%%%%%%%%%%%%%%%%%%%%%%%%%%%%%%%%%%%%%%%%%%%%%%%
%\usecolortheme{albatross} \usecolortheme{beaver} \usecolortheme{beetle}
%\usecolortheme{crane} \usecolortheme{dolphin} \usecolortheme{dove}
%\usecolortheme{fly} \usecolortheme{lily} \usecolortheme{wolverine}
% NNNN %%%%%%%%%%%%%%%%%%%%%%%%%%%%%%%%%%%%%%%%%%%%%%%%%%%%%%%%%%%%%%%%%%
% To remove the footer line in all slides uncomment this line
%\setbeamertemplate{footline} 
% To replace the footer line in all slides uncomment this line
%\setbeamertemplate{footline}[page number] 
% To remove the navigation symbols from the bottom uncomment this line
\setbeamertemplate{navigation symbols}{} 
}

\usepackage{array}       % ZCZC
\usepackage{amssymb}     % ZCZC
\usepackage{bold-extra}  % ZCZC
\usepackage{booktabs}    % Allows \toprule, \midrule and \bottomrule in tables
\usepackage{caption}
\usepackage[T1]{fontenc} % ZCZC << >>
\usepackage{graphicx}    % Allows including images
\usepackage{listings}    % listing
\usepackage{lmodern}     % ZCZC
\usepackage{perpage}     % reset footnote per page
\usepackage{geometry}    % ZCZC
\usepackage{adjustbox}   % ZCZC
\usepackage{multirow}    % ZCZC

% \definecolor{links}{HTML}{2A1B81}
\definecolor{links}{HTML}{0011FF}
\hypersetup{colorlinks,linkcolor=,urlcolor=links}

% \usepackage{xcolor}
% \usepackage[colorlinks = true,
%             linkcolor = blue,
%             urlcolor  = blue,
%             citecolor = blue,
%             anchorcolor = blue]{hyperref}

\captionsetup[table]{name=Tabel}
\makeatletter
\def\input@path{{src/}}
\makeatother
\graphicspath{{src/}}      % src directory
\MakePerPage{footnote}     % reset page

% NNNN %%%%%%%%%%%%%%%%%%%%%%%%%%%%%%%%%%%%%%%%%%%%%%%%%%%%%%%%%%%%%%%%%%

%% % XXXXXXXXXXXXXXXXXXXXXXXXXXXXXXXXXXXXXXXXXXXXXXXXXXXXXXXXXXXXXXXXXXXXXXXXXX
%% % The short title appears at the bottom of every slide, 
%% % the full title is only on the title page
%% \title[Judul Pendek]{Judul Panjang dan Lengkap} 
%% \author{Cecak bin Kadal}
%% \institute[UILA]
%% {
%% University of Indonesia at Lenteng Agung \\ 
%% \medskip
%% \textit{cecak@binKadal.com}
%% }
%% \date{REV00 24-Aug-2016}
%% % \date{\today}
%% 

%% % XXXXXXXXXXXXXXXXXXXXXXXXXXXXXXXXXXXXXXXXXXXXXXXXXXXXXXXXXXXXXXXXXXXXXXXXXX
%% \begin{document}
%% \section{Judul}
%% \begin{frame}
%% \titlepage
%% \end{frame}
%% 
%% % XXXXXXXXXXXXXXXXXXXXXXXXXXXXXXXXXXXXXXXXXXXXXXXXXXXXXXXXXXXXXXXXXXXXXXXXXX
%% \section{Agenda}
%% \begin{frame}
%% \frametitle{Agenda}
%% % Throughout your presentation, if you choose to use \section{} and 
%% % \subsection{} commands, these will automatically be printed on 
%% % this slide as an overview of your presentation
%% \tableofcontents 
%% \end{frame}
%% 
%% % XXXXXXXXXXXXXXXXXXXXXXXXXXXXXXXXXXXXXXXXXXXXXXXXXXXXXXXXXXXXXXXXXXXXXXXXXX
%% \section{UUD dan Pancasila}
%% \subsection{UUD}
%% \begin{frame}
%% \frametitle{Pembukaan}
%% Bahwa sesungguhnya kemerdekaan itu ialah hak segala bangsa dan oleh 
%% sebab itu, maka penjajahan diatas dunia harus dihapuskan karena 
%% tidak sesuai dengan perikemanusiaan dan perikeadilan.
%% \\~\\
%% Atas berkat rahmat Allah Yang Maha Kuasa dan dengan didorongkan oleh 
%% keinginan luhur, supaya berkehidupan kebangsaan yang bebas, maka 
%% rakyat Indonesia menyatakan dengan ini kemerdekaannya.
%% \end{frame}
%% 
%% % XXXXXXXXXXXXXXXXXXXXXXXXXXXXXXXXXXXXXXXXXXXXXXXXXXXXXXXXXXXXXXXXXXXXXXXXXX
%% \begin{frame}
%% \frametitle{Alenia Ketiga}
%% Kemudian daripada itu untuk membentuk suatu pemerintah negara Indonesia 
%% yang melindungi segenap bangsa Indonesia dan seluruh tumpah darah Indonesia 
%% dan untuk memajukan kesejahteraan umum, mencerdaskan kehidupan bangsa, dan 
%% ikut melaksanakan ketertiban dunia yang berdasarkan kemerdekaan, perdamaian 
%% abadi dan keadilan sosial, maka disusunlah kemerdekaan kebangsaan Indonesia 
%% itu dalam suatu Undang-Undang Dasar negara Indonesia, yang terbentuk dalam 
%% suatu susunan negara Republik Indonesia yang berkedaulatan rakyat dengan 
%% berdasar kepada:
%% \begin{itemize}
%% \item Ketuhanan Yang Maha Esa,
%% \item kemanusiaan yang adil dan beradab,
%% \item persatuan Indonesia,
%% \item dan kerakyatan yang dipimpin oleh hikmat kebijaksanaan 
%%       dalam permusyawaratan/ perwakilan,
%% \item serta dengan mewujudkan suatu keadilan sosial bagi seluruh rakyat 
%%       Indonesia.
%% \end{itemize}
%% \end{frame}
%% 
%% % XXXXXXXXXXXXXXXXXXXXXXXXXXXXXXXXXXXXXXXXXXXXXXXXXXXXXXXXXXXXXXXXXXXXXXXXXX
%% \subsection{Pancasila}
%% \begin{frame}
%% \frametitle{Tujuh Kunci Pokok}
%% \begin{block}{Pertama - Kedua - Ketiga}
%% Indonesia ialah negara berdasarkan hukum.
%% Sistem konstitusional.
%% Kekuasaan negara tertinggi di tangan MPR.
%% \end{block}
%% 
%% \begin{block}{Keempat - Kelima}
%% Presiden adalah penyelenggara pemerintahan tertinggi di bawah MPR.
%% Adanya pengawasan DPR.
%% \end{block}
%% 
%% \begin{block}{Keenam}
%% Menteri negara adalah pembantu presiden dan tidak bertanggung jawab 
%% kepada DPR.
%% \end{block}
%% 
%% \begin{block}{Ketujuh}
%% Kekuasaan kepala negara tidak tak tebatas.
%% \end{block}
%% 
%% \end{frame}
%% 
%% % XXXXXXXXXXXXXXXXXXXXXXXXXXXXXXXXXXXXXXXXXXXXXXXXXXXXXXXXXXXXXXXXXXXXXXXXXX
%% \section{Rupa-rupa}
%% \subsection{Kolom}
%% \begin{frame}
%% \frametitle{Kolom}
%% % The "c" option specifies centered vertical alignment 
%% % while the "t" option is used for top vertical alignment
%% \begin{columns}[c] 
%% % Left column and width
%% \column{.45\textwidth} 
%% \textbf{Heading}
%% \begin{enumerate}
%% \item Satu-satu
%% \item Dua-dua
%% \item Tiga-tiga
%% \item Satu-dua-tiga
%% \end{enumerate}
%% 
%% % Right column and width
%% \column{.5\textwidth}
%% Satu-satu~\dots{} aku sayang ibu!
%% Dua-dua~\ldots{} juga sayang ayah!
%% Tiga-tiga~\ldots{} sayang adik kakak!
%% Satu-dua-tiga~\ldots{} sayang semuanya!
%% 
%% \end{columns}
%% \end{frame}
%% 
%% % XXXXXXXXXXXXXXXXXXXXXXXXXXXXXXXXXXXXXXXXXXXXXXXXXXXXXXXXXXXXXXXXXXXXXXXXXX
%% \subsection{Tabel}
%% \begin{frame}
%% \frametitle{Tabel}
%% \begin{table}
%% \begin{tabular}{l l l}
%% \toprule
%% \textbf{Nama} & \textbf{NPM} & \textbf{Tanggal Lahir}\\
%% \midrule
%% Cecak bin Kadal & 1234567890 & 1 Jan 2015 \\
%% Aneh bin Ajaib  & 0987654321 & 31 Des 2014 \\
%% \bottomrule
%% \end{tabular}
%% \caption{Keterangan Tabel}
%% \end{table}
%% \end{frame}
%% 
%% % XXXXXXXXXXXXXXXXXXXXXXXXXXXXXXXXXXXXXXXXXXXXXXXXXXXXXXXXXXXXXXXXXXXXXXXXXX
%% \subsection{Teori}
%% \begin{frame}
%% \frametitle{Teori}
%% \begin{theorem}[Teori Satu Batu]
%% $E = mc^2$
%% \end{theorem}
%% \end{frame}
%% 
%% % XXXXXXXXXXXXXXXXXXXXXXXXXXXXXXXXXXXXXXXXXXXXXXXXXXXXXXXXXXXXXXXXXXXXXXXXXX
%% \subsection{Verbatim}
%% % Need to use the fragile option when verbatim is used in the slide
%% \begin{frame}[fragile] 
%% \frametitle{Verbatim}
%% \begin{example}[Teori Satu Batu]
%% \begin{verbatim}
%% \begin{theorem}[Teori Satu Batu]
%% $E = mc^2$
%% \end{theorem}
%% \end{verbatim}
%% \end{example}
%% \end{frame}
%% 
%% % XXXXXXXXXXXXXXXXXXXXXXXXXXXXXXXXXXXXXXXXXXXXXXXXXXXXXXXXXXXXXXXXXXXXXXXXXX
%% \subsection{Gambar}
%% \begin{frame}
%% \frametitle{Gambar}
%% \begin{figure}
%% \includegraphics[width=0.5\linewidth]{2}
%% \caption{Ini Gambar JPG}
%% \end{figure}
%% \end{frame}
%% 
%% % XXXXXXXXXXXXXXXXXXXXXXXXXXXXXXXXXXXXXXXXXXXXXXXXXXXXXXXXXXXXXXXXXXXXXXXXXX
%% \subsection{Rujukan}
%% % Need to use the fragile option when verbatim is used in the slide
%% \begin{frame}[fragile] 
%% \frametitle{Rujukan dan Kutipan}
%% Contoh penggunaan \verb|\cite| ketika mengutip\cite{p1}.
%% Perhatian: Beamer tidak mengerti \verb|\BibTeX|~\ldots
%% \footnotesize{
%%   \begin{thebibliography}{99} 
%%   \bibitem[Smith, 2012]{p1} John Smith (2012)
%%      \newblock Katak dalam Tempurung
%%      \newblock \emph{Jurnal Kelapa dan Amfibi} 12(3), 45 -- 678.
%%   \end{thebibliography}
%% }
%% \end{frame}
%% 
%% % XXXXXXXXXXXXXXXXXXXXXXXXXXXXXXXXXXXXXXXXXXXXXXXXXXXXXXXXXXXXXXXXXXXXXXXXXX
%% \subsection{Selesai}
%% \begin{frame}
%% \Huge{\centerline{Selesai}}
%% \end{frame}
%% 
%% % XXXXXXXXXXXXXXXXXXXXXXXXXXXXXXXXXXXXXXXXXXXXXXXXXXXXXXXXXXXXXXXXXXXXXXXXXX
%% \end{document}

\newcommand{\revision}{%
REV424: Tue 03 Sep 2024 20:00
}
% w! tmptmp
% REV424: Tue 03 Sep 2024 20:00
% REV419: Wed 24 Jul 2024 17:00
% REV409: Tue 08 Aug 2023 12:00
% REV399: Fri 03 Feb 2023 20:00
% REV339: Sat 04 Sep 2021 12:00
% STARTS: Wed 24 Aug 2016 19:00
%%%%%%%%%%%%%%%%%%%%%%%%%%%%%%%%%%%%%
\newcommand{\kopikopi}{\textcopyright{}2016-2024 CBKadal + VauLSMorg}



% XXXXXXXXXXXXXXXXXXXXXXXXXXXXXXXXXXXXXXXXXXXXXXXXXXXXXXXXXXXXXXXXXXXXXXXXXX
% The short title appears at the bottom of every slide, 
% the full title is only on the title page
% \date{\today}
\title[\kopikopi]
{CSGE602055 Operating Systems \\ 
CSF2600505 Sistem Operasi \\
Week 00:
Overview 1, Virtualization \& Scripting}
\author{C. BinKadal}
\institute[SdnBhd]
{
Sendirian Berhad\\
\medskip
\url{https://docos.vlsm.org/Slides/os00.pdf}
\\ \texttt{Always check for the latest revision!}
}
\date{\revision}

% XXXXXXXXXXXXXXXXXXXXXXXXXXXXXXXXXXXXXXXXXXXXXXXXXXXXXXXXXXXXXXXXXXXXXXXXXX
\begin{document}

\lstset{
basicstyle=\ttfamily\tiny, % \tiny \small \footnotesize 
breakatwhitespace=true,
language=C,
columns=fullflexible,
keepspaces=true,
breaklines=true,
tabsize=3, 
showstringspaces=false,
extendedchars=true}

\section{Start}
\begin{frame}
\titlepage
\end{frame}

% XXXXXXXXXXXXXXXXXXXXXXXXXXXXXXXXXXXXXXXXXXXXXXXXXXXXXXXXXXXXXXXXXXXXXXXXXX

%%%%%%%%%%%%%%%%%%%%%%%%%%%%%%%%%%%%%%%%%%%%%%%%%%%%%%%%%%%%%%%%%%%%%%%%%
% REV418: Tue 30 Jan 2024 22:00
% REV406: Sat 05 Aug 2023 14:00
% REV399: Thu 02 Feb 2023 00:00
% REV369: Mon 14 Feb 2022 09:00
% REV328: Sat 14 Aug 2021 06:00
% STARTX: Wed 14 Sep 2016 10:00
%%%%%%%%%%%%%%%%%%%%%%%%%%%%%%%%%%%%%%%%%%%%%%%%%%%%%%%%%%%%%%%%%%%%%%%%%

\begin{frame}[fragile]
\section{OS241 Schedule}
\frametitle{OS241\footnote{%
) This information will be on \textbf{EVERY} page two (2) of this course material.}): 
Operating Systems Schedule 2023 - 2}

\vspace{5pt}

\scalebox{0.99}{%
\begin{tabular}{|c|c|l|l|}
\hline
\textbf{Week} & 
\textbf{Topic}\footnote{%
) For schedule, see \url{https://os.vlsm.org/\#idx02}}) & \textbf{OSC10}\footnote{%
    ) Silberschatz et. al.: \textbf{Operating System Concepts}, $10^{th}$ Edition, 2018.}) \\
\hline
Week 00  & Overview (1), Assignment of Week 00           & Ch. 1, 2      \\
Week 01  & Overview (2), Virtualization \& Scripting     & Ch. 1, 2, 18. \\
Week 02  & Security, Protection, Privacy, \& C-language. & Ch. 16, 17.   \\
Week 03  & File System \& FUSE  & Ch. 13, 14, 15.                        \\
Week 04  & Addressing, Shared Lib, \& Pointer & Ch. 9. \\
Week 05  & Virtual Memory & Ch. 10. \\
\hline
Week 06  & Concurrency: Processes \& Threads & Ch. 3, 4. \\
Week 07  & Synchronization \& Deadlock & Ch. 6, 7, 8. \\
Week 08  & Scheduling + W06/W07 & Ch. 5. \\
Week 09  & Storage, Firmware, Bootloader, \& Systemd & Ch. 11. \\
Week 10  & I/O \& Programming & Ch. 12. \\%
% MidTerm  & 00 XXX 2020 (XX:XX-XX:XX) & MidTerm (UTS) & \cellcolor{red!44} TBA! \\
% Reserved & 00 XXX - 00 XXX 2020 & Q \& A & \\
% Final    & 00 XXX 2020 XX:XX & First Part Final  (UAS tahap I)  & \cellcolor{red!44} This schedule is   \\
% Extra    & NA & No Extra assignment & \cellcolor{red!44} subject to change. \\
\hline
\end{tabular}
}
\end{frame}

\begin{frame}[fragile]
\frametitle{\textbf{STARTING POINT} --- 
{
\definecolor{links}{HTML}{FDEE00}
\hypersetup{colorlinks,linkcolor=,urlcolor=links}
\url{https://os.vlsm.org/}
}
}
\begin{itemize}
\item[$\square$] \textbf{Text Book} ---
                 Any recent/decent OS book. Eg. (\textbf{OSC10}) Silberschatz et. al.: 
                 \textbf{Operating System Concepts}, $10^{th}$ Edition, 2018.
                 (See \url{https://codex.cs.yale.edu/avi/os-book/OS10/}).
\item[$\square$] \textbf{Resources ({\footnotesize \url{https://os.vlsm.org/\#idx03}})}
\begin{itemize}
\item[$\square$] \href{https://scele.cs.ui.ac.id/course/view.php?id=3743}{\textbf{SCELE}} ---
\url{https://scele.cs.ui.ac.id/course/view.php?id=3743}.\\
The enrollment key is \textbf{XXX}.
\item[$\square$] \textbf{Download Slides and Demos from GitHub.com} --- (\url{https://github.com/os2xx/docOS/})\\
                 {\scriptsize%
                 \href{https://docOS.vlsm.org/Slides/os00.pdf}{\texttt{os00.pdf} (W00)},
                 \href{https://docOS.vlsm.org/Slides/os01.pdf}{\texttt{os01.pdf} (W01)},
                 \href{https://docOS.vlsm.org/Slides/os02.pdf}{\texttt{os02.pdf} (W02)},
                 \href{https://docOS.vlsm.org/Slides/os03.pdf}{\texttt{os03.pdf} (W03)},
                 \href{https://docOS.vlsm.org/Slides/os04.pdf}{\texttt{os04.pdf} (W04)},
                 \href{https://docOS.vlsm.org/Slides/os05.pdf}{\texttt{os05.pdf} (W05)},\\
                 \href{https://docOS.vlsm.org/Slides/os06.pdf}{\texttt{os06.pdf} (W06)},
                 \href{https://docOS.vlsm.org/Slides/os07.pdf}{\texttt{os07.pdf} (W07)},
                 \href{https://docOS.vlsm.org/Slides/os08.pdf}{\texttt{os08.pdf} (W08)},
                 \href{https://docOS.vlsm.org/Slides/os09.pdf}{\texttt{os09.pdf} (W09)},
                 \href{https://docOS.vlsm.org/Slides/os10.pdf}{\texttt{os10.pdf} (W10)}.
                 }
\item[$\square$] \textbf{Problems}\\
                 {\scriptsize% 
                 \href{https://rms46.vlsm.org/2/195.pdf}{\texttt{195.pdf} (W00)},
                 \href{https://rms46.vlsm.org/2/196.pdf}{\texttt{196.pdf} (W01)},
                 \href{https://rms46.vlsm.org/2/197.pdf}{\texttt{197.pdf} (W02)},
                 \href{https://rms46.vlsm.org/2/198.pdf}{\texttt{198.pdf} (W03)},
                 \href{https://rms46.vlsm.org/2/199.pdf}{\texttt{199.pdf} (W04)},
                 \href{https://rms46.vlsm.org/2/200.pdf}{\texttt{200.pdf} (W05)},\\
                 \href{https://rms46.vlsm.org/2/201.pdf}{\texttt{201.pdf} (W06)},
                 \href{https://rms46.vlsm.org/2/202.pdf}{\texttt{202.pdf} (W07)},
                 \href{https://rms46.vlsm.org/2/203.pdf}{\texttt{203.pdf} (W08)},
                 \href{https://rms46.vlsm.org/2/204.pdf}{\texttt{204.pdf} (W09)},
                 \href{https://rms46.vlsm.org/2/205.pdf}{\texttt{205.pdf} (W10)}.}
\item[$\square$] \textbf{LFS} --- \url{http://www.linuxfromscratch.org/lfs/view/stable/}
\item[$\square$] \textbf{OSP4DISS} --- \url{https://osp4diss.vlsm.org/}
\item[$\square$] \textbf{This is How Me Do It!} --- \url{https://doit.vlsm.org/}
\begin{itemize}
\item[$\square$] PS: "Me" rhymes better than "I", duh!
\end{itemize}
\end{itemize}
\end{itemize}
\end{frame}



% XXXXXXXXXXXXXXXXXXXXXXXXXXXXXXXXXXXXXXXXXXXXXXXXXXXXXXXXXXXXXXXXXXXXXXXXXX
% Throughout your presentation, if you choose to use \section{} and 
% \subsection{} commands, these will automatically be printed on 
% this slide as an overview of your presentation
\section{Agenda}
\begin{frame}{Outline}
  \frametitle{Agenda}
  \tableofcontents[sections={1-15}]
\end{frame}
\begin{frame}
   \frametitle{Agenda (2)}
   \tableofcontents[sections={16-}]
\end{frame}

% XXXXXXXXXXXXXXXXXXXXXXXXXXXXXXXXXXXXXXXXXXXXXXXXXXXXXXXXXXXXXXXXXXXXXXXXXX

\input{os00-BRP.tex}

% XXXXXXXXXXXXXXXXXXXXXXXXXXXXXXXXXXXXXXXXXXXXXXXXXXXXXXXXXXXXXXXXXXXXXXXXXX
\section{How to contact the Lecturer}
\begin{frame}[fragile]
\frametitle{How to contact the Lecturer}
\begin{itemize}
\item \textbf{Always introduce yourself}.
\begin{itemize}
\item State your ''\texttt{GitHubAccount}'', ''\texttt{Student ID}'', ''\texttt{Hypervisor}'', and ''\texttt{OS class}''.
\end{itemize}
\item Post a question/query on 
\href{https://scele.cs.ui.ac.id/course/view.php?id=3841}{\textbf{SCELE}} ---
(The enrollment key is \textbf{XXX}):
\href{https://scele.cs.ui.ac.id/course/view.php?id=3841}{https://scele.cs.ui.ac.id/course/view.php?id=3841}.
\textbf{DO NOT} send an email for assignment-related questions.
\item For SIAK related questions, use email with
Subject:\textbf{[OS]} Mailto: \texttt{rms46(AT)ui.ac.id}. 
\item For SCELE QUIZ (not POP QUIZ) grading appeal, use email with
Subject:\textbf{[APPEAL]}  Mailto: \texttt{amril.syalim(AT)cs.ui.ac.id}. 
\end{itemize}

\begin{figure}
\includegraphics[width=0.18\linewidth]{os00-pib}
\caption{Never ever whine and pretend like 
         \href{https://rahmatm.samik-ibrahim.vlsm.org/2013/12/puss-in-boots.html}{this}\footnote{''Puss in Boot'' is a DreamWorks/Paramount Picture character.}!}
\end{figure}
\end{frame}

% XXXXXXXXXXXXXXXXXXXXXXXXXXXXXXXXXXXXXXXXXXXXXXXXXXXXXXXXXXXXXXXXXXXXXXXXXX
\begin{frame}
\frametitle{Emails between a ''Gen Z'' and a ''Babyboomer''}
\begin{figure}
\includegraphics[width=1.01\linewidth]{os-millenial-mail}
\end{figure}
\end{frame}

% XXXXXXXXXXXXXXXXXXXXXXXXXXXXXXXXXXXXXXXXXXXXXXXXXXXXXXXXXXXXXXXXXXXXXXXXXX
\section{Assessment}
\begin{frame}
\frametitle{Assessment}
\begin{itemize}
\item \textbf{11 Weekly Assignments @ 11.11 points}.
\begin{itemize}
\item Assignments will vary from week to week.
\item The assignment deadline will be by the end of every week. 
See \url{https://os.vlsm.org/\#idx02}.
\item Check your points regularly at \url{https://academic.ui.ac.id/}
\item See also, \url{https://os.vlsm.org/Log/}.
\item \textbf{DO NOT COMPLAIN} weeks after! 
\end{itemize}
\item You need to log your weekly activities!
\begin{itemize}
\item See \url{https://doit.vlsm.org/ETC/logCodes.txt}
\item See \url{https://cbkadal.github.io/os242/TXT/mylog.txt}
\item \textbf{4 SKS} (Units) means 12 hours (720 minutes) per week!
\item The average time allocation for each weekly assignment is 
      425 minutes—only 45\% of the four SKS (units) load.
\item Most of the time (44 \%) will be spent on the weekly assignment.
\end{itemize}
\end{itemize}

\end{frame}

% XXXXXXXXXXXXXXXXXXXXXXXXXXXXXXXXXXXXXXXXXXXXXXXXXXXXXXXXXXXXXXXXXXXXXXXXXX
\section{Average Time Allocation}
\begin{frame}
\frametitle{Average Time Allocation}

\begin{figure}
\begin{tikzpicture}[scale=0.91, transform shape]
\pie{44/Assignments,
      5.2/Quizzes,
     24.9/Zoom,
     25.9/Miscellaneous}
\end{tikzpicture}
\caption{Operating Systems classes (2021-2022) student time allocation chart}
\end{figure}

\end{frame}

% XXXXXXXXXXXXXXXXXXXXXXXXXXXXXXXXXXXXXXXXXXXXXXXXXXXXXXXXXXXXXXXXXXXXXXXXXX
\section{NFT: Non-Fungible Tests}
\begin{frame}
\frametitle{NFT: Non-Fungible Tests}
\begin{itemize}

\item Midterm (UTS)
\begin{itemize}
\item The Midterm (UTS) is mandatory \textbf{IF} your SCORE is less than 25.0 before UTS.
      Otherwise, you need to register if you want to take UTS.
\item The UTS result will replace the worst grade of Assignment 00-05, 
      even if the result is less.
\end{itemize}
\item Final Term (UAS)
\begin{itemize}
\item The final term (UAS) is mandatory \textbf{IF} your SCORE is less than 55.0 before UAS.
      Otherwise, you need to register if you want to take UAS.
\item The UAS result will replace the worst grade of Assignments 06-10,
      even if the result is less.
\end{itemize}
\item Both UTS and UAS can only be held offline in an exam room.
\item You can read an A4 size MEMO -- reciprocal -- written in your \textbf{handwriting}.
\item Failure to show up on the day of the exam without reason and evidence will get a score of "0".
\end{itemize}
\end{frame}

% XXXXXXXXXXXXXXXXXXXXXXXXXXXXXXXXXXXXXXXXXXXXXXXXXXXXXXXXXXXXXXXXXXXXXXXXXX
\section{Final Grade}
\begin{frame}
\frametitle{Final Grade (1)}

\begin{itemize}
\item The final grade will be the best 9 out of 11 assignments.
\item \textbf{Two (2) ''spare'' assignments will be more than enough!}
\item In case of emergency, contact your Academic Advisor!
\item C-2C (C minus to C)
\begin{itemize}
\item Up to 5 points, only if:
\begin{itemize}
\item your grade is between 50.00 and 55.00, and
\item you have a ''good'' track record.
\end{itemize}
\end{itemize}

\item Score Range\\[10pt]
\begin{tabular}{l l l l}
\hline
85 - ... = A & 80 - 85 = A- & 75 - 80 = B+ & 70 - 75 = B \\
65 - 70 = B-      & 60 - 65 = C+ & 55 - 60 = C  & 
50 - 55 = D or C\footnote{C-2C: terms and conditions apply --- void where prohibited by law.}  \\
40 - 50 = D  & 30 - 40 = E  & 20 - 30 = \small E & 00 - 20 = \tiny E   \\
\hline \end{tabular}\\[10pt]
\end{itemize}
\end{frame}

% XXXXXXXXXXXXXXXXXXXXXXXXXXXXXXXXXXXXXXXXXXXXXXXXXXXXXXXXXXXXXXXXXXXXXXXXXX
\begin{frame}[fragile]
\frametitle{The eternal recurring chronic problem} 

\textbf{How to avoid this at the end of the semester after grades have been published?}

\begin{figure}
\includegraphics[width=1.01\linewidth]{os-appeal}
\end{figure}

\begin{itemize}
\item Do not ask for any dispensations like a betrothed, broken computer, circumcision (sunat), cold, competitions
      (including Gemastik), deadline extension, getting married, graduation ceremony (acara wisuda),
      influenza, lame excuses, link error, mourning,
      power failure, remedial, return to the village (mudik), SCELE problem, SEB problem, slow network (lemot), 
      two-semester evaluation, umrah,
      wedding ceremony, etc.
\item It also includes: ''It is not my fault but of $\{ X\!: X\ \in\ Lecturer\ \parallel\ Fasilkom\
      \parallel\ UI\ \parallel\ Kampus\ Merdeka\ \parallel\ Immigration\ \parallel\ Foreign\ Embassy\ \parallel\
      else\, \}$.''
\end{itemize}

\end{frame}

% XXXXXXXXXXXXXXXXXXXXXXXXXXXXXXXXXXXXXXXXXXXXXXXXXXXXXXXXXXXXXXXXXXXXXXXXXX
\begin{frame}[fragile]
\frametitle{Grade Examples}

\begin{figure}
\includegraphics[width=0.94\linewidth]{os-siak}
\end{figure}

\end{frame}

% XXXXXXXXXXXXXXXXXXXXXXXXXXXXXXXXXXXXXXXXXXXXXXXXXXXXXXXXXXXXXXXXXXXXXXXXXX
\section{The Three-Strikes Rule}
\begin{frame}[fragile]
\frametitle{The Three-Strikes Rule}

\begin{figure}
\includegraphics[width=0.30\linewidth]{os-cheating}
\end{figure}

\begin{itemize}
\item All major academic rules violations will be handled directly by the Faculty of Computer Science,
University of Indonesia.
\item ''Accidents'' may happen. There will be warnings for the first two minor violations.
\item Your final grade will be reduced for the third warning.
\item Your final grade will be reduced to "D" for the fourth warning.
\item Five (5) or more warnings will be considered as a significant academic-rules violation.
\end{itemize}

\end{frame}

% XXXXXXXXXXXXXXXXXXXXXXXXXXXXXXXXXXXXXXXXXXXXXXXXXXXXXXXXXXXXXXXXXXXXXXXXXX
\begin{frame}[fragile]
\frametitle{AIN'T DIFFICULT, lah!}
\begin{figure}
\includegraphics[width=0.79\linewidth]{os-kambing-kuliah-c}
\caption{Even this Goat will get ''C'' at the end of the semester!}
\end{figure}
\end{frame}

% XXXXXXXXXXXXXXXXXXXXXXXXXXXXXXXXXXXXXXXXXXXXXXXXXXXXXXXXXXXXXXXXXXXXXXXXXX
\section{Study From Anywhere?}
\begin{frame}[fragile]
\frametitle{Study From Anywhere?}
\begin{figure}
\includegraphics[width=0.64\linewidth]{os-cat}
\caption{Who is on Zoom? What? I don't know! Why? Because! Today, I Don't Give a Darn!}
\end{figure}
\end{frame}

% XXXXXXXXXXXXXXXXXXXXXXXXXXXXXXXXXXXXXXXXXXXXXXXXXXXXXXXXXXXXXXXXXXXXXXXXXX
\begin{frame}[fragile]
\frametitle{Prelude: Daisy Bell -- Bicycle Built for Two}
\begin{tabular}{cc}
\begin{minipage}{45mm}
\vspace{1pt}
\includegraphics[width=0.89\linewidth]{os-daisybell}
\end{minipage}
&
\begin{minipage}{65mm}
\vspace{1pt}
\begin{verbatim}
Daisy, Daisy,
Give me your answer, do!
I'm half crazy,
All for the love of you!
It won't be a stylish marriage,
I can't afford a carriage,
But you'll look sweet on the seat
Of a bicycle built for two!
\end{verbatim}
\end{minipage}
\\
\end{tabular}
\\[5mm]

YouTube {\footnotesize  (\url{https://youtu.be/TXK_cE9AqAI})}.
A choir (emulation) of 
\href{https://youtu.be/TXK_cE9AqAI?t=68}{VOCODER}
(pre WW2), 
\href{https://youtu.be/TXK_cE9AqAI?t=99}{IBM704}
(1950s) and 
\href{https://youtu.be/TXK_cE9AqAI?t=130}{Vocaloid4}
(2014).
See also the classical movie \href{https://youtu.be/oR\_e9y-bka0}{2001: A Space Odyssey}. 
\end{frame}

% XXXXXXXXXXXXXXXXXXXXXXXXXXXXXXXXXXXXXXXXXXXXXXXXXXXXXXXXXXXXXXXXXXXXXXXXXX
\begin{frame}
\frametitle{IBM 704 at Los Alamos National Laboratory in the 1950s}
\includegraphics[width=0.75\linewidth]{os-ibm704}

Estimate price (2020 value): USD 8,000,000.

Weight: 8800 kg --- Electricity: ca. 200 kWatt --- 42000 flops --- 
128 kbytes (eq.) core memory --- 64 kbytes (eq.) drum memory --- 3 Mbytes (eq.) Tape Unit.

\end{frame}

% XXXXXXXXXXXXXXXXXXXXXXXXXXXXXXXXXXXXXXXXXXXXXXXXXXXXXXXXXXXXXXXXXXXXXXXXXX
\begin{frame}
\frametitle{Xiaomi 12 Pro -- 12 GB / 256 GB}
\begin{figure}
\includegraphics[width=0.63\linewidth]{xiaome12pro}
\caption{Source: Mi Indonesia (2024)}
\end{figure}
\end{frame}

% XXXXXXXXXXXXXXXXXXXXXXXXXXXXXXXXXXXXXXXXXXXXXXXXXXXXXXXXXXXXXXXXXXXXXXX
\section{Miscellaneous}
\begin{frame}[fragile]
\frametitle{Out of Topic/Intermezzo/Segue}
\begin{itemize}
\item Semiconductor Scalling:
\begin{itemize}
\item Process Shrink: $10 \mu{}m$ (1971), $250 nm$ (1996), $10 nm$ (2016), $5 nm$ (2020), $3 nm$ (2022).
\item Smaller Devices means:
\begin{itemize}
\item Less space.
\item Less power consumption.
\item More density.
\end{itemize}
\end{itemize}
\item Indonesia:
\begin{itemize}
\item Fairchild Semiconductor Indonesia.
\item National Semiconductor Indonesia.
\item Minister of Manpower (Menteri Tenaga Kerja) 1983–1988.
\end{itemize}
\item Technology:
\begin{itemize}
\item SoC: System on a Chip.
\item SiP: System in a Package.
\item Fab/Foundry: Taiwan Semiconductor Manufacturing Company (TSMC), Ltd.
\begin{itemize}
\item Have No Fab? It is OK! E.g., Marvell Technology, Inc (1995).
\end{itemize}
\item Lithography: ASML Holding, N.V: Advanced Semiconductor Materials Lithography.
\item Optics: Carl Zeiss SMT GmbH (This is NOT Optik Seis, Duh :).
\end{itemize}
\end{itemize}
\end{frame}

% XXXXXXXXXXXXXXXXXXXXXXXXXXXXXXXXXXXXXXXXXXXXXXXXXXXXXXXXXXXXXXXXXXXXXXX
\begin{frame}[fragile]
\frametitle{TSMC Logic Nodes}
\begin{figure}
\includegraphics[width=0.95\linewidth]{tsmc_logic_node}
\caption{Source: 
  \href{https://fuse.wikichip.org/wp-content/uploads/2022/09/wikichip_tsmc_logic_node_q2_2022-2.png}{WikiChip}}
\end{figure}
\end{frame}

% XXXXXXXXXXXXXXXXXXXXXXXXXXXXXXXXXXXXXXXXXXXXXXXXXXXXXXXXXXXXXXXXXXXXXXX
\begin{frame}[fragile]
\frametitle{The Computing Diciplines}
\begin{figure}
\includegraphics[width=0.59\linewidth]{pic-cc2005}
\caption{The Computing Diciplines}
\end{figure}
\end{frame}

% XXXXXXXXXXXXXXXXXXXXXXXXXXXXXXXXXXXXXXXXXXXXXXXXXXXXXXXXXXXXXXXXXXXXXXXXXX
\begin{frame}[fragile]
\frametitle{Lessons from the Development of the Boeing 787 Dreamliner}
\begin{itemize}
\item 1997: Boeing acquired the nearly bankrupt  McDonnell Douglas.
\begin{itemize}
\item Result: "Boeing honorable name" with "McDonnell Douglas Greedy Culture."
\end{itemize}
\item 2003: Boeing announced the Boeing 787 Dreamliner project.
\item 2007: An "empty skeleton" prototype was rolled out on schedule. Many fuselage parts were temporarily attached.
\begin{itemize}
\item Result: as expected, its stock price rose sharply.
\end{itemize}
\item 2009: maiden flight after multiple delays.
\begin{itemize}
\item Problem: Boeing and its partners have had no experience with many new technologies.
\end{itemize}
\item 2011: Enter into service, but the problems did not go away.
\begin{itemize}
\item Result: The budget increased from US\$ 5billion to more than US\$ 30billion.
\end{itemize}
\item 2018: Problems did not go away but were overshadowed by the Boeing 737 MAX problems.
\item Lesson learned?
\end{itemize}
\end{frame}

% XXXXXXXXXXXXXXXXXXXXXXXXXXXXXXXXXXXXXXXXXXXXXXXXXXXXXXXXXXXXXXXXXXXXXXXXXX
\section{LFS: Linux From Scratch}
\begin{frame}
\frametitle{LFS: Linux From Scratch (Week 00 --- Week 10)}
\begin{itemize}
\item \href{https://youtu.be/jEoM3qan9Gs}{THIS IS HOW WE DOIT!}
\item \url{http://www.linuxfromscratch.org/lfs/view/stable/}
\item To build a GNU/Linux system from scratch (source code).
\item To learn a GNU/Linux system inside out.
\item To use a Virtual Machine.
\item A Chicken and Egg dependency problem:
\begin{itemize}
\item It would be best if you had the tools to build an Operating System.
\item You need an Operating System to build tools.
\item To build a cross-toolchain (compiler and its libraries).
\item To build cross utilities using the cross-toolchain.
\item To build an Operating System in a chroot environment.
\item To do iterations (if necessary).
\end{itemize}
\item How deep would you like to know of a ''real'' Operating System?
\item Whatever, however, from Week 00 to Week 10!
\item \textbf{YOU} decide!
\end{itemize}
\end{frame}

% XXXXXXXXXXXXXXXXXXXXXXXXXXXXXXXXXXXXXXXXXXXXXXXXXXXXXXXXXXXXXXXXXXXXXXXXXX
\section{What defines an Operating System? (The Three Layers Model)}
\begin{frame}
\frametitle{What defines an Operating System? (The Three Layers Model)}
\begin{multicols}{2}
\begin{table}
\scalebox{0.8}{%
\begin{tabular}{| c |}
\hline \\ [1pt]
Business Goal \\
\vline \\ [1pt]
Application \\
\vline \\ [1pt]
\hline
OS API \\
\vline \\ [1pt]
OS Managers and Utilities \\
\vline \\ [1pt]
OS Drivers \\
\hline
\vline \\ [1pt]
(Hypervisor) \\
\vline \\ [1pt]
Hardware \\ [1pt]
\hline
\end{tabular}}
\end{table}
  \vfill \null
\columnbreak
  \begin{itemize}
    \item The Three Layers Model
  \begin{itemize}
    \item An Operating System is between your Application and your Hardware (or Hypervisor).
  \begin{itemize}
    \item OS API: Application Programming Interface
    \item OS Resources Managers and Utilities: Process, Scheduler, Dispatcher, 
             (Virtual) Memory, Disk, I/O, Network, Security, Protection, etc.
    \item OS Device Drivers: controls devices
  \end{itemize}
    \item Remember that your future "\textbf{Business Goal}" may not directly relate to an Operating System at all!
  \end{itemize}
  \end{itemize}
  \vfill \null
\end{multicols}
\end{frame}

% XXXXXXXXXXXXXXXXXXXXXXXXXXXXXXXXXXXXXXXXXXXXXXXXXXXXXXXXXXXXXXXXXXXXXXXXXX
\section{OSC10 (Silberschatz) Chapter 1 and 2}
\begin{frame}
\frametitle{OSC10 (Silberschatz) Chapter 1 and 2}
\begin{multicols}{2}
  \begin{itemize}
  \item OSC10 Chapter 1
  \begin{itemize}
  \item What Operating Systems Do
  \item Computer-System Organization
  \item Computer-System Architecture
  \item Operating-System Operations
  \item Resource Management
  \item Security and Protection
  \item Virtualization
  \item Distributed Systems
  \item Kernel Data Structures
  \item Computing Environments
  \item Free/Libre and Open-Source Operating Systems
  \end{itemize}
  \end{itemize}
  \vfill \null
\columnbreak
  \begin{itemize}
  \item OSC10 Chapter 2
  \begin{itemize}
  \item Operating System Services
  \item User and Operating System-Interface
  \item System Calls
  \item System Services
  \item Linkers and Loaders
  \item Why Applications are Operating System Specific
  \item Operating-System Design and Implementation
  \item Operating System Structure
  \item Building and Booting an Operating System
  \item Operating System Debugging
  \end{itemize}
  \end{itemize}
  \vfill \null
\end{multicols}
\end{frame}

% XXXXXXXXXXXXXXXXXXXXXXXXXXXXXXXXXXXXXXXXXXXXXXXXXXXXXXXXXXXXXXXXXXXXXXXXXX
\begin{frame}
\frametitle{Remember Computer Organization (POK/DDAK)?}
\begin{itemize}
\item You should understand:
\begin{itemize}
\item von Neumann Model.
\item Buses, Bridges, Transfer Rate, Clock.
\item Memory: DDR, DDR-2, DDR-3, DDR-3+ ...
\item Cache, Buffer, Spool, \& Pipelining.
\item Direct Memory Access (DMA).
\item Port \& Memory Mapped I/O.
\item CPU: (privilege/kernel/supervisor mode) vs. (user mode).
\item Physical (Hardware) Limitation.
\item Priority: Read vs. Write.
\item Interrupts: Polling \& Vectored.
\item Multiprocessors: Symmetric vs. Asymmetric.
\item Multicore \& Multithreading.
\item Clustered Systems.
\item Numbers: base 2, base 8, base 10, base 16.
\begin{itemize}
\item Base 2: $110010101010_2$
\item Base 8: $01234567_8\ =\ 000\ 001\ 010\ 011\ 100\ 101\ 110\ 111_2$
\item Base 10: $012\ 345\ 679$
\item Base 16: $9AB\ CDEF_{16}\ =\ 1001\ 1010\ 1011\ \ 1100\ 1101\ 1110\ 1111_2$
\end{itemize}
\end{itemize}
\end{itemize}
\end{frame}

% XXXXXXXXXXXXXXXXXXXXXXXXXXXXXXXXXXXXXXXXXXXXXXXXXXXXXXXXXXXXXXXXXXXXXXXXXX
\begin{frame}
\frametitle{Physics 101: Signal Length (E.g. 3 GHz)}
\begin{figure}
\includegraphics[width=0.40\linewidth]{os-wave3}
\caption{What is the length of a 3 GHz signal?}
\end{figure}
\end{frame}

% XXXXXXXXXXXXXXXXXXXXXXXXXXXXXXXXXXXXXXXXXXXXXXXXXXXXXXXXXXXXXXXXXXXXXXXXXX
\begin{frame}
\frametitle{Physics 101: Safe Distance for 3 GHz}
\begin{figure}
\includegraphics[width=0.45\linewidth]{os-circle}
\caption{Safe Distance}
\end{figure}
\end{frame}

% XXXXXXXXXXXXXXXXXXXXXXXXXXXXXXXXXXXXXXXXXXXXXXXXXXXXXXXXXXXXXXXXXXXXXXXXXX
\begin{frame}
\frametitle{Physics 101: Serial vs. Parallel Transmission}
\begin{multicols}{2}
  \begin{itemize}
    \item Serial Transmission
  \begin{itemize}
    \item Longer Distance
    \item Easy to implement
  \end{itemize}
  \end{itemize}
  \includegraphics[width=0.99\linewidth]{os-wave4a}
  \vfill \null
\columnbreak
  \begin{itemize}
    \item Parallel Transmission
  \begin{itemize}
    \item Faster
    \item Not easy to implement
  \end{itemize}
  \end{itemize}
  \includegraphics[width=0.99\linewidth]{os-wave4b}
  \vfill \null
\end{multicols}
\end{frame}

% XXXXXXXXXXXXXXXXXXXXXXXXXXXXXXXXXXXXXXXXXXXXXXXXXXXXXXXXXXXXXXXXXXXXXXXXXX
\begin{frame}
\frametitle{Transmission Rate (E.g. \textbf{BUS}: 64 bit/133 MHz)}
\begin{figure}
\includegraphics[width=0.40\linewidth]{os-transfer-rate}
\end{figure}
\begin{itemize}
\item E.g. \textbf{BUS}: 64 bit, \textbf{Clock}: 133 MHz
\begin{itemize}
\item SDRAM (Synchronous Dynamic RAM): 1 transmission/cycle.\\
\textbf{Transfer Rate} = \texttt{64/8 byte x 133M x 1 = 1064 Mbyte/s}.
\item DDR (Double Data Rate): 2 transmission/cycle.\\
\textbf{Transfer Rate} = \texttt{64/8 byte x 133M x 2 = 2128 Mbyte/s}.
\item DDR-2 (Double Data Rate 2): 4 transmission/cycle.\\
\textbf{Transfer Rate} = \texttt{64/8 byte x 133M x 4 = 4256 Mbyte/s}.
\item DDR-3 (Double Data Rate 3): 8 transmission per cycle.\\
\textbf{Transfer Rate} = \texttt{64/8 byte x 133M x 8 = 8512 Mbyte/s}.
\item DDR-3+ = DDR-3 with a better clock rate, lower voltage, and greater capacity.
\end{itemize}
\end{itemize}
\end{frame}

% XXXXXXXXXXXXXXXXXXXXXXXXXXXXXXXXXXXXXXXXXXXXXXXXXXXXXXXXXXXXXXXXXXXXXXXXXX
\begin{frame}
\frametitle{CPU: SuperVisor Mode}
\begin{figure}
\includegraphics[width=0.50\linewidth]{os-super2user}
\caption{SuperVisor (Privilege) Mode to User Mode}
\end{figure}
\begin{itemize}
\item SuperVisor Mode
\begin{itemize}
\item A.k.a. Kernel Mode, Privilege Mode.
\item Initial STATE (Mode) of a CPU (Power On).
\item STATE (Mode) after Interrupt.
\item All operations are allowed, including to switch to User Mode!
\end{itemize}
\end{itemize}
\end{frame}

% XXXXXXXXXXXXXXXXXXXXXXXXXXXXXXXXXXXXXXXXXXXXXXXXXXXXXXXXXXXXXXXXXXXXXXXXXX
\begin{frame}
\frametitle{CPU: User Mode}
\begin{figure}
\includegraphics[width=0.50\linewidth]{os-user2super}
\caption{User Mode to SuperVisor (Privilege)}
\end{figure}
\begin{itemize}
\item User Mode
\begin{itemize}
\item It is not allowed to switch back to SuperVisor Mode.
\item It is not allowed to access I/O directly.
\item It is not allowed to modify the Interrupt Vector.
\item It is allowed to request Interrupt.
\end{itemize}
\end{itemize}
\end{frame}

% XXXXXXXXXXXXXXXXXXXXXXXXXXXXXXXXXXXXXXXXXXXXXXXXXXXXXXXXXXXXXXXXXXXXXXXXXX
\begin{frame}
\frametitle{Can you read a Block Diagram?}
\begin{figure}
\includegraphics[width=0.79\linewidth]{intel-chipset}
\caption{Block Diagram}
\end{figure}
\end{frame}

% XXXXXXXXXXXXXXXXXXXXXXXXXXXXXXXXXXXXXXXXXXXXXXXXXXXXXXXXXXXXXXXXXXXXXXXXXX
\begin{frame}
\frametitle{Block Diagram}
\begin{itemize}
  \item eDP: Embedded DisplayPort, for internal displays.
  \item DDI: Digital Display Interface
\begin{itemize}
  \item DP: Display Port
  \item HDMI: High-Definition Multimedia Interface
\end{itemize}
  \item DMI: Direct Media Interface
  \item PCIe: Peripheral Component Interconnect express
  \item eSPI: Enhanced Serial Peripheral Interface
  \item SPI: Serial Peripheral Interface
  \item SMBus: System Management Bus
  \item HD Audio: High Definition Audio
  \item USB: Universal Serial Bus
\end{itemize}
\end{frame}

% XXXXXXXXXXXXXXXXXXXXXXXXXXXXXXXXXXXXXXXXXXXXXXXXXXXXXXXXXXXXXXXXXXXXXXXXXX
\begin{frame}
\frametitle{What is an APIC?!}
\begin{figure}
\includegraphics[width=0.44\linewidth]{os00-xapic}
\caption{APIC (Advanced Programmable Interrupt Controller)}
\end{figure}
\end{frame}

% XXXXXXXXXXXXXXXXXXXXXXXXXXXXXXXXXXXXXXXXXXXXXXXXXXXXXXXXXXXXXXXXXXXXXXXXXX
\begin{frame}
\frametitle{And, what is ''Interrupt Handling''?}
\begin{figure}
\includegraphics[width=0.40\linewidth]{os00-int-protection}
\caption{Interrupt Handling with PIC (Programmable Interrupt Controller)}
\end{figure}
\end{frame}

% XXXXXXXXXXXXXXXXXXXXXXXXXXXXXXXXXXXXXXXXXXXXXXXXXXXXXXXXXXXXXXXXXXXXXXXXXX
\begin{frame}
\frametitle{The Operating System Managers}
\begin{itemize}
\item Process Manager: 
\begin{itemize}
\item Creating/Deleting; Suspending/Resuming; Synchronization; Communication; Scheduling
\end{itemize}
\item Memory Manager:
\begin{itemize}
\item Tracking; Move In/Move Out; Allocating/Deallocating.
\end{itemize}
\item Storage/File System Manager:
\begin{itemize}
\item Create/Delete; Open/Close; Read/Write.
\end{itemize}
\item Mass Storage Manager:
\begin{itemize}
\item Scheduling; Allocating; Free Space.
\end{itemize}
\item I/O Manager:
\begin{itemize}
\item Buffering; Caching; Spooling.
\item Interfacing (driving).
\end{itemize}
\item Protecting \& Security Manager:
\begin{itemize}
\item Protecting.
\item Security.
\end{itemize}
\end{itemize}
\end{frame}

% XXXXXXXXXXXXXXXXXXXXXXXXXXXXXXXXXXXXXXXXXXXXXXXXXXXXXXXXXXXXXXXXXXXXXXXXXX
\begin{frame}
\frametitle{Any idea what these following terms mean?!}
\begin{itemize}
\item Scripting: bash, regex, sed, awk
\item Security and Protection
\item File System
\item Data Structure in a (logical) Memory
\item Virtual Memory
\item Concurrency
\item Synchronization
\item Mass Storage
\item UEFI, GRUB, and systemd
\item I/O
\item I/O Programming
\end{itemize}
\end{frame}

% XXXXXXXXXXXXXXXXXXXXXXXXXXXXXXXXXXXXXXXXXXXXXXXXXXXXXXXXXXXXXXXXXXXXXXXXXX
\begin{frame}
\frametitle{Week 00: QUIZ Example \#1 (from OSC2e)}
\begin{multicols}{2}

\includegraphics[width=0.97\linewidth]{os00-osc2e}

\columnbreak
  \null \vfill 

  \textbf{True or False?}

  The advantages of a multiprocessor system include: 
  increased throughput, economy of scale, and increased reliability.

  {\footnotesize (from MidTerm 2016)}
  \vfill \null
\end{multicols}
\end{frame}

% XXXXXXXXXXXXXXXXXXXXXXXXXXXXXXXXXXXXXXXXXXXXXXXXXXXXXXXXXXXXXXXXXXXXXXXXXX
\begin{frame}
\frametitle{Week 00: QUIZ Example \#2 (from OSC10)}

  \textbf{Preface}

  Operating systems are an essential part of any computer system. 
  Similarly, a course on operating systems is an essential part of any computer science education. 
  This field is undergoing rapid change, as computers are now prevalent in virtually 
  every arena of day-to-day life—from embedded devices in automobiles through the most 
  sophisticated planning tools for governments and multinational firms. 
  Yet the fundamental concepts remain fairly clear, and it is on these that we base this book.

  \begin{itemize}
  \item \textbf{T/F} 
     Operating systems are an essential part of any computer system.
  \item \textbf{T/F} 
     Operating systems are not an essential part of any computer system.
  \item \textbf{T/F} 
     A course on operating systems is essential to any computer science education.
  \item \textbf{T/F} 
     A course on operating systems is optional to any computer science education.
  \item \textbf{T/F} 
     The Operating System field is undergoing rapid change, as computers are now prevalent 
     in virtually every arena of day-to-day life.
  \item \textbf{T/F} 
     The Operating System field is not undergoing rapid change, as computers are now prevalent in virtual machines.
  \end{itemize}

\end{frame}

% XXXXXXXXXXXXXXXXXXXXXXXXXXXXXXXXXXXXXXXXXXXXXXXXXXXXXXXXXXXXXXXXXXXXXXXXXX
\begin{frame}
\frametitle{Week 00: QUIZ Example \#3}
\begin{itemize}
\item \textbf{TRUE/FALSE}\\
      The best way to get any help is to send an email to \texttt{rms46 AT ui.ac.id}.
\item \textbf{TRUE/FALSE}\\
      Questions regarding assignments should be posted at SCELE.
\item \textbf{TRUE/FALSE}\\
      Making a \textbf{PUSS IN BOOT} face is increasing the chance to get a better deal.
\item \textbf{TRUE/FALSE}\\
      Anyone can appeal any time, even after the (official) final grade is announced on SIAK.
\item \textbf{TRUE/FALSE}\\
      There are bonus points for early assignment submission.
\end{itemize}
\end{frame}

% XXXXXXXXXXXXXXXXXXXXXXXXXXXXXXXXXXXXXXXXXXXXXXXXXXXXXXXXXXXXXXXXXXXXXXXXXX
\section{Assignments}
\begin{frame}[fragile]
\frametitle{Assignments}
\begin{itemize}
\item You need to run ''VirtualBox'' or ''UTM'' on a computer with more than 4GB RAM and at least 64 GB disk space.
\item Each weekly assignment will be due seven days after it is announced.
      The weekly schedule will be at \url{https://os.vlsm.org/\#idx02}.
\item Use the \textbf{''GitHub web interface''} for the Week 00 assignment.
      However, starting Week 01, you need to understand \textbf{''pull, add, commit, push, and ssh-keys''}.
\item Submit (push) the assignments to \url{https://github.com/}.
      If you still don't have one, you must sign up for a \href{https://github.com/}{GitHub} account.
      More information will follow.
\item See the assignment list at \url{https://demos.vlsm.org/#idx000}.
\end{itemize}
\end{frame}

% XXXXXXXXXXXXXXXXXXXXXXXXXXXXXXXXXXXXXXXXXXXXXXXXXXXXXXXXXXXXXXXXXXXXXXXXXX
\section{Course Highlights and Syllabus} 
\begin{frame}
\frametitle{Course Highlights and Syllabus}
\begin{block}{Coverage}
This is an introduction to a modern operating systems course. 
It will cover
general overview,
computer architecture review,
operating system overview,
GNU/Linux CLI,
scripting,
C language overview,
protection,
security,
privacy,
systemd,
I/O,
addressing and pointers,
memory management, 
processes and threads, 
virtual memory,
synchronization,
mutual exclusion, 
deadlock, 
CPU scheduling algorithms, 
file systems,
and
I/O programming.
\end{block}

\begin{block}{Student-Centered}
This course is student-centered where responsibility is
in the hands of the students. Students are expected to 
be prepared for the class meeting.
\end{block}

\begin{block}{GNU/Linux}
Students will have a thorough understanding of how GNU/Linux 
provides services by using a Command Line Interface.
\end{block}

\end{frame}


% XXXXXXXXXXXXXXXXXXXXXXXXXXXXXXXXXXXXXXXXXXXXXXXXXXXXXXXXXXXXXXXXXXXXXXXXXX
\input{os00-BRP.tex}
\input{os01-BRP.tex}
\input{os02-BRP.tex}
\input{os03-BRP.tex}
\input{os04-BRP.tex}
\input{os05-BRP.tex}
\input{os06-BRP.tex}
\input{os07-BRP.tex}
\input{os08-BRP.tex}
\input{os09-BRP.tex}
\input{os10-BRP.tex}

% XXXXXXXXXXXXXXXXXXXXXXXXXXXXXXXXXXXXXXXXXXXXXXXXXXXXXXXXXXXXXXXXXXXXXXXXXX

\section{Week 00: Summary}
\begin{frame}
\frametitle{Week 00: Summary}
\begin{itemize}
\item What is an Operating System?
\begin{itemize}
\item Definition: Resource Allocator \& Control Program.
\item Why taking an Operating System class?
\end{itemize}
\item Computer Organization Review
\item The Manager Set
\begin{itemize}
\item Process Manager, Memory Manager, I/O Manager, Storage Manager.
\end{itemize}
\item Security and Protection
\item Virtualization
\begin{itemize}
\item Hypervisor type 0, 1, 2
\item Paravirtualization, Emulators, Containers.
\item VCPU: Virtual CPU
\item Virtualization Implementation:
\begin{itemize}
\item Trap-and-Emulate mode
\item Binary Translation mode
\end{itemize}
\end{itemize}
\end{itemize}
\end{frame}

% XXXXXXXXXXXXXXXXXXXXXXXXXXXXXXXXXXXXXXXXXXXXXXXXXXXXXXXXXXXXXXXXXXXXXXXXXX

\input{os00-TIPS.tex}

% 12 XXXXXXXXXXXXXXXXXXXXXXXXXXXXXXXXXXXXXXXXXXXXXXXXXXXXXXXXXXXXXXXXXXXXXXX
% XXXXXXXXXXXXXXXXXXXXXXXXXXXXXXXXXXXXXXXXXXXXXXXXXXXXXXXXXXXXXXXXXXXXXXXXXX
% \section{The End}
\begin{frame}
\frametitle{The End}
\begin{itemize}
\item[$\square$] This is the end of the presentation.
\item[$\boxtimes$] This is the end of the presentation.
\item This is the end of the presentation.
\end{itemize}
\end{frame}

% XXXXXXXXXXXXXXXXXXXXXXXXXXXXXXXXXXXXXXXXXXXXXXXXXXXXXXXXXXXXXXXXXXXXXXXXXX
\end{document}

