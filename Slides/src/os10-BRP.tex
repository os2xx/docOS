%%%%%%%%%%%%%%%%%%%%%%%%%%%%%%%
% REV412: Tue 22 Aug 2023 14:00
% REV406: Sat 05 Aug 2023 12:00
% REV171: Thu 22 Nov 2018 20:00
% REV154: Thu 23 Aug 2018 11:00
% START0: Thu 26 Jul 2018 20:00
%%%%%%%%%%%%%%%%%%%%%%%%%%%%%%%

\section{Week 10}
\begin{frame}[fragile]
\frametitle{Week 10 I/O \& Programming:
Topics\footnote{Source: ACM IEEE CS Curricula}}

\begin{itemize}
\item 
Characteristics of serial and parallel devices
\item 
Abstracting device differences
\item 
Buffering strategies
\item 
Direct memory access
\item 
Recovery from failures
\item
I/O Programming
\item
Network Programming
\end{itemize}
\end{frame}

\begin{frame}[fragile]
\frametitle{Week 10 I/O \& Programming:
Learning Outcomes\footnote{Source: ACM IEEE CS Curricula}}

\begin{itemize}
\item Explain the key difference between serial and parallel devices and identify the conditions in which each is appropriate. [Familiarity]
\item Identify the relationship between the physical hardware and the virtual devices maintained by the operating system. [Usage]
\item Explain buffering and describe strategies for implementing it. [Familiarity]
\item Differentiate the mechanisms used in interfacing a range of devices (including hand-held devices, networks, multimedia) to a computer and explain the implications of these for the design of an operating system. [Usage]
\item Describe the advantages and disadvantages of direct memory access and discuss the circumstances in
which its use is warranted. [Usage]
\item Identify the requirements for failure recovery. [Familiarity]
\item Implement a simple device driver for a range of possible devices. [Usage]

\item I/O Programming [Usage]
\item Network Programming [Usage]
\end{itemize}

\end{frame}



