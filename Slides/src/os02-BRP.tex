%%%%%%%%%%%%%%%%%%%%%%%%%%%%%%%
% REV412: Tue 22 Aug 2023 14:00
% REV406: Sat 05 Aug 2023 12:00
% REV346: Sun 12 Sep 2021 17:00
% REV154: Thu 23 Aug 2018 11:00
% START0: Thu 26 Jul 2018 20:00
%%%%%%%%%%%%%%%%%%%%%%%%%%%%%%%

\section{Week 02 Security \& Protection}
\begin{frame}[fragile]
\frametitle{Week 02 Security \& Protection:
Topics\footnote{Source: ACM IEEE CS Curricula}}

\begin{itemize}
\item Overview of system security 
\item Cyber Security Introduction
\item Policy/mechanism separation 
\item Security methods and devices 
\item Protection, access control, and authentication 
\item Backups 
\item Safety and Privacy
\item Threads
\item Cryptography: (Symmetric and Asymmetric) Encryption,
\item C Language
\end{itemize}

\end{frame}

\begin{frame}[fragile]
\frametitle{Week 02 Security \& Protection:
Learning Outcomes\footnote{Source: ACM IEEE CS Curricula}}

\begin{itemize}
\item Articulate the need for protection and security in an OS (cross-reference IAS/Security Architecture and Systems Administration/Investigating Operating Systems Security for various systems). [Assessment]
\item Summarize the features and limitations of an operating system used to provide protection and security [Familiarity] 
\item Explain the mechanisms available in an OS to control access to resources [Familiarity] 
\item Carry out simple system administration tasks according to a security policy, for example creating accounts, setting permissions, applying patches, and arranging for regular backups [Usage] 
\end{itemize}
\end{frame}

