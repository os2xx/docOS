%%%%%%%%%%%%%%%%%%%%%%%%%%%%%%%%%%%%%%%%%%%%%%%%%%%%%%%%%%%%%%%%%%%%%%%%
% Beamer Presentation - LaTeX - Template Version 1.0 (10/11/12)
% This template has been downloaded from: http://www.LaTeXTemplates.com
% License: % CC BY-NC-SA 3.0 (http://creativecommons.org/)
% Modified by Rahmat M. Samik-Ibrahim

% REV417: Sun 28 Jan 2024 16:00
% REV398: Wed 02 Nov 2022 13:00
% REV386: Thu 28 Jul 2022 10:00
% REV371: Mon 28 Feb 2022 10:00
% REV363: Mon 20 Dec 2021 15:00
% STARTX: Wed 14 Sep 2016 10:00
%%%%%%%%%%%%%%%%%%%%%%%%%%%%%%%%%%%%%%%%%%%%%%%%%%%%%%%%%%%%%%%%%%%%%%%%%

% PACKAGES AND THEMES 
\documentclass[aspectratio=169, xcolor=table, notheorems, hyperref={pdfpagelabels=false}]{beamer}
%%%%%%%%%%%%%%%%%%%%%%%%%%%%%%%%%%%%%%%%%%%%%%%%%%%%%%%%%%%%%%%%%%%%%%%%
% Beamer Presentation - LaTeX - Template Version 1.0 (10/11/12)
% This template has been downloaded from: http://www.LaTeXTemplates.com
% License: % CC BY-NC-SA 3.0 (http://creativecommons.org/)
% Modified by Bin Kadal, Sdn Bhd.
% REV023: Tue 30 Jan 2024 16:00
% REV006: Mon 22 Jan 2018 19:00
% STARTX: Thu 25 Aug 2016 14:00
%%%%%%%%%%%%%%%%%%%%%%%%%%%%%%%%%%%%%%%%%%%%%%%%%%%%%%%%%%%%%%%%%%%%%%%%%

%% ZCZC NNNN
\newtheorem{example}{Example}

%%%%%%%%%%%%%%%%%%%%%%%%%%%%%%%%%%%%%%%%%%%%%%%%%%%%%%%%%%%%%%%%%%%%%%%%%

\let\Tiny=\tiny
\mode<presentation> {
% The Beamer class comes with a number of default slide themes
% which change the colors and layouts of slides. Below this is a list
% of all the themes, uncomment each in turn to see what they look like.
%\usetheme{Boadilla}
\usetheme{Madrid}
% ZCZC %%%%%%%%%%%%%%%%%%%%%%%%%%%%%%%%%%%%%%%%%%%%%%%%%%%%%%%%%%%%%%%%%%
% \usetheme{default} \usetheme{AnnArbor} \usetheme{Antibes} \usetheme{Bergen}
% \usetheme{Berkeley} \usetheme{Berlin} \usetheme{CambridgeUS} 
% \usetheme{Copenhagen} \usetheme{Darmstadt} \usetheme{Dresden}
% \usetheme{Frankfurt} \usetheme{Goettingen} \usetheme{Hannover}
% \usetheme{Ilmenau} \usetheme{JuanLesPins} \usetheme{Luebeck}
% \usetheme{Malmoe} \usetheme{Marburg} \usetheme{Montpellier}
% \usetheme{PaloAlto} \usetheme{Pittsburgh} \usetheme{Rochester}
% \usetheme{Singapore} \usetheme{Szeged} \usetheme{Warsaw}
% NNNN %%%%%%%%%%%%%%%%%%%%%%%%%%%%%%%%%%%%%%%%%%%%%%%%%%%%%%%%%%%%%%%%%%
% As well as themes, the Beamer class has a number of color themes
% for any slide theme. Uncomment each of these in turn to see how it
% changes the colors of your current slide theme.
%\usecolortheme{orchid}
%\usecolortheme{rose}
%\usecolortheme{seagull}
%\usecolortheme{seahorse}
\usecolortheme{whale}
% ZCZC %%%%%%%%%%%%%%%%%%%%%%%%%%%%%%%%%%%%%%%%%%%%%%%%%%%%%%%%%%%%%%%%%%
%\usecolortheme{albatross} \usecolortheme{beaver} \usecolortheme{beetle}
%\usecolortheme{crane} \usecolortheme{dolphin} \usecolortheme{dove}
%\usecolortheme{fly} \usecolortheme{lily} \usecolortheme{wolverine}
% NNNN %%%%%%%%%%%%%%%%%%%%%%%%%%%%%%%%%%%%%%%%%%%%%%%%%%%%%%%%%%%%%%%%%%
% To remove the footer line in all slides uncomment this line
%\setbeamertemplate{footline} 
% To replace the footer line in all slides uncomment this line
%\setbeamertemplate{footline}[page number] 
% To remove the navigation symbols from the bottom uncomment this line
\setbeamertemplate{navigation symbols}{} 
}

\usepackage{array}       % ZCZC
\usepackage{amssymb}     % ZCZC
\usepackage{bold-extra}  % ZCZC
\usepackage{booktabs}    % Allows \toprule, \midrule and \bottomrule in tables
\usepackage{caption}
\usepackage[T1]{fontenc} % ZCZC << >>
\usepackage{graphicx}    % Allows including images
\usepackage{listings}    % listing
\usepackage{lmodern}     % ZCZC
\usepackage{perpage}     % reset footnote per page
\usepackage{geometry}    % ZCZC
\usepackage{adjustbox}   % ZCZC
\usepackage{multirow}    % ZCZC

% \definecolor{links}{HTML}{2A1B81}
\definecolor{links}{HTML}{0011FF}
\hypersetup{colorlinks,linkcolor=,urlcolor=links}

% \usepackage{xcolor}
% \usepackage[colorlinks = true,
%             linkcolor = blue,
%             urlcolor  = blue,
%             citecolor = blue,
%             anchorcolor = blue]{hyperref}

\captionsetup[table]{name=Tabel}
\makeatletter
\def\input@path{{src/}}
\makeatother
\graphicspath{{src/}}      % src directory
\MakePerPage{footnote}     % reset page

% NNNN %%%%%%%%%%%%%%%%%%%%%%%%%%%%%%%%%%%%%%%%%%%%%%%%%%%%%%%%%%%%%%%%%%

%% % XXXXXXXXXXXXXXXXXXXXXXXXXXXXXXXXXXXXXXXXXXXXXXXXXXXXXXXXXXXXXXXXXXXXXXXXXX
%% % The short title appears at the bottom of every slide, 
%% % the full title is only on the title page
%% \title[Judul Pendek]{Judul Panjang dan Lengkap} 
%% \author{Cecak bin Kadal}
%% \institute[UILA]
%% {
%% University of Indonesia at Lenteng Agung \\ 
%% \medskip
%% \textit{cecak@binKadal.com}
%% }
%% \date{REV00 24-Aug-2016}
%% % \date{\today}
%% 

%% % XXXXXXXXXXXXXXXXXXXXXXXXXXXXXXXXXXXXXXXXXXXXXXXXXXXXXXXXXXXXXXXXXXXXXXXXXX
%% \begin{document}
%% \section{Judul}
%% \begin{frame}
%% \titlepage
%% \end{frame}
%% 
%% % XXXXXXXXXXXXXXXXXXXXXXXXXXXXXXXXXXXXXXXXXXXXXXXXXXXXXXXXXXXXXXXXXXXXXXXXXX
%% \section{Agenda}
%% \begin{frame}
%% \frametitle{Agenda}
%% % Throughout your presentation, if you choose to use \section{} and 
%% % \subsection{} commands, these will automatically be printed on 
%% % this slide as an overview of your presentation
%% \tableofcontents 
%% \end{frame}
%% 
%% % XXXXXXXXXXXXXXXXXXXXXXXXXXXXXXXXXXXXXXXXXXXXXXXXXXXXXXXXXXXXXXXXXXXXXXXXXX
%% \section{UUD dan Pancasila}
%% \subsection{UUD}
%% \begin{frame}
%% \frametitle{Pembukaan}
%% Bahwa sesungguhnya kemerdekaan itu ialah hak segala bangsa dan oleh 
%% sebab itu, maka penjajahan diatas dunia harus dihapuskan karena 
%% tidak sesuai dengan perikemanusiaan dan perikeadilan.
%% \\~\\
%% Atas berkat rahmat Allah Yang Maha Kuasa dan dengan didorongkan oleh 
%% keinginan luhur, supaya berkehidupan kebangsaan yang bebas, maka 
%% rakyat Indonesia menyatakan dengan ini kemerdekaannya.
%% \end{frame}
%% 
%% % XXXXXXXXXXXXXXXXXXXXXXXXXXXXXXXXXXXXXXXXXXXXXXXXXXXXXXXXXXXXXXXXXXXXXXXXXX
%% \begin{frame}
%% \frametitle{Alenia Ketiga}
%% Kemudian daripada itu untuk membentuk suatu pemerintah negara Indonesia 
%% yang melindungi segenap bangsa Indonesia dan seluruh tumpah darah Indonesia 
%% dan untuk memajukan kesejahteraan umum, mencerdaskan kehidupan bangsa, dan 
%% ikut melaksanakan ketertiban dunia yang berdasarkan kemerdekaan, perdamaian 
%% abadi dan keadilan sosial, maka disusunlah kemerdekaan kebangsaan Indonesia 
%% itu dalam suatu Undang-Undang Dasar negara Indonesia, yang terbentuk dalam 
%% suatu susunan negara Republik Indonesia yang berkedaulatan rakyat dengan 
%% berdasar kepada:
%% \begin{itemize}
%% \item Ketuhanan Yang Maha Esa,
%% \item kemanusiaan yang adil dan beradab,
%% \item persatuan Indonesia,
%% \item dan kerakyatan yang dipimpin oleh hikmat kebijaksanaan 
%%       dalam permusyawaratan/ perwakilan,
%% \item serta dengan mewujudkan suatu keadilan sosial bagi seluruh rakyat 
%%       Indonesia.
%% \end{itemize}
%% \end{frame}
%% 
%% % XXXXXXXXXXXXXXXXXXXXXXXXXXXXXXXXXXXXXXXXXXXXXXXXXXXXXXXXXXXXXXXXXXXXXXXXXX
%% \subsection{Pancasila}
%% \begin{frame}
%% \frametitle{Tujuh Kunci Pokok}
%% \begin{block}{Pertama - Kedua - Ketiga}
%% Indonesia ialah negara berdasarkan hukum.
%% Sistem konstitusional.
%% Kekuasaan negara tertinggi di tangan MPR.
%% \end{block}
%% 
%% \begin{block}{Keempat - Kelima}
%% Presiden adalah penyelenggara pemerintahan tertinggi di bawah MPR.
%% Adanya pengawasan DPR.
%% \end{block}
%% 
%% \begin{block}{Keenam}
%% Menteri negara adalah pembantu presiden dan tidak bertanggung jawab 
%% kepada DPR.
%% \end{block}
%% 
%% \begin{block}{Ketujuh}
%% Kekuasaan kepala negara tidak tak tebatas.
%% \end{block}
%% 
%% \end{frame}
%% 
%% % XXXXXXXXXXXXXXXXXXXXXXXXXXXXXXXXXXXXXXXXXXXXXXXXXXXXXXXXXXXXXXXXXXXXXXXXXX
%% \section{Rupa-rupa}
%% \subsection{Kolom}
%% \begin{frame}
%% \frametitle{Kolom}
%% % The "c" option specifies centered vertical alignment 
%% % while the "t" option is used for top vertical alignment
%% \begin{columns}[c] 
%% % Left column and width
%% \column{.45\textwidth} 
%% \textbf{Heading}
%% \begin{enumerate}
%% \item Satu-satu
%% \item Dua-dua
%% \item Tiga-tiga
%% \item Satu-dua-tiga
%% \end{enumerate}
%% 
%% % Right column and width
%% \column{.5\textwidth}
%% Satu-satu~\dots{} aku sayang ibu!
%% Dua-dua~\ldots{} juga sayang ayah!
%% Tiga-tiga~\ldots{} sayang adik kakak!
%% Satu-dua-tiga~\ldots{} sayang semuanya!
%% 
%% \end{columns}
%% \end{frame}
%% 
%% % XXXXXXXXXXXXXXXXXXXXXXXXXXXXXXXXXXXXXXXXXXXXXXXXXXXXXXXXXXXXXXXXXXXXXXXXXX
%% \subsection{Tabel}
%% \begin{frame}
%% \frametitle{Tabel}
%% \begin{table}
%% \begin{tabular}{l l l}
%% \toprule
%% \textbf{Nama} & \textbf{NPM} & \textbf{Tanggal Lahir}\\
%% \midrule
%% Cecak bin Kadal & 1234567890 & 1 Jan 2015 \\
%% Aneh bin Ajaib  & 0987654321 & 31 Des 2014 \\
%% \bottomrule
%% \end{tabular}
%% \caption{Keterangan Tabel}
%% \end{table}
%% \end{frame}
%% 
%% % XXXXXXXXXXXXXXXXXXXXXXXXXXXXXXXXXXXXXXXXXXXXXXXXXXXXXXXXXXXXXXXXXXXXXXXXXX
%% \subsection{Teori}
%% \begin{frame}
%% \frametitle{Teori}
%% \begin{theorem}[Teori Satu Batu]
%% $E = mc^2$
%% \end{theorem}
%% \end{frame}
%% 
%% % XXXXXXXXXXXXXXXXXXXXXXXXXXXXXXXXXXXXXXXXXXXXXXXXXXXXXXXXXXXXXXXXXXXXXXXXXX
%% \subsection{Verbatim}
%% % Need to use the fragile option when verbatim is used in the slide
%% \begin{frame}[fragile] 
%% \frametitle{Verbatim}
%% \begin{example}[Teori Satu Batu]
%% \begin{verbatim}
%% \begin{theorem}[Teori Satu Batu]
%% $E = mc^2$
%% \end{theorem}
%% \end{verbatim}
%% \end{example}
%% \end{frame}
%% 
%% % XXXXXXXXXXXXXXXXXXXXXXXXXXXXXXXXXXXXXXXXXXXXXXXXXXXXXXXXXXXXXXXXXXXXXXXXXX
%% \subsection{Gambar}
%% \begin{frame}
%% \frametitle{Gambar}
%% \begin{figure}
%% \includegraphics[width=0.5\linewidth]{2}
%% \caption{Ini Gambar JPG}
%% \end{figure}
%% \end{frame}
%% 
%% % XXXXXXXXXXXXXXXXXXXXXXXXXXXXXXXXXXXXXXXXXXXXXXXXXXXXXXXXXXXXXXXXXXXXXXXXXX
%% \subsection{Rujukan}
%% % Need to use the fragile option when verbatim is used in the slide
%% \begin{frame}[fragile] 
%% \frametitle{Rujukan dan Kutipan}
%% Contoh penggunaan \verb|\cite| ketika mengutip\cite{p1}.
%% Perhatian: Beamer tidak mengerti \verb|\BibTeX|~\ldots
%% \footnotesize{
%%   \begin{thebibliography}{99} 
%%   \bibitem[Smith, 2012]{p1} John Smith (2012)
%%      \newblock Katak dalam Tempurung
%%      \newblock \emph{Jurnal Kelapa dan Amfibi} 12(3), 45 -- 678.
%%   \end{thebibliography}
%% }
%% \end{frame}
%% 
%% % XXXXXXXXXXXXXXXXXXXXXXXXXXXXXXXXXXXXXXXXXXXXXXXXXXXXXXXXXXXXXXXXXXXXXXXXXX
%% \subsection{Selesai}
%% \begin{frame}
%% \Huge{\centerline{Selesai}}
%% \end{frame}
%% 
%% % XXXXXXXXXXXXXXXXXXXXXXXXXXXXXXXXXXXXXXXXXXXXXXXXXXXXXXXXXXXXXXXXXXXXXXXXXX
%% \end{document}

\newcommand{\revision}{%
REV424: Tue 03 Sep 2024 20:00
}
% w! tmptmp
% REV424: Tue 03 Sep 2024 20:00
% REV419: Wed 24 Jul 2024 17:00
% REV409: Tue 08 Aug 2023 12:00
% REV399: Fri 03 Feb 2023 20:00
% REV339: Sat 04 Sep 2021 12:00
% STARTS: Wed 24 Aug 2016 19:00
%%%%%%%%%%%%%%%%%%%%%%%%%%%%%%%%%%%%%
\newcommand{\kopikopi}{\textcopyright{}2016-2024 CBKadal + VauLSMorg}



% XXXXXXXXXXXXXXXXXXXXXXXXXXXXXXXXXXXXXXXXXXXXXXXXXXXXXXXXXXXXXXXXXXXXXXXXXX
% The short title appears at the bottom of every slide, 
% the full title is only on the title page
% \date{\today}
\title[\kopikopi]
{CSGE602055 Operating Systems \\ 
CSF2600505 Sistem Operasi \\
Week 06:
Concurrency: Processes \& Threads}
\author{C. BinKadal}
\institute[SdnBhd]
{
Sendirian Berhad\\
\medskip
\url{https://docOS.vlsm.org/Slides/os06.pdf}
\\ \texttt{Always check for the latest revision!}
}
\date{\revision}

% XXXXXXXXXXXXXXXXXXXXXXXXXXXXXXXXXXXXXXXXXXXXXXXXXXXXXXXXXXXXXXXXXXXXXXXXXX
\begin{document}

\lstset{
basicstyle=\ttfamily\tiny, % \tiny \small \footnotesize 
breakatwhitespace=true,
language=C,
columns=fullflexible,
keepspaces=true,
breaklines=true,
tabsize=3, 
showstringspaces=false,
extendedchars=true}

\section{Start}
\begin{frame}
\titlepage
\end{frame}

% XXXXXXXXXXXXXXXXXXXXXXXXXXXXXXXXXXXXXXXXXXXXXXXXXXXXXXXXXXXXXXXXXXXXXXXXXX

%%%%%%%%%%%%%%%%%%%%%%%%%%%%%%%%%%%%%%%%%%%%%%%%%%%%%%%%%%%%%%%%%%%%%%%%%
% REV418: Tue 30 Jan 2024 22:00
% REV406: Sat 05 Aug 2023 14:00
% REV399: Thu 02 Feb 2023 00:00
% REV369: Mon 14 Feb 2022 09:00
% REV328: Sat 14 Aug 2021 06:00
% STARTX: Wed 14 Sep 2016 10:00
%%%%%%%%%%%%%%%%%%%%%%%%%%%%%%%%%%%%%%%%%%%%%%%%%%%%%%%%%%%%%%%%%%%%%%%%%

\begin{frame}[fragile]
\section{OS241 Schedule}
\frametitle{OS241\footnote{%
) This information will be on \textbf{EVERY} page two (2) of this course material.}): 
Operating Systems Schedule 2023 - 2}

\vspace{5pt}

\scalebox{0.99}{%
\begin{tabular}{|c|c|l|l|}
\hline
\textbf{Week} & 
\textbf{Topic}\footnote{%
) For schedule, see \url{https://os.vlsm.org/\#idx02}}) & \textbf{OSC10}\footnote{%
    ) Silberschatz et. al.: \textbf{Operating System Concepts}, $10^{th}$ Edition, 2018.}) \\
\hline
Week 00  & Overview (1), Assignment of Week 00           & Ch. 1, 2      \\
Week 01  & Overview (2), Virtualization \& Scripting     & Ch. 1, 2, 18. \\
Week 02  & Security, Protection, Privacy, \& C-language. & Ch. 16, 17.   \\
Week 03  & File System \& FUSE  & Ch. 13, 14, 15.                        \\
Week 04  & Addressing, Shared Lib, \& Pointer & Ch. 9. \\
Week 05  & Virtual Memory & Ch. 10. \\
\hline
Week 06  & Concurrency: Processes \& Threads & Ch. 3, 4. \\
Week 07  & Synchronization \& Deadlock & Ch. 6, 7, 8. \\
Week 08  & Scheduling + W06/W07 & Ch. 5. \\
Week 09  & Storage, Firmware, Bootloader, \& Systemd & Ch. 11. \\
Week 10  & I/O \& Programming & Ch. 12. \\%
% MidTerm  & 00 XXX 2020 (XX:XX-XX:XX) & MidTerm (UTS) & \cellcolor{red!44} TBA! \\
% Reserved & 00 XXX - 00 XXX 2020 & Q \& A & \\
% Final    & 00 XXX 2020 XX:XX & First Part Final  (UAS tahap I)  & \cellcolor{red!44} This schedule is   \\
% Extra    & NA & No Extra assignment & \cellcolor{red!44} subject to change. \\
\hline
\end{tabular}
}
\end{frame}

\begin{frame}[fragile]
\frametitle{\textbf{STARTING POINT} --- 
{
\definecolor{links}{HTML}{FDEE00}
\hypersetup{colorlinks,linkcolor=,urlcolor=links}
\url{https://os.vlsm.org/}
}
}
\begin{itemize}
\item[$\square$] \textbf{Text Book} ---
                 Any recent/decent OS book. Eg. (\textbf{OSC10}) Silberschatz et. al.: 
                 \textbf{Operating System Concepts}, $10^{th}$ Edition, 2018.
                 (See \url{https://codex.cs.yale.edu/avi/os-book/OS10/}).
\item[$\square$] \textbf{Resources ({\footnotesize \url{https://os.vlsm.org/\#idx03}})}
\begin{itemize}
\item[$\square$] \href{https://scele.cs.ui.ac.id/course/view.php?id=3743}{\textbf{SCELE}} ---
\url{https://scele.cs.ui.ac.id/course/view.php?id=3743}.\\
The enrollment key is \textbf{XXX}.
\item[$\square$] \textbf{Download Slides and Demos from GitHub.com} --- (\url{https://github.com/os2xx/docOS/})\\
                 {\scriptsize%
                 \href{https://docOS.vlsm.org/Slides/os00.pdf}{\texttt{os00.pdf} (W00)},
                 \href{https://docOS.vlsm.org/Slides/os01.pdf}{\texttt{os01.pdf} (W01)},
                 \href{https://docOS.vlsm.org/Slides/os02.pdf}{\texttt{os02.pdf} (W02)},
                 \href{https://docOS.vlsm.org/Slides/os03.pdf}{\texttt{os03.pdf} (W03)},
                 \href{https://docOS.vlsm.org/Slides/os04.pdf}{\texttt{os04.pdf} (W04)},
                 \href{https://docOS.vlsm.org/Slides/os05.pdf}{\texttt{os05.pdf} (W05)},\\
                 \href{https://docOS.vlsm.org/Slides/os06.pdf}{\texttt{os06.pdf} (W06)},
                 \href{https://docOS.vlsm.org/Slides/os07.pdf}{\texttt{os07.pdf} (W07)},
                 \href{https://docOS.vlsm.org/Slides/os08.pdf}{\texttt{os08.pdf} (W08)},
                 \href{https://docOS.vlsm.org/Slides/os09.pdf}{\texttt{os09.pdf} (W09)},
                 \href{https://docOS.vlsm.org/Slides/os10.pdf}{\texttt{os10.pdf} (W10)}.
                 }
\item[$\square$] \textbf{Problems}\\
                 {\scriptsize% 
                 \href{https://rms46.vlsm.org/2/195.pdf}{\texttt{195.pdf} (W00)},
                 \href{https://rms46.vlsm.org/2/196.pdf}{\texttt{196.pdf} (W01)},
                 \href{https://rms46.vlsm.org/2/197.pdf}{\texttt{197.pdf} (W02)},
                 \href{https://rms46.vlsm.org/2/198.pdf}{\texttt{198.pdf} (W03)},
                 \href{https://rms46.vlsm.org/2/199.pdf}{\texttt{199.pdf} (W04)},
                 \href{https://rms46.vlsm.org/2/200.pdf}{\texttt{200.pdf} (W05)},\\
                 \href{https://rms46.vlsm.org/2/201.pdf}{\texttt{201.pdf} (W06)},
                 \href{https://rms46.vlsm.org/2/202.pdf}{\texttt{202.pdf} (W07)},
                 \href{https://rms46.vlsm.org/2/203.pdf}{\texttt{203.pdf} (W08)},
                 \href{https://rms46.vlsm.org/2/204.pdf}{\texttt{204.pdf} (W09)},
                 \href{https://rms46.vlsm.org/2/205.pdf}{\texttt{205.pdf} (W10)}.}
\item[$\square$] \textbf{LFS} --- \url{http://www.linuxfromscratch.org/lfs/view/stable/}
\item[$\square$] \textbf{OSP4DISS} --- \url{https://osp4diss.vlsm.org/}
\item[$\square$] \textbf{This is How Me Do It!} --- \url{https://doit.vlsm.org/}
\begin{itemize}
\item[$\square$] PS: "Me" rhymes better than "I", duh!
\end{itemize}
\end{itemize}
\end{itemize}
\end{frame}



% XXXXXXXXXXXXXXXXXXXXXXXXXXXXXXXXXXXXXXXXXXXXXXXXXXXXXXXXXXXXXXXXXXXXXXXXXX
% Throughout your presentation, if you choose to use \section{} and 
% \subsection{} commands, these will automatically be printed on 
% this slide as an overview of your presentation
\section{Agenda}
\begin{frame}{Outline}
  \frametitle{Agenda}
  \tableofcontents[sections={1-15}]
\end{frame}
\begin{frame}
   \frametitle{Agenda (2)}
   \tableofcontents[sections={16-}]
\end{frame}

% XXXXXXXXXXXXXXXXXXXXXXXXXXXXXXXXXXXXXXXXXXXXXXXXXXXXXXXXXXXXXXXXXXXXXXXXXX

\input{os06-BRP.tex}

% XXXXXXXXXXXXXXXXXXXXXXXXXXXXXXXXXXXXXXXXXXXXXXXXXXXXXXXXXXXXXXXXXXXXXXXXXX
\section{OSC10 (Silberschatz) Chapter 3: Processes and Chapter 4: Threads \& Concurrency}

\begin{frame}
\frametitle{OSC10 (Silberschatz) Chapter 3 and Chapter 4}
\begin{multicols}{2}
  \begin{itemize}
  \item Chapter 3: Processes 
  \begin{itemize}
  \item Process Concept
  \item Process Scheduling
  \item Operations on Processes
  \item Interprocess Communication
  \item IPC in Shared-Memory Systems
  \item IPC in Message-Passing Systems
  \item Examples of IPC Systems
  \item Communication in Client-Server Systems
  \end{itemize}
  \end{itemize}
  \vfill \null
\columnbreak
  \begin{itemize}
  \item Chapter 4: Threads \& Concurrency
  \begin{itemize}
  \item Overview
  \item Multicore Programming
  \item Multithreading Models
  \item Thread Libraries
  \item Implicit Threading
  \item Threading Issues
  \item Operating System Examples
  \end{itemize}
  \end{itemize}
  \vfill \null
\end{multicols}
\end{frame}

% XXXXXXXXXXXXXXXXXXXXXXXXXXXXXXXXXXXXXXXXXXXXXXXXXXXXXXXXXXXXXXXXXXXXXXXXXX
\section{Week 06}
\begin{frame}[fragile]
\frametitle{Week 06: Concurrency: Processes \& Threads}
\begin{itemize}
\item Reference: (OSC10-ch03 OSC10-ch04 demo-w06)
\item Process Concept
\begin{itemize}
\item Program (passive) $\leftrightarrow$ Process (active)
\item Process in Memory: $ \mid Stack \cdots Heap \mid Data \mid Text \mid $
\item Process State: $ \mid running \mid waiting \mid ready \mid $
\item Process Control Block (PCB)
\begin{itemize}
\item \texttt{/proc/}, Process State, Program Counter, Registers, Management Information.
\end{itemize}
\end{itemize}
\item Process Creation
\begin{itemize}
\item PID: Process Identifier (uniq)
\item The Parent Process forms a tree of Children Processes
\item \texttt{fork()}, new process system call (clone)
\item \texttt{execlp()}, replaces the clone with a new program.
\end{itemize}
\item Process Termination
\begin{itemize}
\item \texttt{wait()}, until the child process is terminated.
\end{itemize}
\item PCB (Context) Switch
\end{itemize}
\end{frame}

% XXXXXXXXXXXXXXXXXXXXXXXXXXXXXXXXXXXXXXXXXXXXXXXXXXXXXXXXXXXXXXXXXXXXXXXXXX
\section{Process Map}
\begin{frame}[fragile]
\frametitle{Process Map (1)}
\begin{figure}
\includegraphics[width=0.43\linewidth]{os06-memory}
\caption{A Process in (\textbf{logical}) Memory}
\end{figure}
\end{frame}

% 10 XXXXXXXXXXXXXXXXXXXXXXXXXXXXXXXXXXXXXXXXXXXXXXXXXXXXXXXXXXXXXXXXXXXXXXXX
\begin{frame}[fragile]
\frametitle{Process Map (2)}
% \begin{lstlisting}[basicstyle=\ttfamily\footnotesize] %  72
% \begin{lstlisting}[basicstyle=\ttfamily\small]        %  65
% \begin{lstlisting}[basicstyle=\ttfamily\large]        %  54
\begin{lstlisting}[basicstyle=\ttfamily\tiny]         % 108

/*
 * Copyright (C) 2021 Rahmat M. Samik-Ibrahim
 * START: Sat 03 Apr 2021 06:20:43 WIB
 */

#include <stdio.h>
#include <stdlib.h>

typedef void* AnyAddrPtr;
typedef char* ChrPtr;
typedef char  Chr;

Chr    aGlobalArray[16];
ChrPtr aGlobalCharacter1;
ChrPtr aGlobalCharacter2;
ChrPtr aGlobalCharacterPointer=aGlobalArray;

void printMyAddress (AnyAddrPtr address, ChrPtr message) {
    printf("[%p] %s\n", address, message);
}

int main(void) {
    ChrPtr aHeapCharacterPointer=malloc(16);
    Chr    aLocalArray[16];
    ChrPtr aLocalCharacterPointer=aGlobalArray;
    ChrPtr aLocalCharacter1;
    ChrPtr aLocalCharacter2;

    // ...
}

\end{lstlisting}
\end{frame}

% 10 XXXXXXXXXXXXXXXXXXXXXXXXXXXXXXXXXXXXXXXXXXXXXXXXXXXXXXXXXXXXXXXXXXXXXXXX
\begin{frame}[fragile]
\frametitle{Process Map (3)}
% \begin{lstlisting}[basicstyle=\ttfamily\footnotesize] %  72
% \begin{lstlisting}[basicstyle=\ttfamily\small]        %  65
% \begin{lstlisting}[basicstyle=\ttfamily\large]        %  54
% \begin{lstlisting}[basicstyle=\ttfamily\tiny]         % 108
\begin{lstlisting}[basicstyle=\ttfamily\footnotesize] %  72

[0x55559fcf9169] printMyAddress          (function, TEXT)
[0x55559fcf919c] main                    (function, TEXT)

[0x55559fcfc010] aGlobalCharacterPointer (global variable, DATA)
[0x55559fcfc030] aGlobalCharacter1       (global variable, DATA)
[0x55559fcfc040] aGlobalArray            (global variable, DATA)
[0x55559fcfc050] aGlobalCharacter2       (global variable, DATA)

[0x5555a0d192a0] aHeapCharacterPointer   (HEAP)

[0x7f9377bc9e10] printf                  (library, SHARED)
[0x7f9377c02260] malloc                  (library, SHARED)

[0x7fff8caa0010] aHeapCharacterPointer   (Pointer Variable, STACK)
[0x7ffd98ce1a10] aLocalCharacterPointer  (local variable, STACK)
[0x7ffd98ce1a18] aLocalCharacter1        (local variable, STACK)
[0x7ffd98ce1a20] aLocalCharacter2        (local variable, STACK)
[0x7ffd98ce1a30] aLocalArray             (local variable, STACK)

\end{lstlisting}
\end{frame}

% XXXXXXXXXXXXXXXXXXXXXXXXXXXXXXXXXXXXXXXXXXXXXXXXXXXXXXXXXXXXXXXXXXXXXXXXXX
\section{Process State}
\begin{frame}[fragile]
\frametitle{Process State}
\begin{figure}
\includegraphics[width=0.55\linewidth]{os06-state}
\caption{A Process State}
\end{figure}
\end{frame}

% XXXXXXXXXXXXXXXXXXXXXXXXXXXXXXXXXXXXXXXXXXXXXXXXXXXXXXXXXXXXXXXXXXXXXXXXXX
\begin{frame}
\frametitle{Process Scheduling}
\begin{itemize}
\item Scheduling Queue
\item Schedulers
\begin{itemize}
\item Long Term (non VM) vs Short Term (CPU)
\item (I/O vs CPU) Bound Processes
\end{itemize}
\item Context Switch
\item I/O Queue Scheduling
\item Android Systems
\begin{itemize}
\item Dalvik VM Performance Problem: Replaced with ART (Android Runtime).
\item Foreground Processes: with an User Interface (UI) for Videos, Images, Sounds, Texts, etc.
\item Background Processes: with a service with no UI and small memory footprint.
\end{itemize}
\end{itemize}
\end{frame}

% XXXXXXXXXXXXXXXXXXXXXXXXXXXXXXXXXXXXXXXXXXXXXXXXXXXXXXXXXXXXXXXXXXXXXXXXXX
\begin{frame}
\frametitle{Inter-Process Communication (IPC)}
\begin{itemize}
\item Independent vs Cooperating Processes.
\begin{itemize}
\item Cooperation: Information Sharing, Computational Speedup, Modularity, Convenience.
\end{itemize}
\item Shared Memory vs Message Passing.
\begin{itemize}
\item Message Passing: Direct vs Indirect Comunication
\end{itemize}
\item Client-Server Systems
\begin{itemize}
\item Sockets
\item RPC: Remote Procedure Calls
\item Pipes
\end{itemize}
\end{itemize}
\end{frame}

% XXXXXXXXXXXXXXXXXXXXXXXXXXXXXXXXXXXXXXXXXXXXXXXXXXXXXXXXXXXXXXXXXXXXXXXXXX
\begin{frame}
\frametitle{Threads}
\begin{itemize}
\item Single vs Multithreaded Process
\begin{itemize}
\item MultiT Benefits: Responsiveness, Resource Sharing, Economy, Scalability
\end{itemize}
\item Multicore Programming
\begin{itemize}
\item Concurrency vs. Parallelism
\end{itemize}
\item Multithreading Models (Kernel vs User Thread)
\begin{itemize}
\item Many to One
\item One to One
\item Many to Many
\item Multilevel Models
\end{itemize}
\item Threading Issues
\begin{itemize}
\item Parallelism on a multi-core system.
\end{itemize}
\item Pthreads
\end{itemize}
\end{frame}

% XXXXXXXXXXXXXXXXXXXXXXXXXXXXXXXXXXXXXXXXXXXXXXXXXXXXXXXXXXXXXXXXXXXXXXXXXX
\section{Makefile}
\begin{frame}[fragile]
\frametitle{Makefile}
\begin{lstlisting}[basicstyle=\ttfamily\tiny]
CC='gcc'
CFLAGS='-std=c99'
 
P00=00-show-pid
...
P15=15-uas171
P16=16-uas172

EXECS= \
	$(P00) \
	$(P01) \
	....
	$(P15) \
	$(P16) \

all:	$(EXECS)

$(P00): $(P00).c
	$(CC) $(P00).c -o $(P00)

$(P01): $(P01).c
	$(CC) $(P01).c -o $(P01)
...

$(P16): $(P16).c
	$(CC) $(P16).c -o $(P16)

clean:
	rm -f $(EXECS)
\end{lstlisting}
\end{frame}

% XXXXXXXXXXXXXXXXXXXXXXXXXXXXXXXXXXXXXXXXXXXXXXXXXXXXXXXXXXXXXXXXXXXXXXXXXX
\section{00-show-pid}
\begin{frame}[fragile]
\frametitle{00-show-pid}
% \begin{lstlisting}[basicstyle=\ttfamily\tiny]         % 108
% \begin{lstlisting}[basicstyle=\ttfamily\footnotesize] %  72
% \begin{lstlisting}[basicstyle=\ttfamily\small]        %  65
% \begin{lstlisting}[basicstyle=\ttfamily\large]        %  54
\begin{lstlisting}[basicstyle=\ttfamily\footnotesize]
/*
 * (c) 2016-2020 Rahmat M. Samik-Ibrahim
 * https://rahmatm.samik-ibrahim.vlsm.org/
 * This is free software.
 * REV07 Tue Mar 24 12:06:10 WIB 2020
 * START Mon Oct 24 09:42:05 WIB 2016
 */

#include <stdio.h>
#include <unistd.h>
#include <sys/types.h>

void main(void) {
   printf("  [[[ This is 00-show-pid: PID[%d] PPID[%d] ]]]\n",
             getpid(), getppid());
}

>>>>> $ ./00-show-pid

  [[[ This is 00-show-pid: PID[5777] PPID[1350] ]]]

\end{lstlisting}
\end{frame}

% XXXXXXXXXXXXXXXXXXXXXXXXXXXXXXXXXXXXXXXXXXXXXXXXXXXXXXXXXXXXXXXXXXXXXXXXXX
\section{01-fork}
\begin{frame}[fragile]
\frametitle{01-fork}
% \begin{lstlisting}[basicstyle=\ttfamily\tiny]         % 108
% \begin{lstlisting}[basicstyle=\ttfamily\footnotesize] %  72
% \begin{lstlisting}[basicstyle=\ttfamily\small]        %  65
% \begin{lstlisting}[basicstyle=\ttfamily\large]        %  54
\begin{lstlisting}[basicstyle=\ttfamily\tiny]
>>>>> $ cat 01-fork.c ; echo "======" ; ./01-fork 
/* (c) 2016-2017 Rahmat M. Samik-Ibrahim
 * https://rahmatm.samik-ibrahim.vlsm.org/
 * This is free software.
 */
#include <stdio.h>
#include <unistd.h>
#include <sys/types.h>
#include <sys/wait.h>

void main(void) {
   char *iAM="PARENT";
  
   printf("PID[%d] PPID[%d] (START:%s)\n", getpid(), getppid(), iAM);
   if (fork() > 0) {
      sleep(1);     /* LOOK THIS ************** */
      printf("PID[%d] PPID[%d] (IFF0:%s)\n", getpid(), getppid(), iAM);
   } else {
      iAM="CHILD";
      printf("PID[%d] PPID[%d] (ELSE:%s)\n", getpid(), getppid(), iAM);
   }
   printf("PID[%d] PPID[%d] (STOP:%s)\n", getpid(), getppid(), iAM);
}
======
PID[5784] PPID[1350] (START:PARENT)
PID[5785] PPID[5784] (ELSE:CHILD)
PID[5785] PPID[5784] (STOP:CHILD)
PID[5784] PPID[1350] (IFF0:PARENT)
PID[5784] PPID[1350] (STOP:PARENT)
>>>>> $ 

\end{lstlisting}
\end{frame}

% XXXXXXXXXXXXXXXXXXXXXXXXXXXXXXXXXXXXXXXXXXXXXXXXXXXXXXXXXXXXXXXXXXXXXXXXXX
\section{02-fork}
\begin{frame}[fragile]
\frametitle{02-fork}
% \begin{lstlisting}[basicstyle=\ttfamily\tiny]         % 108
% \begin{lstlisting}[basicstyle=\ttfamily\footnotesize] %  72
% \begin{lstlisting}[basicstyle=\ttfamily\small]        %  65
% \begin{lstlisting}[basicstyle=\ttfamily\large]        %  54
\begin{lstlisting}[basicstyle=\ttfamily\tiny]
>>>>> $ cat 02-fork.c ; echo "======" ; ./02-fork 
/* (c) 2016-2017 Rahmat M. Samik-Ibrahim
 * https://rahmatm.samik-ibrahim.vlsm.org/
 * This is free software.
 */
#include <stdio.h>
#include <unistd.h>
#include <sys/types.h>
#include <sys/wait.h>

void main(void) {
   char *iAM="PARENT";
  
   printf("PID[%d] PPID[%d] (START:%s)\n", getpid(), getppid(), iAM);
   if (fork() > 0) {
      printf("PID[%d] PPID[%d] (IFF0:%s)\n", getpid(), getppid(), iAM);
   } else {
      iAM="CHILD";
      printf("PID[%d] PPID[%d] (ELSE:%s)\n", getpid(), getppid(), iAM);
      sleep(1);     /* LOOK THIS ************** */
   }
   printf("PID[%d] PPID[%d] (STOP:%s)\n", getpid(), getppid(), iAM);
}
======
PID[5792] PPID[1350] (START:PARENT)
PID[5792] PPID[1350] (IFF0:PARENT)
PID[5792] PPID[1350] (STOP:PARENT)
PID[5793] PPID[5792] (ELSE:CHILD)
>>>>> $ PID[5793] PPID[1] (STOP:CHILD)
>>>>> $
\end{lstlisting}
\end{frame}

% XXXXXXXXXXXXXXXXXXXXXXXXXXXXXXXXXXXXXXXXXXXXXXXXXXXXXXXXXXXXXXXXXXXXXXXXXX
\section{03-fork}
\begin{frame}[fragile]
\frametitle{03-fork}
% \begin{lstlisting}[basicstyle=\ttfamily\tiny]         % 108
% \begin{lstlisting}[basicstyle=\ttfamily\footnotesize] %  72
% \begin{lstlisting}[basicstyle=\ttfamily\small]        %  65
% \begin{lstlisting}[basicstyle=\ttfamily\large]        %  54
\begin{lstlisting}[basicstyle=\ttfamily\tiny]
>>>>> $ cat 03-fork.c ; echo "======" ; ./03-fork 
/* (c) 2016-2017 Rahmat M. Samik-Ibrahim
 * https://rahmatm.samik-ibrahim.vlsm.org/
 * This is free software.
 */
#include <stdio.h>
#include <unistd.h>
#include <sys/types.h>
#include <sys/wait.h>

void main(void) {
   char *iAM="PARENT";
  
   printf("PID[%d] PPID[%d] (START:%s)\n", getpid(), getppid(), iAM);
   if (fork() > 0) {
      wait(NULL);     /* LOOK THIS ************** */
      printf("PID[%d] PPID[%d] (IFF0:%s)\n", getpid(), getppid(), iAM);
   } else {
      iAM="CHILD";
      printf("PID[%d] PPID[%d] (ELSE:%s)\n", getpid(), getppid(), iAM);
   }
   printf("PID[%d] PPID[%d] (STOP:%s)\n", getpid(), getppid(), iAM);
}
======
PID[5799] PPID[1350] (START:PARENT)
PID[5800] PPID[5799] (ELSE:CHILD)
PID[5800] PPID[5799] (STOP:CHILD)
PID[5799] PPID[1350] (IFF0:PARENT)
PID[5799] PPID[1350] (STOP:PARENT)
>>>>> $ 

\end{lstlisting}
\end{frame}

% XXXXXXXXXXXXXXXXXXXXXXXXXXXXXXXXXXXXXXXXXXXXXXXXXXXXXXXXXXXXXXXXXXXXXXXXXX
\section{01-fork vs 02-fork vs 03-fork}
\begin{frame}[fragile]
\frametitle{01-fork vs 02-fork vs 03-fork}
% \begin{lstlisting}[basicstyle=\ttfamily\tiny]         % 108
% \begin{lstlisting}[basicstyle=\ttfamily\footnotesize] %  72
% \begin{lstlisting}[basicstyle=\ttfamily\small]        %  65
% \begin{lstlisting}[basicstyle=\ttfamily\large]        %  54
\begin{lstlisting}[basicstyle=\ttfamily\footnotesize]
>>>>> $ ./01-fork 
PID[5803] PPID[1350] (START:PARENT)
PID[5804] PPID[5803] (ELSE:CHILD)
PID[5804] PPID[5803] (STOP:CHILD)
PID[5803] PPID[1350] (IFF0:PARENT)
PID[5803] PPID[1350] (STOP:PARENT)
>>>>> $ ./02-fork 
PID[5805] PPID[1350] (START:PARENT)
PID[5805] PPID[1350] (IFF0:PARENT)
PID[5805] PPID[1350] (STOP:PARENT)
PID[5806] PPID[5805] (ELSE:CHILD)
>>>>> $ PID[5806] PPID[1] (STOP:CHILD)
>>>>> $ ./03-fork 
PID[5807] PPID[1350] (START:PARENT)
PID[5808] PPID[5807] (ELSE:CHILD)
PID[5808] PPID[5807] (STOP:CHILD)
PID[5807] PPID[1350] (IFF0:PARENT)
PID[5807] PPID[1350] (STOP:PARENT)
>>>>> $ 
\end{lstlisting}
\end{frame}

% XXXXXXXXXXXXXXXXXXXXXXXXXXXXXXXXXXXXXXXXXXXXXXXXXXXXXXXXXXXXXXXXXXXXXXXXXX
\section{04-sleep}
\begin{frame}[fragile]
\frametitle{04-sleep}
% \begin{lstlisting}[basicstyle=\ttfamily\tiny]         % 108
% \begin{lstlisting}[basicstyle=\ttfamily\footnotesize] %  72
% \begin{lstlisting}[basicstyle=\ttfamily\small]        %  65
% \begin{lstlisting}[basicstyle=\ttfamily\large]        %  54
\begin{lstlisting}[basicstyle=\ttfamily\footnotesize]
#include <stdio.h>
#include <unistd.h>
void main(void) {
   int ii;
   printf("Sleeping 3s with fflush(): ");
   fflush(NULL);
   for (ii=0; ii < 3; ii++) {
      sleep(1); printf("x ");
      fflush(NULL);
   }
   printf("\nSleeping with no fflush(): ");
   for (ii=0; ii < 3; ii++) {
      sleep(1); printf("x ");
   }
   printf("\n");
}
Sleeping 3s with fflush(): x x x 
Sleeping with no fflush(): x x x 
\end{lstlisting}
\end{frame}

% XXXXXXXXXXXXXXXXXXXXXXXXXXXXXXXXXXXXXXXXXXXXXXXXXXXXXXXXXXXXXXXXXXXXXXXXXX
\section{05-fork}
\begin{frame}[fragile]
\frametitle{05a-fork}
% \begin{lstlisting}[basicstyle=\ttfamily\tiny]         % 108
% \begin{lstlisting}[basicstyle=\ttfamily\footnotesize] %  72
% \begin{lstlisting}[basicstyle=\ttfamily\small]        %  65
% \begin{lstlisting}[basicstyle=\ttfamily\large]        %  54
\begin{lstlisting}[basicstyle=\ttfamily\tiny]
#include <stdio.h>
#include <unistd.h>
#include <sys/types.h>
#include <sys/wait.h>

void main(void) {
   printf("Start:           PID[%d] PPID[%d]\n", getpid(), getppid());
   fflush(NULL);
   if (fork() == 0) {
      /* START BLOCK
      execlp("./00-fork", "00-fork", NULL);
         END   BLOCK */
      printf("Child:           ");
   } else {
      wait(NULL);
      printf("Parent:          ");
   }
   printf(        "PID[%d] PPID[%d]  <<< <<< <<<\n", getpid(), getppid());
}

no execlp ===================
Start:           PID[6040] PPID[1350]
Child:           PID[6041] PPID[6040]  <<< <<< <<<
Parent:          PID[6040] PPID[1350]  <<< <<< <<<

\end{lstlisting}
\end{frame}

% XXXXXXXXXXXXXXXXXXXXXXXXXXXXXXXXXXXXXXXXXXXXXXXXXXXXXXXXXXXXXXXXXXXXXXXXXX
\begin{frame}[fragile]
\frametitle{05b-fork}
% \begin{lstlisting}[basicstyle=\ttfamily\tiny]         % 108
% \begin{lstlisting}[basicstyle=\ttfamily\footnotesize] %  72
% \begin{lstlisting}[basicstyle=\ttfamily\small]        %  65
% \begin{lstlisting}[basicstyle=\ttfamily\large]        %  54
\begin{lstlisting}[basicstyle=\ttfamily\tiny]
#include <stdio.h>
#include <unistd.h>
#include <sys/types.h>
#include <sys/wait.h>

void main(void) {
   printf("Start:           PID[%d] PPID[%d]\n", getpid(), getppid());
   fflush(NULL);
   if (fork() == 0) {
      /* START BLOCK
         END   BLOCK */
      execlp("./00-fork", "00-fork", NULL);
      printf("Child:           ");
   } else {
      wait(NULL);
      printf("Parent:          ");
   }
   printf(        "PID[%d] PPID[%d]  <<< <<< <<<\n", getpid(), getppid());
}

execlp ======================
Start:           PID[6007] PPID[1350]
  [[[ This is 00-show-pid: PID[6008] PPID[6007] ]]]
Parent:          PID[6007] PPID[1350]  <<< <<< <<<

\end{lstlisting}
\end{frame}

% XXXXXXXXXXXXXXXXXXXXXXXXXXXXXXXXXXXXXXXXXXXXXXXXXXXXXXXXXXXXXXXXXXXXXXXXXX
\section{06-fork}
\begin{frame}[fragile]
\frametitle{06a-fork}
% \begin{lstlisting}[basicstyle=\ttfamily\tiny]         % 108
% \begin{lstlisting}[basicstyle=\ttfamily\footnotesize] %  72
% \begin{lstlisting}[basicstyle=\ttfamily\small]        %  65
% \begin{lstlisting}[basicstyle=\ttfamily\large]        %  54
\begin{lstlisting}[basicstyle=\ttfamily\tiny]
#include <sys/types.h>
#include <sys/wait.h>
#include <stdio.h>
#include <stdlib.h>
#include <unistd.h>
/*************************************************** main ** */
void main(void) {
   pid_t val1, val2, val3;
   val3 = val2 = val1 = 1000;
   printf("PID==%4d ==== ==== ==== ====\n", getpid());
/* ***** ***** ***** ***** START BLOCK *
   fflush(NULL);
   val1 = fork();
   wait(NULL);
   val2 = fork();
   wait(NULL);
   val3 = fork();
   wait(NULL);
   ***** ***** ***** ***** END** BLOCK */
   printf("VAL1=%4d VAL2=%4d VAL3=%4d\n", val1, val2, val3);
}
======
PID==[13965] ==== ======= ==== =======
VAL1=[01000] VAL2=[01000] VAL3=[01000]
\end{lstlisting}
\end{frame}

% XXXXXXXXXXXXXXXXXXXXXXXXXXXXXXXXXXXXXXXXXXXXXXXXXXXXXXXXXXXXXXXXXXXXXXXXXX
\begin{frame}[fragile]
\frametitle{06b-fork}
% \begin{lstlisting}[basicstyle=\ttfamily\tiny]         % 108
% \begin{lstlisting}[basicstyle=\ttfamily\footnotesize] %  72
% \begin{lstlisting}[basicstyle=\ttfamily\small]        %  65
% \begin{lstlisting}[basicstyle=\ttfamily\large]        %  54
\begin{lstlisting}[basicstyle=\ttfamily\tiny]
#include <sys/types.h>
#include <sys/wait.h>
#include <stdio.h>
#include <stdlib.h>
#include <unistd.h>
/*************************************************** main ** */
void main(void) {
   pid_t val1, val2, val3;
   val3 = val2 = val1 = 1000;
   printf("PID==%4d ==== ==== ==== ====\n", getpid());
   fflush(NULL);
   val1 = fork();
   wait(NULL);
/* ***** ***** ***** ***** START BLOCK *
   val2 = fork();
   wait(NULL);
   val3 = fork();
   wait(NULL);
   ***** ***** ***** ***** END** BLOCK */
   printf("VAL1=%4d VAL2=%4d VAL3=%4d\n", val1, val2, val3);
}
======
PID==[13969] ==== ======= ==== =======
VAL1=[00000] VAL2=[01000] VAL3=[01000]
VAL1=[13970] VAL2=[01000] VAL3=[01000]

\end{lstlisting}
\end{frame}

% XXXXXXXXXXXXXXXXXXXXXXXXXXXXXXXXXXXXXXXXXXXXXXXXXXXXXXXXXXXXXXXXXXXXXXXXXX
\begin{frame}[fragile]
\frametitle{06c-fork}
% \begin{lstlisting}[basicstyle=\ttfamily\tiny]         % 108
% \begin{lstlisting}[basicstyle=\ttfamily\footnotesize] %  72
% \begin{lstlisting}[basicstyle=\ttfamily\small]        %  65
% \begin{lstlisting}[basicstyle=\ttfamily\large]        %  54
\begin{lstlisting}[basicstyle=\ttfamily\tiny]
#include <sys/types.h>
#include <sys/wait.h>
#include <stdio.h>
#include <stdlib.h>
#include <unistd.h>
/*************************************************** main ** */
void main(void) {
   pid_t val1, val2, val3;
   val3 = val2 = val1 = 1000;
   printf("PID==%4d ==== ==== ==== ====\n", getpid());
   fflush(NULL);
   val1 = fork();
   wait(NULL);
   val2 = fork();
   wait(NULL);
/* ***** ***** ***** ***** START BLOCK *
   val3 = fork();
   wait(NULL);
   ***** ***** ***** ***** END** BLOCK */
   printf("VAL1=%4d VAL2=%4d VAL3=%4d\n", val1, val2, val3);
}
======
PID==[13971] ==== ======= ==== =======
VAL1=[00000] VAL2=[00000] VAL3=[01000]
VAL1=[00000] VAL2=[13973] VAL3=[01000]
VAL1=[13972] VAL2=[00000] VAL3=[01000]
VAL1=[13972] VAL2=[13974] VAL3=[01000]

\end{lstlisting}
\end{frame}

% XXXXXXXXXXXXXXXXXXXXXXXXXXXXXXXXXXXXXXXXXXXXXXXXXXXXXXXXXXXXXXXXXXXXXXXXXX
\begin{frame}[fragile]
\frametitle{06d-fork}
% \begin{lstlisting}[basicstyle=\ttfamily\tiny]         % 108
% \begin{lstlisting}[basicstyle=\ttfamily\footnotesize] %  72
% \begin{lstlisting}[basicstyle=\ttfamily\small]        %  65
% \begin{lstlisting}[basicstyle=\ttfamily\large]        %  54
\begin{lstlisting}[basicstyle=\ttfamily\tiny]
#include <sys/types.h>
#include <sys/wait.h>
#include <stdio.h>
#include <stdlib.h>
#include <unistd.h>
/*************************************************** main ** */
void main(void) {
   pid_t val1, val2, val3;
   val3 = val2 = val1 = 1000;
   printf("PID==%4d ==== ==== ==== ====\n", getpid());
   fflush(NULL);
   val1 = fork();
   wait(NULL);
   val2 = fork();
   wait(NULL);
   val3 = fork();
   wait(NULL);
/* ***** ***** ***** ***** START BLOCK * ***** ***** ***** ***** END** BLOCK */
   printf("VAL1=%4d VAL2=%4d VAL3=%4d\n", val1, val2, val3);
}
======
PID==[13976] ==== ======= ==== =======
VAL1=[00000] VAL2=[00000] VAL3=[00000]
VAL1=[00000] VAL2=[00000] VAL3=[13979]
VAL1=[00000] VAL2=[13978] VAL3=[00000]
VAL1=[00000] VAL2=[13978] VAL3=[13980]
VAL1=[13977] VAL2=[00000] VAL3=[00000]
VAL1=[13977] VAL2=[00000] VAL3=[13982]
VAL1=[13977] VAL2=[13981] VAL3=[00000]
VAL1=[13977] VAL2=[13981] VAL3=[13983]

\end{lstlisting}
\end{frame}

% XXXXXXXXXXXXXXXXXXXXXXXXXXXXXXXXXXXXXXXXXXXXXXXXXXXXXXXXXXXXXXXXXXXXXXXXXX
\section{07-execlp}
\begin{frame}[fragile]
\frametitle{07-execlp}
% \begin{lstlisting}[basicstyle=\ttfamily\tiny]         % 108
% \begin{lstlisting}[basicstyle=\ttfamily\footnotesize] %  72
% \begin{lstlisting}[basicstyle=\ttfamily\small]        %  65
% \begin{lstlisting}[basicstyle=\ttfamily\large]        %  54
\begin{lstlisting}[basicstyle=\ttfamily\tiny]
>>>>> $ cat 07-execlp.c 
/* (c) 2019-2020 Rahmat M. Samik-Ibrahim
 * https://rahmatm.samik-ibrahim.vlsm.org/
 * This is free software.
 * REV01 Tue Mar 24 16:29:50 WIB 2020
 * START Mon Dec  9 16:28:36 WIB 2019
 */
#include <stdio.h>
#include <sys/types.h>
#include <unistd.h>
void main(int argc, char* argv[]) {
   printf("START %11s PID[%d]\n", argv[0], getpid());
   if(argc == 1) {
      execlp(argv[0], "EXECLP", "WhatEver", NULL);
   } else {
      printf("ELSE  %11s PID[%d]\n", argv[1], getpid());
   }
   printf("END   %11s PID[%d]\n", argv[0], getpid());
}

$ ./07-execlp 
START ./07-execlp PID[14172]
START      EXECLP PID[14172]
ELSE     WhatEver PID[14172]
END        EXECLP PID[14172]
$ ./07-execlp XYZZYPLUGH
START ./07-execlp PID[14174]
ELSE   XYZZYPLUGH PID[14174]
END   ./07-execlp PID[14174]
$ 

\end{lstlisting}
\end{frame}

% XXXXXXXXXXXXXXXXXXXXXXXXXXXXXXXXXXXXXXXXXXXXXXXXXXXXXXXXXXXXXXXXXXXXXXXXXX
\section{08-fork}
\begin{frame}[fragile]
\frametitle{08-fork}
% \begin{lstlisting}[basicstyle=\ttfamily\tiny]         % 108
% \begin{lstlisting}[basicstyle=\ttfamily\footnotesize] %  72
% \begin{lstlisting}[basicstyle=\ttfamily\small]        %  65
% \begin{lstlisting}[basicstyle=\ttfamily\large]        %  54
\begin{lstlisting}[basicstyle=\ttfamily\tiny]
/* (c) 2005-2017 Rahmat M. Samik-Ibrahim https://rahmatm.samik-ibrahim.vlsm.org/ This is free software.
 * REV02 Thu Oct 26 12:27:30 WIB 2017
 * START 2005
 */
#include <sys/types.h>
#include <sys/wait.h>
#include <stdio.h>
#include <stdlib.h>
#include <unistd.h>
void main(void) {
   int ii=0;
   if (fork() == 0) ii++;
   wait(NULL);
   if (fork() == 0) ii++;
   wait(NULL);
   if (fork() == 0) ii++;
   wait(NULL);
   printf ("Result = %d \n",ii);
   exit(0);
}
======
Result = 3 
Result = 2 
Result = 2 
Result = 1 
Result = 2 
Result = 1 
Result = 1 
Result = 0 
>>>>> $ 

\end{lstlisting}
\end{frame}

% XXXXXXXXXXXXXXXXXXXXXXXXXXXXXXXXXXXXXXXXXXXXXXXXXXXXXXXXXXXXXXXXXXXXXXXXXX
\section{09-fork}
\begin{frame}[fragile]
\frametitle{09-fork}
% \begin{lstlisting}[basicstyle=\ttfamily\tiny]         % 108
% \begin{lstlisting}[basicstyle=\ttfamily\footnotesize] %  72
% \begin{lstlisting}[basicstyle=\ttfamily\small]        %  65
% \begin{lstlisting}[basicstyle=\ttfamily\large]        %  54
\begin{lstlisting}[basicstyle=\ttfamily\tiny]
/*
 * (c) 2015-2017 Rahmat M. Samik-Ibrahim https://rahmatm.samik-ibrahim.vlsm.org/
 * REV03 Mon Oct 30 11:04:10 WIB 2017
 * REV00 Mon Oct 24 10:43:00 WIB 2016
 * START 2015
 */
#include <stdio.h>
#include <sys/types.h>
#include <sys/wait.h>
#include <unistd.h>

void main(void) {
   int value;

   value=fork();
   wait(NULL);
   printf("I am PID[%4d] -- The fork() return value is: %4d)\n", getpid(), value);

   value=fork();
   wait(NULL);
   printf("I am PID[%4d] -- The fork() return value is: %4d)\n", getpid(), value);
}
======
I am PID[6225] -- The fork() return value is:    0)
I am PID[6226] -- The fork() return value is:    0)
I am PID[6225] -- The fork() return value is: 6226)
I am PID[6224] -- The fork() return value is: 6225)
I am PID[6227] -- The fork() return value is:    0)
I am PID[6224] -- The fork() return value is: 6227)
>>>>> $ 

\end{lstlisting}
\end{frame}

% XXXXXXXXXXXXXXXXXXXXXXXXXXXXXXXXXXXXXXXXXXXXXXXXXXXXXXXXXXXXXXXXXXXXXXXXXX
\section{10-fork}
\begin{frame}[fragile]
\frametitle{10-fork}
% \begin{lstlisting}[basicstyle=\ttfamily\tiny]         % 108
% \begin{lstlisting}[basicstyle=\ttfamily\footnotesize] %  72
% \begin{lstlisting}[basicstyle=\ttfamily\small]        %  65
% \begin{lstlisting}[basicstyle=\ttfamily\large]        %  54
\begin{lstlisting}[basicstyle=\ttfamily\tiny]
/* (c) 2016-2017 Rahmat M. Samik-Ibrahim https://rahmatm.samik-ibrahim.vlsm.org/ This is free software.
 * REV02 Mon Oct 30 20:25:44 WIB 2017
 */

#include <stdio.h>
#include <sys/types.h>
#include <sys/wait.h>
#include <unistd.h>

void procStatus(int level) {
   printf("L%d: PID[%d] (PPID[%d])\n", level, getpid(), getppid());
   fflush(NULL);
}

int addLevelAndFork(int level) {
   if (fork() == 0) level++;
   wait(NULL);
   return level;
}

void main(void) {
   int level = 0;
   procStatus(level);
   level = addLevelAndFork(level);
   procStatus(level);
}
======
L0: PID[7540] (PPID[1350])
L1: PID[7541] (PPID[7540])
L0: PID[7540] (PPID[1350])

\end{lstlisting}
\end{frame}

% XXXXXXXXXXXXXXXXXXXXXXXXXXXXXXXXXXXXXXXXXXXXXXXXXXXXXXXXXXXXXXXXXXXXXXXXXX
\section{11-fork}
\begin{frame}[fragile]
\frametitle{11-fork}
% \begin{lstlisting}[basicstyle=\ttfamily\tiny]         % 108
% \begin{lstlisting}[basicstyle=\ttfamily\footnotesize] %  72
% \begin{lstlisting}[basicstyle=\ttfamily\small]        %  65
% \begin{lstlisting}[basicstyle=\ttfamily\large]        %  54
\begin{lstlisting}[basicstyle=\ttfamily\tiny]
/* (c) 2016-2017 Rahmat M. Samik-Ibrahim https://rahmatm.samik-ibrahim.vlsm.org/ This is free software.
 * REV02 Mon Oct 30 20:27:24 WIB 2017
 * START Mon Oct 24 09:42:05 WIB 2016
 */

#define  LOOP   3
#include <stdio.h>
#include <sys/types.h>
#include <sys/wait.h>
#include <unistd.h>

void procStatus(int level) {
   printf("L%d: PID[%d] (PPID[%d])\n", level, getpid(), getppid());
   fflush(NULL);
}

int addLevelAndFork(int level) {
   if (fork() == 0) level++;
   wait(NULL);
   return level;
}

void main(void) {
   int ii, level = 0;
   procStatus(level);
   for (ii=0;ii<LOOP;ii++) {
      level = addLevelAndFork(level);
      procStatus(level);
   }
}
 
\end{lstlisting}
\end{frame}

% XXXXXXXXXXXXXXXXXXXXXXXXXXXXXXXXXXXXXXXXXXXXXXXXXXXXXXXXXXXXXXXXXXXXXXXXXX
\begin{frame}[fragile]
\frametitle{11-fork (2)}
% \begin{lstlisting}[basicstyle=\ttfamily\tiny]         % 108
% \begin{lstlisting}[basicstyle=\ttfamily\footnotesize] %  72
% \begin{lstlisting}[basicstyle=\ttfamily\small]        %  65
% \begin{lstlisting}[basicstyle=\ttfamily\large]        %  54
\begin{lstlisting}[basicstyle=\ttfamily\footnotesize]

L0: PID[7548] (PPID[1350])
L1: PID[7549] (PPID[7548])
L2: PID[7550] (PPID[7549])
L3: PID[7551] (PPID[7550])
L2: PID[7550] (PPID[7549])
L1: PID[7549] (PPID[7548])
L2: PID[7552] (PPID[7549])
L1: PID[7549] (PPID[7548])
L0: PID[7548] (PPID[1350])
L1: PID[7553] (PPID[7548])
L2: PID[7554] (PPID[7553])
L1: PID[7553] (PPID[7548])
L0: PID[7548] (PPID[1350])
L1: PID[7555] (PPID[7548])
L0: PID[7548] (PPID[1350])

\end{lstlisting}
\end{frame}

% XXXXXXXXXXXXXXXXXXXXXXXXXXXXXXXXXXXXXXXXXXXXXXXXXXXXXXXXXXXXXXXXXXXXXXXXXX
\section{12-fork}
\begin{frame}[fragile]
\frametitle{12-fork}
% \begin{lstlisting}[basicstyle=\ttfamily\tiny]         % 108
% \begin{lstlisting}[basicstyle=\ttfamily\footnotesize] %  72
% \begin{lstlisting}[basicstyle=\ttfamily\small]        %  65
% \begin{lstlisting}[basicstyle=\ttfamily\large]        %  54
\begin{lstlisting}[basicstyle=\ttfamily\tiny]
#include <stdio.h>
#include <unistd.h>
#include <sys/types.h>
#include <sys/wait.h>
void waitAndPrintPID(void) {
   wait(NULL);
   printf("PID: %d\n", getpid());
   fflush(NULL);
}
void main(int argc, char *argv[]) {
   int rc, status;
   waitAndPrintPID();
   rc = fork();
   waitAndPrintPID();
   if (rc == 0) {
      fork();
      waitAndPrintPID();
      execlp("./00-fork", "00-fork", NULL);
   }
   waitAndPrintPID();
}
======
PID: 7614
PID: 7615
PID: 7616
  [[[ This is 00-fork: PID[7616] PPID[7615] ]]]
PID: 7615
  [[[ This is 00-fork: PID[7615] PPID[7614] ]]]
PID: 7614
PID: 7614
>>>>> $ 

\end{lstlisting}
\end{frame}

% XXXXXXXXXXXXXXXXXXXXXXXXXXXXXXXXXXXXXXXXXXXXXXXXXXXXXXXXXXXXXXXXXXXXXXXXXX
\section{13-uas161}
\begin{frame}[fragile]
\frametitle{13-uas161}
% \begin{lstlisting}[basicstyle=\ttfamily\tiny]         % 108
% \begin{lstlisting}[basicstyle=\ttfamily\footnotesize] %  72
% \begin{lstlisting}[basicstyle=\ttfamily\small]        %  65
% \begin{lstlisting}[basicstyle=\ttfamily\large]        %  54
\begin{lstlisting}[basicstyle=\ttfamily\tiny]
/*
 * Copyright (C) 2015-2020 Rahmat M. Samik-Ibrahim http://rahmatm.samik-ibrahim.vlsm.org/ This program is free script/software.
 * REV10 Tue Mar 24 16:38:29 WIB 2020
 * START Xxx Xxx XX XX:XX:XX XXX XXXX
 */

#include <stdio.h>
#include <unistd.h>
#include <sys/types.h>
#include <sys/wait.h>

void main(void) {
   pid_t  pid1, pid2, pid3;

   pid1 = pid2 = pid3 = getpid();
   printf(" 2016   2015   Lainnya\n=====================\n");
   printf("[%5.5d][%5.5d][%5.5d]\n", pid1, pid2, pid3);
   fork(); 
   pid1 = getpid();
   wait(NULL);
   pid2 = getpid();
   if(!fork()) {
     pid2 = getpid();
     fork();
   }
   pid3 = getpid();
   wait(NULL);
   printf("[%5.5d][%5.5d][%5.5d]\n", pid1, pid2, pid3);
}

/*
# INFO: UTS 2016-1 (midterm)
 */

\end{lstlisting}
\end{frame}

% XXXXXXXXXXXXXXXXXXXXXXXXXXXXXXXXXXXXXXXXXXXXXXXXXXXXXXXXXXXXXXXXXXXXXXXXXX
\begin{frame}[fragile]
\frametitle{13-uas161}
% \begin{lstlisting}[basicstyle=\ttfamily\tiny]         % 108
% \begin{lstlisting}[basicstyle=\ttfamily\footnotesize] %  72
% \begin{lstlisting}[basicstyle=\ttfamily\small]        %  65
\begin{lstlisting}[basicstyle=\ttfamily\large]        %  54

/*
# INFO: UTS 2016-1 (midterm)
 */

$ ./13-uas161 
 2016   2015   Lainnya
=====================
[14492][14492][14492]
[14493][14494][14495]
[14493][14494][14494]
[14493][14493][14493]
[14492][14496][14497]
[14492][14496][14496]
[14492][14492][14492]

\end{lstlisting}
\end{frame}

% XXXXXXXXXXXXXXXXXXXXXXXXXXXXXXXXXXXXXXXXXXXXXXXXXXXXXXXXXXXXXXXXXXXXXXXXXX
\section{14-uas162}
\begin{frame}[fragile]
\frametitle{14-uas162}
\begin{lstlisting}[basicstyle=\ttfamily\tiny]         % 108
% \begin{lstlisting}[basicstyle=\ttfamily\footnotesize] %  72
% \begin{lstlisting}[basicstyle=\ttfamily\small]        %  65
% \begin{lstlisting}[basicstyle=\ttfamily\large]        %  54

/* Copyright (C) 2016-2020 Rahmat M. Samik-Ibrahim http://rahmatm.samik-ibrahim.vlsm.org/
 * This program is free script/software.
 * REV08 Tue Mar 24 16:40:28 WIB 2020
 * START Sun Dec 04 00:00:00 WIB 2016
 * wait()     =  suspends until its child terminates. 
 * fflush()   =  flushes the user-space buffers.
 * getppid()  =  get parent PID
 * ASSUME pid >= 1000 && pid > ppid **
 */

#include <stdio.h>
#include <sys/types.h>
#include <unistd.h>
#include <sys/wait.h>
#define  NN 2

void main(void) {
   int ii, rPID, rPPID, id1000=getpid();
   for (ii=1; ii<=NN; ii++) {
      fork();
      wait(NULL);
      rPID = getpid()-id1000+1000; /* "relative" */
      rPPID=getppid()-id1000+1000; /* "relative" */
      if (rPPID < 1000 || rPPID > rPID) rPPID=999;
      printf("Loop [%d] - rPID[%d] - rPPID[%4d]\n", ii, rPID, rPPID);
      fflush(NULL);
   }
}

\end{lstlisting}
\end{frame}

% XXXXXXXXXXXXXXXXXXXXXXXXXXXXXXXXXXXXXXXXXXXXXXXXXXXXXXXXXXXXXXXXXXXXXXXXXX
\begin{frame}[fragile]
\frametitle{14-uas162}
% \begin{lstlisting}[basicstyle=\ttfamily\tiny]         % 108
% \begin{lstlisting}[basicstyle=\ttfamily\footnotesize] %  72
% \begin{lstlisting}[basicstyle=\ttfamily\small]        %  65
\begin{lstlisting}[basicstyle=\ttfamily\large]        %  54

/*
# INFO: UTS 2016-2 (midterm)
 */

$ ./14-uas162 
Loop [1] - rPID[1001] - rPPID[1000]
Loop [2] - rPID[1002] - rPPID[1001]
Loop [2] - rPID[1001] - rPPID[1000]
Loop [1] - rPID[1000] - rPPID[ 999]
Loop [2] - rPID[1003] - rPPID[1000]
Loop [2] - rPID[1000] - rPPID[ 999]

\end{lstlisting}
\end{frame}

% XXXXXXXXXXXXXXXXXXXXXXXXXXXXXXXXXXXXXXXXXXXXXXXXXXXXXXXXXXXXXXXXXXXXXXXXXX
\section{15-uas171}
\begin{frame}[fragile]
\frametitle{15-uas171}
\begin{lstlisting}[basicstyle=\ttfamily\tiny]         % 108
% \begin{lstlisting}[basicstyle=\ttfamily\footnotesize] %  72
% \begin{lstlisting}[basicstyle=\ttfamily\small]        %  65
% \begin{lstlisting}[basicstyle=\ttfamily\large]        %  54

/* Copyright (C) 2005-2020 Rahmat M. Samik-Ibrahim http://rahmatm.samik-ibrahim.vlsm.org/ This program is free script/software. 
 * REV00 Wed May  3 17:07:09 WIB 2017
 * START 2005
 * fflush(NULL): flushes all open output streams
 * fork():       creates  a new process by cloning
 * getpid():     get PID (Process ID)
 * wait(NULL):   wait until the child is terminated
 */

#include <stdio.h>
#include <unistd.h>
#include <sys/types.h>
#include <sys/wait.h>
#include <stdlib.h>

void main(void) {
   int firstPID = (int) getpid();
   int   RelPID;

   fork();
   wait(NULL);
   fork();
   wait(NULL);
   fork();
   wait(NULL);

   RelPID=(int)getpid()-firstPID+1000;
   printf("RelPID: %d\n", RelPID);
   fflush(NULL);
}

\end{lstlisting}
\end{frame}

% XXXXXXXXXXXXXXXXXXXXXXXXXXXXXXXXXXXXXXXXXXXXXXXXXXXXXXXXXXXXXXXXXXXXXXXXXX
\begin{frame}[fragile]
\frametitle{15-uas171}
% \begin{lstlisting}[basicstyle=\ttfamily\tiny]         % 108
% \begin{lstlisting}[basicstyle=\ttfamily\footnotesize] %  72
% \begin{lstlisting}[basicstyle=\ttfamily\small]        %  65
\begin{lstlisting}[basicstyle=\ttfamily\large]        %  54

/*
# INFO: UTS 2017-1 (midterm)
 */

$ ./15-uas171 
RelPID: 1003
RelPID: 1002
RelPID: 1004
RelPID: 1001
RelPID: 1006
RelPID: 1005
RelPID: 1007
RelPID: 1000
$ 


\end{lstlisting}
\end{frame}

% XXXXXXXXXXXXXXXXXXXXXXXXXXXXXXXXXXXXXXXXXXXXXXXXXXXXXXXXXXXXXXXXXXXXXXXXXX
\section{16-uas172}
\begin{frame}[fragile]
\frametitle{16-uas172}
\begin{lstlisting}[basicstyle=\ttfamily\tiny]         % 108
% \begin{lstlisting}[basicstyle=\ttfamily\footnotesize] %  72
% \begin{lstlisting}[basicstyle=\ttfamily\small]        %  65
% \begin{lstlisting}[basicstyle=\ttfamily\large]        %  54

/*
 * (c) 2017-2020 Rahmat M. Samik-Ibrahim
 * http://rahmatm.samik-ibrahim.vlsm.org/
 * This is free software.
 * REV03 Tue Mar 24 16:42:16 WIB 2020
 * REV02 Mon Dec 11 17:46:01 WIB 2017
 * START Sun Dec  3 18:00:08 WIB 2017
 */

#include <stdio.h>
#include <unistd.h>
#include <sys/types.h>
#include <sys/wait.h>

#define LOOP   3
#define OFFSET 1000

void main(void) {
   int basePID = getpid() - OFFSET;

   for (int ii=0; ii < LOOP; ii++) {
      if(!fork()) {
         printf("PID[%d]-PPID[%d]\n", 
                 getpid()  - basePID, 
                 getppid() - basePID);
         fflush(NULL);
      }
      wait(NULL);
   }
}

\end{lstlisting}
\end{frame}

% XXXXXXXXXXXXXXXXXXXXXXXXXXXXXXXXXXXXXXXXXXXXXXXXXXXXXXXXXXXXXXXXXXXXXXXXXX
\begin{frame}[fragile]
\frametitle{16-uas172}
% \begin{lstlisting}[basicstyle=\ttfamily\tiny]         % 108
% \begin{lstlisting}[basicstyle=\ttfamily\footnotesize] %  72
% \begin{lstlisting}[basicstyle=\ttfamily\small]        %  65
\begin{lstlisting}[basicstyle=\ttfamily\large]        %  54

/*
# INFO: UTS 2017-2 (midterm)
 */

$ ./16-uas172 
PID[1001]-PPID[1000]
PID[1002]-PPID[1001]
PID[1003]-PPID[1002]
PID[1004]-PPID[1001]
PID[1005]-PPID[1000]
PID[1006]-PPID[1005]
PID[1007]-PPID[1000]
$ 

\end{lstlisting}
\end{frame}

% XXXXXXXXXXXXXXXXXXXXXXXXXXXXXXXXXXXXXXXXXXXXXXXXXXXXXXXXXXXXXXXXXXXXXXXXXX
\section{Assignment Week06}
\begin{frame}[fragile]
\frametitle{mylib.h (1)}
% \begin{lstlisting}[basicstyle=\ttfamily\tiny]         % 108
% \begin{lstlisting}[basicstyle=\ttfamily\footnotesize] %  72
% \begin{lstlisting}[basicstyle=\ttfamily\small]        %  65
% \begin{lstlisting}[basicstyle=\ttfamily\large]        %  54
\begin{lstlisting}[basicstyle=\ttfamily\tiny]         % 108

/*
 * Copyright (C) 2021-2021 Rahmat M. Samik-Ibrahim
 * http://rahmatm.samik-ibrahim.vlsm.org/
 * This program is free script/software. This program is distributed in the
 * hope that it will be useful, but WITHOUT ANY WARRANTY; without even the
 * implied warranty of MERCHANTABILITY or FITNESS FOR A PARTICULAR PURPOSE.
 * REV08: Sun 04 Apr 07:28:09 WIB 2021
 * REV07: Sun 04 Apr 00:11:43 WIB 2021
 * REV06: Sat 03 Apr 11:00:46 WIB 2021
 * REV05: Tue 30 Mar 14:55:36 WIB 2021
 * REV04: Tue 30 Mar 10:35:13 WIB 2021
 * START: Mon 22 Mar 16:14:36 WIB 2021
 *
# INFO: mylib.h
 */

#define TOKEN          "OS212W06"
#define WEEKFILE       "WEEK06-MEMORY-SHARE.bin"
#define FORKS          4

#define BUFFERSIZE     256
#define SSIZE          4
#define STAMPSIZE      11
#define CHMOD          0666
#define CMDSHA1 "echo %s | sha1sum | cut -c1-4 | tr '[:lower:]' '[:upper:]' "
#define MYFLAGS        O_CREAT|O_RDWR
#define MYPROTECTION   PROT_READ|PROT_WRITE
#define MYVISIBILITY   MAP_SHARED

\end{lstlisting}
\end{frame}

% 10 XXXXXXXXXXXXXXXXXXXXXXXXXXXXXXXXXXXXXXXXXXXXXXXXXXXXXXXXXXXXXXXXXXXXXXXX
\begin{frame}[fragile]
\frametitle{mylib.h (2)}
% \begin{lstlisting}[basicstyle=\ttfamily\tiny]         % 108
% \begin{lstlisting}[basicstyle=\ttfamily\footnotesize] %  72
% \begin{lstlisting}[basicstyle=\ttfamily\small]        %  65
% \begin{lstlisting}[basicstyle=\ttfamily\large]        %  54
% \end{lstlisting}
% \end{frame}
\begin{lstlisting}[basicstyle=\ttfamily\tiny]         % 108

#include <fcntl.h>
#include <stdio.h>
#include <stdlib.h>
#include <string.h>
#include <sys/mman.h>
#include <sys/stat.h>
#include <sys/types.h>
#include <sys/wait.h>
#include <time.h>
#include <unistd.h>

typedef           char  Chr;
typedef           char* ChrPtr;
typedef  unsigned char  uChr;
typedef  unsigned char* uChrPtr;
typedef  struct {
    Chr counter;
    Chr blank;
    Chr stamp[FORKS][BUFFERSIZE];
    Chr end;
    Chr zero;
} memStruct;
typedef memStruct* memStructPtr;

void            chktoken         (uChrPtr result, uChrPtr token);
memStructPtr createShareMemory(memStructPtr mymap, int memorySize, ChrPtr memoryName);
void            getTimeStamp     (uChrPtr timeStamp);
void            mySHA1           (uChrPtr output, uChrPtr input, int length);
void            pickToken        (uChrPtr result, uChrPtr token);
void            verifyToken      (uChrPtr result, uChrPtr token, uChrPtr input);

\end{lstlisting}
\end{frame}

% 10 XXXXXXXXXXXXXXXXXXXXXXXXXXXXXXXXXXXXXXXXXXXXXXXXXXXXXXXXXXXXXXXXXXXXXXXX
\begin{frame}[fragile]
\frametitle{mylib.c (1)}
% \begin{lstlisting}[basicstyle=\ttfamily\tiny]         % 108
% \begin{lstlisting}[basicstyle=\ttfamily\footnotesize] %  72
% \begin{lstlisting}[basicstyle=\ttfamily\small]        %  65
% \begin{lstlisting}[basicstyle=\ttfamily\large]        %  54
\begin{lstlisting}[basicstyle=\ttfamily\tiny]         % 108

/*
 * Copyright (C) 2021-2021 Rahmat M. Samik-Ibrahim
 * http://rahmatm.samik-ibrahim.vlsm.org/
 * This program is free script/software. This program is distributed in the 
 * hope that it will be useful, but WITHOUT ANY WARRANTY; without even the 
 * implied warranty of MERCHANTABILITY or FITNESS FOR A PARTICULAR PURPOSE.
 * REV08: Sun 04 Apr 07:25:24 WIB 2021
 * REV07: Sun 04 Apr 00:11:43 WIB 2021
 * REV04: Tue 30 Mar 10:35:13 WIB 2021
 * START: Mon 22 Mar 16:14:36 WIB 2021
 *
# INFO: mylib.c
 */

#include "mylib.h"

void mySHA1(uChrPtr output, uChrPtr input, int length) {
    Chr     cmd[BUFFERSIZE];
    sprintf(cmd, CMDSHA1, input);
    FILE* filePtr = popen(cmd, "r");
    fgets(output, length+1, filePtr);
    output[length]=0;
    pclose(filePtr);
}

void getTimeStamp(uChrPtr timeStamp) {
    time_t tt    =  time(NULL);
    struct tm tm = *localtime(&tt);
    sprintf(timeStamp, "%2.2d%2.2d", tm.tm_min, tm.tm_sec);
}

\end{lstlisting}
\end{frame}

% 10 XXXXXXXXXXXXXXXXXXXXXXXXXXXXXXXXXXXXXXXXXXXXXXXXXXXXXXXXXXXXXXXXXXXXXXXX
\begin{frame}[fragile]
\frametitle{mylib.c (2)}
% \begin{lstlisting}[basicstyle=\ttfamily\tiny]         % 108
% \begin{lstlisting}[basicstyle=\ttfamily\footnotesize] %  72
% \begin{lstlisting}[basicstyle=\ttfamily\small]        %  65
% \begin{lstlisting}[basicstyle=\ttfamily\large]        %  54
\begin{lstlisting}[basicstyle=\ttfamily\tiny]         % 108

void chktoken (uChrPtr result, uChrPtr token) {
    uChr timeStamp[] = "MMSS";
    getTimeStamp(timeStamp);
    uChr input [BUFFERSIZE];
    strcpy(input,timeStamp);
    uChrPtr user=getenv("USER");
    strcat(input,user);
    strcat(input,token);
    uChr   output [BUFFERSIZE];
    mySHA1(output, input, SSIZE);
    sprintf(result, "%s %s-%s", user, timeStamp, output);
}

void verifyToken(uChrPtr result, uChrPtr token, uChrPtr input) {
    uChr    tmpStr1[BUFFERSIZE];
    uChr    tmpStr2[BUFFERSIZE];
    strcpy(tmpStr1,input);
    uChrPtr user=strtok(tmpStr1," ");
    uChrPtr timeStamp=strtok(NULL,"-");
    strcpy(tmpStr2,timeStamp);
    strcat(tmpStr2,user);
    strcat(tmpStr2,token);
    uChr   output [BUFFERSIZE];
    mySHA1(output, tmpStr2, SSIZE);
    uChrPtr tmpStr3=strtok(NULL,"-");
    if (strcmp(output, tmpStr3) == 0 ) sprintf(result, "Verified");
    else sprintf(result, "Error");
}

\end{lstlisting}
\end{frame}

% 10 XXXXXXXXXXXXXXXXXXXXXXXXXXXXXXXXXXXXXXXXXXXXXXXXXXXXXXXXXXXXXXXXXXXXXXXX
\begin{frame}[fragile]
\frametitle{mylib.c (3)}
% \begin{lstlisting}[basicstyle=\ttfamily\footnotesize] %  72
% \begin{lstlisting}[basicstyle=\ttfamily\small]        %  65
% \begin{lstlisting}[basicstyle=\ttfamily\large]        %  54
\begin{lstlisting}[basicstyle=\ttfamily\tiny]         % 108

void pickToken (uChrPtr result, uChrPtr token) {
    uChr   tmpStr1[BUFFERSIZE];
    strcpy(tmpStr1,token);
    strtok(tmpStr1," ");
    strcpy(result, strtok(NULL," "));
}

memStructPtr createShareMemory(memStructPtr mymap, int memorySize, ChrPtr memoryName) {
    int      fd    = open(memoryName, MYFLAGS, CHMOD);
    fchmod   (fd, CHMOD);
    ftruncate(fd, memorySize);
    mymap = mmap(NULL, memorySize, MYPROTECTION, MYVISIBILITY, fd, 0);
    close(fd);
    return mymap;
}

\end{lstlisting}
\end{frame}

% 10 XXXXXXXXXXXXXXXXXXXXXXXXXXXXXXXXXXXXXXXXXXXXXXXXXXXXXXXXXXXXXXXXXXXXXXXX
\begin{frame}[fragile]
\frametitle{chktoken.c (1)}
% \begin{lstlisting}[basicstyle=\ttfamily\footnotesize] %  72
% \begin{lstlisting}[basicstyle=\ttfamily\small]        %  65
% \begin{lstlisting}[basicstyle=\ttfamily\large]        %  54
\begin{lstlisting}[basicstyle=\ttfamily\tiny]         % 108

/*
 * Copyright (C) 2021-2021 Rahmat M. Samik-Ibrahim
 * http://rahmatm.samik-ibrahim.vlsm.org/
 * This program is free script/software. This program is distributed in the 
 * hope that it will be useful, but WITHOUT ANY WARRANTY; without even the 
 * implied warranty of MERCHANTABILITY or FITNESS FOR A PARTICULAR PURPOSE.
# INFO: chktoken TOKEN
 * REV02 Sun 04 Apr 2021 08:05:57 WIB
 * REV01 Sun 04 Apr 2021 00:11:27 WIB
 * START Sat 03 Apr 2021 15:10:28 WIB
 */

#include "mylib.h"

int main(int argc, ChrPtr argv[]) {
    if (argc < 2) return -1;
    uChr     result1[BUFFERSIZE];
    chktoken (result1, argv[1]);
    printf("%s\n", result1);
}

\end{lstlisting}
\end{frame}

% 10 XXXXXXXXXXXXXXXXXXXXXXXXXXXXXXXXXXXXXXXXXXXXXXXXXXXXXXXXXXXXXXXXXXXXXXXX
\begin{frame}[fragile]
\frametitle{verifyToken.c (1)}
% \begin{lstlisting}[basicstyle=\ttfamily\footnotesize] %  72
% \begin{lstlisting}[basicstyle=\ttfamily\small]        %  65
% \begin{lstlisting}[basicstyle=\ttfamily\large]        %  54
\begin{lstlisting}[basicstyle=\ttfamily\tiny]         % 108

/*
 * Copyright (C) 2021-2021 Rahmat M. Samik-Ibrahim
 * http://rahmatm.samik-ibrahim.vlsm.org/
 * This program is free script/software. This program is distributed in the 
 * hope that it will be useful, but WITHOUT ANY WARRANTY; without even the 
 * implied warranty of MERCHANTABILITY or FITNESS FOR A PARTICULAR PURPOSE.
# INFO: TOP (Table of Processes)
 * REV02 Sun 04 Apr 2021 07:24:22 WIB
 * REV01 Sun 04 Apr 2021 00:11:27 WIB
 * START Sat 03 Apr 2021 15:10:28 WIB
 */

#include "mylib.h"

int main(int argc, ChrPtr argv[]) {
    if (argc < 4) return -1;
    uChr     result1[BUFFERSIZE];
    uChr     result2[BUFFERSIZE];
    strcpy(result1,argv[2]);
    strcat(result1," ");
    strcat(result1,argv[3]);
    verifyToken(result2, argv[1], result1);
    printf("%s\n", result2);
}

\end{lstlisting}
\end{frame}

% 10 XXXXXXXXXXXXXXXXXXXXXXXXXXXXXXXXXXXXXXXXXXXXXXXXXXXXXXXXXXXXXXXXXXXXXXXX
\begin{frame}[fragile]
\frametitle{myfork.c (1)}
% \begin{lstlisting}[basicstyle=\ttfamily\footnotesize] %  72
% \begin{lstlisting}[basicstyle=\ttfamily\small]        %  65
% \begin{lstlisting}[basicstyle=\ttfamily\large]        %  54
\begin{lstlisting}[basicstyle=\ttfamily\tiny]         % 108

/*
 * Copyright (C) 2021-2021 Rahmat M. Samik-Ibrahim
 * http://rahmatm.samik-ibrahim.vlsm.org/
 * This program is free script/software. This program is distributed in the 
 * hope that it will be useful, but WITHOUT ANY WARRANTY; without even the 
 * implied warranty of MERCHANTABILITY or FITNESS FOR A PARTICULAR PURPOSE.
# INFO: myfork00
 * START Sun 04 Apr 2021 11:00:01 AM WIB
 */

#include "mylib.h"

int main(void) {
    memStructPtr  mymap = createShareMemory(mymap, sizeof(memStruct), WEEKFILE);
    mymap->counter='1';
    int counter=mymap->counter-'1';
    mymap->blank=' ';
    mymap->end='\n';
    mymap->zero=0;
    uChr      result1[BUFFERSIZE];
    chktoken (result1, TOKEN);

\end{lstlisting}
\end{frame}

% 10 XXXXXXXXXXXXXXXXXXXXXXXXXXXXXXXXXXXXXXXXXXXXXXXXXXXXXXXXXXXXXXXXXXXXXXXX
\begin{frame}[fragile]
\frametitle{myfork.c (2)}
% \begin{lstlisting}[basicstyle=\ttfamily\footnotesize] %  72
% \begin{lstlisting}[basicstyle=\ttfamily\small]        %  65
% \begin{lstlisting}[basicstyle=\ttfamily\large]        %  54
\begin{lstlisting}[basicstyle=\ttfamily\tiny]         % 108

    if (fork() == 0) {
        sleep(1);
        mymap->counter++;
        counter=mymap->counter-'1';
        chktoken (result1, TOKEN);
        if (fork() == 0) {
            sleep(1);
            mymap->counter++;
            counter=mymap->counter-'1';
            chktoken (result1, TOKEN);
            if (fork() == 0) {
                sleep(1);
                mymap->counter++;
                counter=mymap->counter-'1';
                chktoken (result1, TOKEN);
            }
            wait(NULL);
        }
        wait(NULL);
    }
    wait(NULL);
    strcpy(mymap->stamp[counter], result1);
    strcat(mymap->stamp[counter], " ");
    printf("PID[%d][%s]-[%d]\n", getpid(), result1, counter);
    wait(NULL);
}

\end{lstlisting}
\end{frame}

% 10 XXXXXXXXXXXXXXXXXXXXXXXXXXXXXXXXXXXXXXXXXXXXXXXXXXXXXXXXXXXXXXXXXXXXXXXX
\begin{frame}[fragile]
\frametitle{mytest.c (1)}
% \begin{lstlisting}[basicstyle=\ttfamily\footnotesize] %  72
% \begin{lstlisting}[basicstyle=\ttfamily\small]        %  65
% \begin{lstlisting}[basicstyle=\ttfamily\large]        %  54
\begin{lstlisting}[basicstyle=\ttfamily\tiny]         % 108

/*
 * Copyright (C) 2021-2021 Rahmat M. Samik-Ibrahim
 * http://rahmatm.samik-ibrahim.vlsm.org/
 * This program is free script/software. This program is distributed in the 
 * hope that it will be useful, but WITHOUT ANY WARRANTY; without even the 
 * implied warranty of MERCHANTABILITY or FITNESS FOR A PARTICULAR PURPOSE.
# INFO: TOP (Table of Processes)
 * REV01 Sun 04 Apr 2021 00:11:59 WIB
 * START Sat 03 Apr 2021 15:10:28 WIB
 */

#include "mylib.h"

int main(void) {
    uChr     result1[BUFFERSIZE];
    chktoken (result1, TOKEN);
    printf("%s\n", result1);
    uChr     result2[BUFFERSIZE];
    verifyToken (result2, TOKEN, result1);
    printf("%s: %s\n", TOKEN, result2);
    verifyToken (result2, "DODOLGRT", "rms46 0605-0687");
    printf("%s: %s\n", "DODOLGRT", result2);
    verifyToken (result2, "DODOLGRT", "rms46 1820-2A46");
    printf("%s: %s\n", "DODOLGRT", result2);
    sleep (1);
    chktoken (result1, TOKEN);
    printf("%s\n", result1);
    pickToken(result2, result1);
    printf("%s\n", result2);
}

\end{lstlisting}
\end{frame}

% 10 XXXXXXXXXXXXXXXXXXXXXXXXXXXXXXXXXXXXXXXXXXXXXXXXXXXXXXXXXXXXXXXXXXXXXXXX
\begin{frame}[fragile]
\frametitle{mytest.sh (1)}
% \begin{lstlisting}[basicstyle=\ttfamily\footnotesize] %  72
% \begin{lstlisting}[basicstyle=\ttfamily\small]        %  65
% \begin{lstlisting}[basicstyle=\ttfamily\large]        %  54
\begin{lstlisting}[basicstyle=\ttfamily\tiny]         % 108

#!/bin/bash
# REV01 Mon  5 Apr 17:08:58 WIB 2021
# START Sun  4 Apr 17:22:46 WIB 2021
# Copyright (C) 2021-2021 Rahmat M. Samik-Ibrahim http://rahmatm.samik-ibrahim.vlsm.org/
# This program is free script/software. This program is distributed in the 
# hope that it will be useful, but WITHOUT ANY WARRANTY; without even the 
# implied warranty of MERCHANTABILITY or FITNESS FOR A PARTICULAR PURPOSE.
# INFO: myfork00

CLEANFILE="WEEK06-MEMORY-SHARE.txt"
WEEKFILE="WEEK06-MEMORY-SHARE.bin"

TOKEN="OS212W06"

[ -f $CLEANFILE ] || { echo "No $CLEANFILE"; exit; }

sleep 1
echo "ZCZC $(date)"
echo -n "ZCZC $(./chktoken $TOKEN): "
echo "$(./verifyToken $TOKEN $(./chktoken $TOKEN))"
echo "ZCZC BINSIZE $(wc -c < $WEEKFILE)"
echo "ZCZC TXTSIZE $(wc -c < $CLEANFILE)"
FIRST=""
for II in $(cat $CLEANFILE) ; do
    [ ! -z "${II##*[!0-9]*}" ] && continue
    [ -z "$FIRST" ] && { FIRST=$II ; continue; }
    echo -n "ZCZC $FIRST $II: "
    echo "$(./verifyToken $TOKEN $FIRST $II)"
    FIRST=""
done

\end{lstlisting}
\end{frame}

% 10 XXXXXXXXXXXXXXXXXXXXXXXXXXXXXXXXXXXXXXXXXXXXXXXXXXXXXXXXXXXXXXXXXXXXXXXX
\begin{frame}[fragile]
\frametitle{Makefile (1)}
% \begin{lstlisting}[basicstyle=\ttfamily\footnotesize] %  72
% \begin{lstlisting}[basicstyle=\ttfamily\small]        %  65
% \begin{lstlisting}[basicstyle=\ttfamily\large]        %  54
\begin{lstlisting}[basicstyle=\ttfamily\tiny]         % 108

# REV03 Mon 05 Apr 17:55:47 WIB 2021
# REV02 Sun 04 Apr 07:22:23 WIB 2021
# REV01 Sat 03 Apr 10:51:58 WIB 2021
# START Tue 13 Sep 11:44:18 WIB 2016

# INFO: With this "Makefile", just run:
# INFO:                     make

CC            = gcc
CPP           = cpp
CFLAGS        = -std=gnu18
LDFLAGS       = 
CPPFLAGS      =
DEPFLAGS      = -MM -MT $(@:.d=.o) 
OUTPUT_OPTION = -o $@
COMPILE       = $(CC) $(DEPFLAGS) $(CFLAGS) $(CPPFLAGS) -c
SRCS          = $(wildcard *.c)
OBJ           = $(SRCS:.c=.o)
DEP           = $(OBJ:.o=.d)
PROGS         = $(SRCS:.c=  )

P01=mytest
P02=chktoken
P03=verifyToken
P04=myfork

L99=mylib
WEEKFILE=WEEK06-MEMORY-SHARE.bin
CLEANFILE=WEEK06-MEMORY-SHARE.txt

\end{lstlisting}
\end{frame}

% 10 XXXXXXXXXXXXXXXXXXXXXXXXXXXXXXXXXXXXXXXXXXXXXXXXXXXXXXXXXXXXXXXXXXXXXXXX
\begin{frame}[fragile]
\frametitle{Makefile (2)}
% \begin{lstlisting}[basicstyle=\ttfamily\footnotesize] %  72
% \begin{lstlisting}[basicstyle=\ttfamily\small]        %  65
% \begin{lstlisting}[basicstyle=\ttfamily\large]        %  54
\begin{lstlisting}[basicstyle=\ttfamily\tiny]         % 108

EXECS= \
   $(P01) \
   $(P02) \
   $(P03) \
   $(P04) \


all:  $(EXECS)

test: $(EXECS)
	./$(P04) 
	cat $(WEEKFILE) | wc -c > $(CLEANFILE)
	cat $(WEEKFILE) | tr -dc '[:alnum:]\n -_' >> $(CLEANFILE)
	bash mytest.sh

$(EXECS): %: %.c $(DEPS) $(L99).c
	$(CC) $(CFLAGS)  $(L99).c $< -o $@ $(LDFLAGS)

clean:
	rm -f $(EXECS)
	rm -f *.map 
	rm -f $(WEEKFILE) $(CLEANFILE)

.phony: clean all test

\end{lstlisting}
\begin{table}
\end{table}
\end{frame}

% XXXXXXXXXXXXXXXXXXXXXXXXXXXXXXXXXXXXXXXXXXXXXXXXXXXXXXXXXXXXXXXXXXXXXXXXXX
\end{document}

