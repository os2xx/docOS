%%%%%%%%%%%%%%%%%%%%%%%%%%%%%%%%%%%%%%%%%%%%%%%%%%%%%%%%%%%%%%%%%%%%%%%%
% Beamer Presentation - LaTeX - Template Version 1.0 (10/11/12)
% This template has been downloaded from: http://www.LaTeXTemplates.com
% License: % CC BY-NC-SA 3.0 (http://creativecommons.org/)
% Modified by Rahmat M. Samik-Ibrahim

% REV417: Sun 28 Jan 2024 16:00
% REV395: Sun 11 Sep 2022 22:00
% REV386: Thu 28 Jul 2022 12:00
% REV371: Mon 28 Feb 2022 10:00
% REV363: Mon 20 Dec 2021 15:00
% STARTX: Wed 14 Sep 2016 10:00
%%%%%%%%%%%%%%%%%%%%%%%%%%%%%%%%%%%%%%%%%%%%%%%%%%%%%%%%%%%%%%%%%%%%%%%%%

% PACKAGES AND THEMES 
\documentclass[aspectratio=169, xcolor=table, notheorems, hyperref={pdfpagelabels=false}]{beamer}
%%%%%%%%%%%%%%%%%%%%%%%%%%%%%%%%%%%%%%%%%%%%%%%%%%%%%%%%%%%%%%%%%%%%%%%%
% Beamer Presentation - LaTeX - Template Version 1.0 (10/11/12)
% This template has been downloaded from: http://www.LaTeXTemplates.com
% License: % CC BY-NC-SA 3.0 (http://creativecommons.org/)
% Modified by Bin Kadal, Sdn Bhd.
% REV023: Tue 30 Jan 2024 16:00
% REV006: Mon 22 Jan 2018 19:00
% STARTX: Thu 25 Aug 2016 14:00
%%%%%%%%%%%%%%%%%%%%%%%%%%%%%%%%%%%%%%%%%%%%%%%%%%%%%%%%%%%%%%%%%%%%%%%%%

%% ZCZC NNNN
\newtheorem{example}{Example}

%%%%%%%%%%%%%%%%%%%%%%%%%%%%%%%%%%%%%%%%%%%%%%%%%%%%%%%%%%%%%%%%%%%%%%%%%

\let\Tiny=\tiny
\mode<presentation> {
% The Beamer class comes with a number of default slide themes
% which change the colors and layouts of slides. Below this is a list
% of all the themes, uncomment each in turn to see what they look like.
%\usetheme{Boadilla}
\usetheme{Madrid}
% ZCZC %%%%%%%%%%%%%%%%%%%%%%%%%%%%%%%%%%%%%%%%%%%%%%%%%%%%%%%%%%%%%%%%%%
% \usetheme{default} \usetheme{AnnArbor} \usetheme{Antibes} \usetheme{Bergen}
% \usetheme{Berkeley} \usetheme{Berlin} \usetheme{CambridgeUS} 
% \usetheme{Copenhagen} \usetheme{Darmstadt} \usetheme{Dresden}
% \usetheme{Frankfurt} \usetheme{Goettingen} \usetheme{Hannover}
% \usetheme{Ilmenau} \usetheme{JuanLesPins} \usetheme{Luebeck}
% \usetheme{Malmoe} \usetheme{Marburg} \usetheme{Montpellier}
% \usetheme{PaloAlto} \usetheme{Pittsburgh} \usetheme{Rochester}
% \usetheme{Singapore} \usetheme{Szeged} \usetheme{Warsaw}
% NNNN %%%%%%%%%%%%%%%%%%%%%%%%%%%%%%%%%%%%%%%%%%%%%%%%%%%%%%%%%%%%%%%%%%
% As well as themes, the Beamer class has a number of color themes
% for any slide theme. Uncomment each of these in turn to see how it
% changes the colors of your current slide theme.
%\usecolortheme{orchid}
%\usecolortheme{rose}
%\usecolortheme{seagull}
%\usecolortheme{seahorse}
\usecolortheme{whale}
% ZCZC %%%%%%%%%%%%%%%%%%%%%%%%%%%%%%%%%%%%%%%%%%%%%%%%%%%%%%%%%%%%%%%%%%
%\usecolortheme{albatross} \usecolortheme{beaver} \usecolortheme{beetle}
%\usecolortheme{crane} \usecolortheme{dolphin} \usecolortheme{dove}
%\usecolortheme{fly} \usecolortheme{lily} \usecolortheme{wolverine}
% NNNN %%%%%%%%%%%%%%%%%%%%%%%%%%%%%%%%%%%%%%%%%%%%%%%%%%%%%%%%%%%%%%%%%%
% To remove the footer line in all slides uncomment this line
%\setbeamertemplate{footline} 
% To replace the footer line in all slides uncomment this line
%\setbeamertemplate{footline}[page number] 
% To remove the navigation symbols from the bottom uncomment this line
\setbeamertemplate{navigation symbols}{} 
}

\usepackage{array}       % ZCZC
\usepackage{amssymb}     % ZCZC
\usepackage{bold-extra}  % ZCZC
\usepackage{booktabs}    % Allows \toprule, \midrule and \bottomrule in tables
\usepackage{caption}
\usepackage[T1]{fontenc} % ZCZC << >>
\usepackage{graphicx}    % Allows including images
\usepackage{listings}    % listing
\usepackage{lmodern}     % ZCZC
\usepackage{perpage}     % reset footnote per page
\usepackage{geometry}    % ZCZC
\usepackage{adjustbox}   % ZCZC
\usepackage{multirow}    % ZCZC

% \definecolor{links}{HTML}{2A1B81}
\definecolor{links}{HTML}{0011FF}
\hypersetup{colorlinks,linkcolor=,urlcolor=links}

% \usepackage{xcolor}
% \usepackage[colorlinks = true,
%             linkcolor = blue,
%             urlcolor  = blue,
%             citecolor = blue,
%             anchorcolor = blue]{hyperref}

\captionsetup[table]{name=Tabel}
\makeatletter
\def\input@path{{src/}}
\makeatother
\graphicspath{{src/}}      % src directory
\MakePerPage{footnote}     % reset page

% NNNN %%%%%%%%%%%%%%%%%%%%%%%%%%%%%%%%%%%%%%%%%%%%%%%%%%%%%%%%%%%%%%%%%%

%% % XXXXXXXXXXXXXXXXXXXXXXXXXXXXXXXXXXXXXXXXXXXXXXXXXXXXXXXXXXXXXXXXXXXXXXXXXX
%% % The short title appears at the bottom of every slide, 
%% % the full title is only on the title page
%% \title[Judul Pendek]{Judul Panjang dan Lengkap} 
%% \author{Cecak bin Kadal}
%% \institute[UILA]
%% {
%% University of Indonesia at Lenteng Agung \\ 
%% \medskip
%% \textit{cecak@binKadal.com}
%% }
%% \date{REV00 24-Aug-2016}
%% % \date{\today}
%% 

%% % XXXXXXXXXXXXXXXXXXXXXXXXXXXXXXXXXXXXXXXXXXXXXXXXXXXXXXXXXXXXXXXXXXXXXXXXXX
%% \begin{document}
%% \section{Judul}
%% \begin{frame}
%% \titlepage
%% \end{frame}
%% 
%% % XXXXXXXXXXXXXXXXXXXXXXXXXXXXXXXXXXXXXXXXXXXXXXXXXXXXXXXXXXXXXXXXXXXXXXXXXX
%% \section{Agenda}
%% \begin{frame}
%% \frametitle{Agenda}
%% % Throughout your presentation, if you choose to use \section{} and 
%% % \subsection{} commands, these will automatically be printed on 
%% % this slide as an overview of your presentation
%% \tableofcontents 
%% \end{frame}
%% 
%% % XXXXXXXXXXXXXXXXXXXXXXXXXXXXXXXXXXXXXXXXXXXXXXXXXXXXXXXXXXXXXXXXXXXXXXXXXX
%% \section{UUD dan Pancasila}
%% \subsection{UUD}
%% \begin{frame}
%% \frametitle{Pembukaan}
%% Bahwa sesungguhnya kemerdekaan itu ialah hak segala bangsa dan oleh 
%% sebab itu, maka penjajahan diatas dunia harus dihapuskan karena 
%% tidak sesuai dengan perikemanusiaan dan perikeadilan.
%% \\~\\
%% Atas berkat rahmat Allah Yang Maha Kuasa dan dengan didorongkan oleh 
%% keinginan luhur, supaya berkehidupan kebangsaan yang bebas, maka 
%% rakyat Indonesia menyatakan dengan ini kemerdekaannya.
%% \end{frame}
%% 
%% % XXXXXXXXXXXXXXXXXXXXXXXXXXXXXXXXXXXXXXXXXXXXXXXXXXXXXXXXXXXXXXXXXXXXXXXXXX
%% \begin{frame}
%% \frametitle{Alenia Ketiga}
%% Kemudian daripada itu untuk membentuk suatu pemerintah negara Indonesia 
%% yang melindungi segenap bangsa Indonesia dan seluruh tumpah darah Indonesia 
%% dan untuk memajukan kesejahteraan umum, mencerdaskan kehidupan bangsa, dan 
%% ikut melaksanakan ketertiban dunia yang berdasarkan kemerdekaan, perdamaian 
%% abadi dan keadilan sosial, maka disusunlah kemerdekaan kebangsaan Indonesia 
%% itu dalam suatu Undang-Undang Dasar negara Indonesia, yang terbentuk dalam 
%% suatu susunan negara Republik Indonesia yang berkedaulatan rakyat dengan 
%% berdasar kepada:
%% \begin{itemize}
%% \item Ketuhanan Yang Maha Esa,
%% \item kemanusiaan yang adil dan beradab,
%% \item persatuan Indonesia,
%% \item dan kerakyatan yang dipimpin oleh hikmat kebijaksanaan 
%%       dalam permusyawaratan/ perwakilan,
%% \item serta dengan mewujudkan suatu keadilan sosial bagi seluruh rakyat 
%%       Indonesia.
%% \end{itemize}
%% \end{frame}
%% 
%% % XXXXXXXXXXXXXXXXXXXXXXXXXXXXXXXXXXXXXXXXXXXXXXXXXXXXXXXXXXXXXXXXXXXXXXXXXX
%% \subsection{Pancasila}
%% \begin{frame}
%% \frametitle{Tujuh Kunci Pokok}
%% \begin{block}{Pertama - Kedua - Ketiga}
%% Indonesia ialah negara berdasarkan hukum.
%% Sistem konstitusional.
%% Kekuasaan negara tertinggi di tangan MPR.
%% \end{block}
%% 
%% \begin{block}{Keempat - Kelima}
%% Presiden adalah penyelenggara pemerintahan tertinggi di bawah MPR.
%% Adanya pengawasan DPR.
%% \end{block}
%% 
%% \begin{block}{Keenam}
%% Menteri negara adalah pembantu presiden dan tidak bertanggung jawab 
%% kepada DPR.
%% \end{block}
%% 
%% \begin{block}{Ketujuh}
%% Kekuasaan kepala negara tidak tak tebatas.
%% \end{block}
%% 
%% \end{frame}
%% 
%% % XXXXXXXXXXXXXXXXXXXXXXXXXXXXXXXXXXXXXXXXXXXXXXXXXXXXXXXXXXXXXXXXXXXXXXXXXX
%% \section{Rupa-rupa}
%% \subsection{Kolom}
%% \begin{frame}
%% \frametitle{Kolom}
%% % The "c" option specifies centered vertical alignment 
%% % while the "t" option is used for top vertical alignment
%% \begin{columns}[c] 
%% % Left column and width
%% \column{.45\textwidth} 
%% \textbf{Heading}
%% \begin{enumerate}
%% \item Satu-satu
%% \item Dua-dua
%% \item Tiga-tiga
%% \item Satu-dua-tiga
%% \end{enumerate}
%% 
%% % Right column and width
%% \column{.5\textwidth}
%% Satu-satu~\dots{} aku sayang ibu!
%% Dua-dua~\ldots{} juga sayang ayah!
%% Tiga-tiga~\ldots{} sayang adik kakak!
%% Satu-dua-tiga~\ldots{} sayang semuanya!
%% 
%% \end{columns}
%% \end{frame}
%% 
%% % XXXXXXXXXXXXXXXXXXXXXXXXXXXXXXXXXXXXXXXXXXXXXXXXXXXXXXXXXXXXXXXXXXXXXXXXXX
%% \subsection{Tabel}
%% \begin{frame}
%% \frametitle{Tabel}
%% \begin{table}
%% \begin{tabular}{l l l}
%% \toprule
%% \textbf{Nama} & \textbf{NPM} & \textbf{Tanggal Lahir}\\
%% \midrule
%% Cecak bin Kadal & 1234567890 & 1 Jan 2015 \\
%% Aneh bin Ajaib  & 0987654321 & 31 Des 2014 \\
%% \bottomrule
%% \end{tabular}
%% \caption{Keterangan Tabel}
%% \end{table}
%% \end{frame}
%% 
%% % XXXXXXXXXXXXXXXXXXXXXXXXXXXXXXXXXXXXXXXXXXXXXXXXXXXXXXXXXXXXXXXXXXXXXXXXXX
%% \subsection{Teori}
%% \begin{frame}
%% \frametitle{Teori}
%% \begin{theorem}[Teori Satu Batu]
%% $E = mc^2$
%% \end{theorem}
%% \end{frame}
%% 
%% % XXXXXXXXXXXXXXXXXXXXXXXXXXXXXXXXXXXXXXXXXXXXXXXXXXXXXXXXXXXXXXXXXXXXXXXXXX
%% \subsection{Verbatim}
%% % Need to use the fragile option when verbatim is used in the slide
%% \begin{frame}[fragile] 
%% \frametitle{Verbatim}
%% \begin{example}[Teori Satu Batu]
%% \begin{verbatim}
%% \begin{theorem}[Teori Satu Batu]
%% $E = mc^2$
%% \end{theorem}
%% \end{verbatim}
%% \end{example}
%% \end{frame}
%% 
%% % XXXXXXXXXXXXXXXXXXXXXXXXXXXXXXXXXXXXXXXXXXXXXXXXXXXXXXXXXXXXXXXXXXXXXXXXXX
%% \subsection{Gambar}
%% \begin{frame}
%% \frametitle{Gambar}
%% \begin{figure}
%% \includegraphics[width=0.5\linewidth]{2}
%% \caption{Ini Gambar JPG}
%% \end{figure}
%% \end{frame}
%% 
%% % XXXXXXXXXXXXXXXXXXXXXXXXXXXXXXXXXXXXXXXXXXXXXXXXXXXXXXXXXXXXXXXXXXXXXXXXXX
%% \subsection{Rujukan}
%% % Need to use the fragile option when verbatim is used in the slide
%% \begin{frame}[fragile] 
%% \frametitle{Rujukan dan Kutipan}
%% Contoh penggunaan \verb|\cite| ketika mengutip\cite{p1}.
%% Perhatian: Beamer tidak mengerti \verb|\BibTeX|~\ldots
%% \footnotesize{
%%   \begin{thebibliography}{99} 
%%   \bibitem[Smith, 2012]{p1} John Smith (2012)
%%      \newblock Katak dalam Tempurung
%%      \newblock \emph{Jurnal Kelapa dan Amfibi} 12(3), 45 -- 678.
%%   \end{thebibliography}
%% }
%% \end{frame}
%% 
%% % XXXXXXXXXXXXXXXXXXXXXXXXXXXXXXXXXXXXXXXXXXXXXXXXXXXXXXXXXXXXXXXXXXXXXXXXXX
%% \subsection{Selesai}
%% \begin{frame}
%% \Huge{\centerline{Selesai}}
%% \end{frame}
%% 
%% % XXXXXXXXXXXXXXXXXXXXXXXXXXXXXXXXXXXXXXXXXXXXXXXXXXXXXXXXXXXXXXXXXXXXXXXXXX
%% \end{document}

\newcommand{\revision}{%
REV424: Tue 03 Sep 2024 20:00
}
% w! tmptmp
% REV424: Tue 03 Sep 2024 20:00
% REV419: Wed 24 Jul 2024 17:00
% REV409: Tue 08 Aug 2023 12:00
% REV399: Fri 03 Feb 2023 20:00
% REV339: Sat 04 Sep 2021 12:00
% STARTS: Wed 24 Aug 2016 19:00
%%%%%%%%%%%%%%%%%%%%%%%%%%%%%%%%%%%%%
\newcommand{\kopikopi}{\textcopyright{}2016-2024 CBKadal + VauLSMorg}



% XXXXXXXXXXXXXXXXXXXXXXXXXXXXXXXXXXXXXXXXXXXXXXXXXXXXXXXXXXXXXXXXXXXXXXXXXX
% The short title appears at the bottom of every slide, 
% the full title is only on the title page
% \date{\today}
\title[\kopikopi]
{CSGE602055 Operating Systems \\ 
CSF2600505 Sistem Operasi \\
Week 05:
Virtual Memory}
\author{C. BinKadal}
\institute[SdnBhd]
{
Sendirian Berhad\\
\medskip
\url{https://docOS.vlsm.org/Slides/os05.pdf}
\\ \texttt{Always check for the latest revision!}
}
\date{\revision}

% XXXXXXXXXXXXXXXXXXXXXXXXXXXXXXXXXXXXXXXXXXXXXXXXXXXXXXXXXXXXXXXXXXXXXXXXXX
\begin{document}

\lstset{
basicstyle=\ttfamily\tiny, % \tiny \small \footnotesize 
breakatwhitespace=true,
language=C,
columns=fullflexible,
keepspaces=true,
breaklines=true,
tabsize=3, 
showstringspaces=false,
extendedchars=true}

\section{Start}
\begin{frame}
\titlepage
\end{frame}

% XXXXXXXXXXXXXXXXXXXXXXXXXXXXXXXXXXXXXXXXXXXXXXXXXXXXXXXXXXXXXXXXXXXXXXXXXX

%%%%%%%%%%%%%%%%%%%%%%%%%%%%%%%%%%%%%%%%%%%%%%%%%%%%%%%%%%%%%%%%%%%%%%%%%
% REV418: Tue 30 Jan 2024 22:00
% REV406: Sat 05 Aug 2023 14:00
% REV399: Thu 02 Feb 2023 00:00
% REV369: Mon 14 Feb 2022 09:00
% REV328: Sat 14 Aug 2021 06:00
% STARTX: Wed 14 Sep 2016 10:00
%%%%%%%%%%%%%%%%%%%%%%%%%%%%%%%%%%%%%%%%%%%%%%%%%%%%%%%%%%%%%%%%%%%%%%%%%

\begin{frame}[fragile]
\section{OS241 Schedule}
\frametitle{OS241\footnote{%
) This information will be on \textbf{EVERY} page two (2) of this course material.}): 
Operating Systems Schedule 2023 - 2}

\vspace{5pt}

\scalebox{0.99}{%
\begin{tabular}{|c|c|l|l|}
\hline
\textbf{Week} & 
\textbf{Topic}\footnote{%
) For schedule, see \url{https://os.vlsm.org/\#idx02}}) & \textbf{OSC10}\footnote{%
    ) Silberschatz et. al.: \textbf{Operating System Concepts}, $10^{th}$ Edition, 2018.}) \\
\hline
Week 00  & Overview (1), Assignment of Week 00           & Ch. 1, 2      \\
Week 01  & Overview (2), Virtualization \& Scripting     & Ch. 1, 2, 18. \\
Week 02  & Security, Protection, Privacy, \& C-language. & Ch. 16, 17.   \\
Week 03  & File System \& FUSE  & Ch. 13, 14, 15.                        \\
Week 04  & Addressing, Shared Lib, \& Pointer & Ch. 9. \\
Week 05  & Virtual Memory & Ch. 10. \\
\hline
Week 06  & Concurrency: Processes \& Threads & Ch. 3, 4. \\
Week 07  & Synchronization \& Deadlock & Ch. 6, 7, 8. \\
Week 08  & Scheduling + W06/W07 & Ch. 5. \\
Week 09  & Storage, Firmware, Bootloader, \& Systemd & Ch. 11. \\
Week 10  & I/O \& Programming & Ch. 12. \\%
% MidTerm  & 00 XXX 2020 (XX:XX-XX:XX) & MidTerm (UTS) & \cellcolor{red!44} TBA! \\
% Reserved & 00 XXX - 00 XXX 2020 & Q \& A & \\
% Final    & 00 XXX 2020 XX:XX & First Part Final  (UAS tahap I)  & \cellcolor{red!44} This schedule is   \\
% Extra    & NA & No Extra assignment & \cellcolor{red!44} subject to change. \\
\hline
\end{tabular}
}
\end{frame}

\begin{frame}[fragile]
\frametitle{\textbf{STARTING POINT} --- 
{
\definecolor{links}{HTML}{FDEE00}
\hypersetup{colorlinks,linkcolor=,urlcolor=links}
\url{https://os.vlsm.org/}
}
}
\begin{itemize}
\item[$\square$] \textbf{Text Book} ---
                 Any recent/decent OS book. Eg. (\textbf{OSC10}) Silberschatz et. al.: 
                 \textbf{Operating System Concepts}, $10^{th}$ Edition, 2018.
                 (See \url{https://codex.cs.yale.edu/avi/os-book/OS10/}).
\item[$\square$] \textbf{Resources ({\footnotesize \url{https://os.vlsm.org/\#idx03}})}
\begin{itemize}
\item[$\square$] \href{https://scele.cs.ui.ac.id/course/view.php?id=3743}{\textbf{SCELE}} ---
\url{https://scele.cs.ui.ac.id/course/view.php?id=3743}.\\
The enrollment key is \textbf{XXX}.
\item[$\square$] \textbf{Download Slides and Demos from GitHub.com} --- (\url{https://github.com/os2xx/docOS/})\\
                 {\scriptsize%
                 \href{https://docOS.vlsm.org/Slides/os00.pdf}{\texttt{os00.pdf} (W00)},
                 \href{https://docOS.vlsm.org/Slides/os01.pdf}{\texttt{os01.pdf} (W01)},
                 \href{https://docOS.vlsm.org/Slides/os02.pdf}{\texttt{os02.pdf} (W02)},
                 \href{https://docOS.vlsm.org/Slides/os03.pdf}{\texttt{os03.pdf} (W03)},
                 \href{https://docOS.vlsm.org/Slides/os04.pdf}{\texttt{os04.pdf} (W04)},
                 \href{https://docOS.vlsm.org/Slides/os05.pdf}{\texttt{os05.pdf} (W05)},\\
                 \href{https://docOS.vlsm.org/Slides/os06.pdf}{\texttt{os06.pdf} (W06)},
                 \href{https://docOS.vlsm.org/Slides/os07.pdf}{\texttt{os07.pdf} (W07)},
                 \href{https://docOS.vlsm.org/Slides/os08.pdf}{\texttt{os08.pdf} (W08)},
                 \href{https://docOS.vlsm.org/Slides/os09.pdf}{\texttt{os09.pdf} (W09)},
                 \href{https://docOS.vlsm.org/Slides/os10.pdf}{\texttt{os10.pdf} (W10)}.
                 }
\item[$\square$] \textbf{Problems}\\
                 {\scriptsize% 
                 \href{https://rms46.vlsm.org/2/195.pdf}{\texttt{195.pdf} (W00)},
                 \href{https://rms46.vlsm.org/2/196.pdf}{\texttt{196.pdf} (W01)},
                 \href{https://rms46.vlsm.org/2/197.pdf}{\texttt{197.pdf} (W02)},
                 \href{https://rms46.vlsm.org/2/198.pdf}{\texttt{198.pdf} (W03)},
                 \href{https://rms46.vlsm.org/2/199.pdf}{\texttt{199.pdf} (W04)},
                 \href{https://rms46.vlsm.org/2/200.pdf}{\texttt{200.pdf} (W05)},\\
                 \href{https://rms46.vlsm.org/2/201.pdf}{\texttt{201.pdf} (W06)},
                 \href{https://rms46.vlsm.org/2/202.pdf}{\texttt{202.pdf} (W07)},
                 \href{https://rms46.vlsm.org/2/203.pdf}{\texttt{203.pdf} (W08)},
                 \href{https://rms46.vlsm.org/2/204.pdf}{\texttt{204.pdf} (W09)},
                 \href{https://rms46.vlsm.org/2/205.pdf}{\texttt{205.pdf} (W10)}.}
\item[$\square$] \textbf{LFS} --- \url{http://www.linuxfromscratch.org/lfs/view/stable/}
\item[$\square$] \textbf{OSP4DISS} --- \url{https://osp4diss.vlsm.org/}
\item[$\square$] \textbf{This is How Me Do It!} --- \url{https://doit.vlsm.org/}
\begin{itemize}
\item[$\square$] PS: "Me" rhymes better than "I", duh!
\end{itemize}
\end{itemize}
\end{itemize}
\end{frame}



% XXXXXXXXXXXXXXXXXXXXXXXXXXXXXXXXXXXXXXXXXXXXXXXXXXXXXXXXXXXXXXXXXXXXXXXXXX
% Throughout your presentation, if you choose to use \section{} and 
% \subsection{} commands, these will automatically be printed on 
% this slide as an overview of your presentation
\section{Agenda}
\begin{frame}{Outline}
  \frametitle{Agenda}
  \tableofcontents[sections={1-}]
\end{frame}
% \begin{frame}
%    \frametitle{Agenda (2)}
%    \tableofcontents[sections={12-}]
% \end{frame}

% XXXXXXXXXXXXXXXXXXXXXXXXXXXXXXXXXXXXXXXXXXXXXXXXXXXXXXXXXXXXXXXXXXXXXXXXXX

\input{os05-BRP.tex}

% XXXXXXXXXXXXXXXXXXXXXXXXXXXXXXXXXXXXXXXXXXXXXXXXXXXXXXXXXXXXXXXXXXXXXXXXXX

\section{OSC10 (Silberschatz) Chapter 10: Virtual Memory}

\begin{frame}
\frametitle{OSC10 (Silberschatz) Chapter 10: Virtual Memory}
  \begin{itemize}
  \item OSC10 Chapter 10
  \begin{itemize}
  \item Background
  \item Demand Paging
  \item Copy-on-Write
  \item Page Replacement
  \item Allocation of Frames
  \item Thrashing
  \item Memory-Mapped Files
  \item Allocating Kernel Memory
  \item Other Considerations
  \item Operating-System Examples
  \end{itemize}
  \end{itemize}
  \vfill \null
\end{frame}

% XXXXXXXXXXXXXXXXXXXXXXXXXXXXXXXXXXXXXXXXXXXXXXXXXXXXXXXXXXXXXXXXXXXXXXXXXX

\section{Virtual Memory}
\begin{frame}
\frametitle{Virtual Memory}
\begin{itemize}
\item Virtual Memory: Separation Logical from Physical.
\item Virtual Address Space: logical view.
\item Demand Paging
\item Page Flags: Valid / Invalid
\item Page Fault
\item Demand Paging Performance
\item Copy On Write (COW)
\item Page Replacement Algorithm
\begin{itemize}
\item Reference String
\item First-In-First-Out (FIFO)
\item Belady Anomaly
\item Optimal Algorithm
\item Least Recently Used (LRU)
\item LRU Implementation
\item Lease Frequently Used (LFU)
\item Most Frequently Used (MFU) 
\end{itemize}
\end{itemize}
\end{frame}

% XXXXXXXXXXXXXXXXXXXXXXXXXXXXXXXXXXXXXXXXXXXXXXXXXXXXXXXXXXXXXXXXXXXXXXXXXX
\section{Memory Allocation Algorithm}
\begin{frame}
\frametitle{Memory Allocation Algorithm}
\begin{itemize}
\item Page-Buffering Algorithms
\item Allocation of Frames
\item Fixed Allocation
\item Priority Allocation
\item Global vs. Local Allocation
\item Non-Uniform Memory Access (NUMA)
\item Thrashing
\item Working-Set Model
\item Shared Memory via Memory-Mapped I/O
\item Kernel
\begin{itemize}
\item Buddy System Allocator
\item Slab Allocator
\end{itemize}
\end{itemize}
\end{frame}

% XXXXXXXXXXXXXXXXXXXXXXXXXXXXXXXXXXXXXXXXXXXXXXXXXXXXXXXXXXXXXXXXXXXXXXXXXX
\section{TOP: Table of Processes}
\begin{frame}[fragile]
\frametitle{TOP: Table of Processes (12-memory.c) (01)}

See also \url{https://osp4diss.vlsm.org/osp-101.html}

% \begin{lstlisting}[basicstyle=\ttfamily\tiny]         % 108
% \begin{lstlisting}[basicstyle=\ttfamily\footnotesize] %  72
% \begin{lstlisting}[basicstyle=\ttfamily\small]        %  65
% \begin{lstlisting}[basicstyle=\ttfamily\large]        %  54
\begin{lstlisting}[basicstyle=\ttfamily\tiny]

/*
 * Copyright (C) 2016-2021 Rahmat M. Samik-Ibrahim
 * http://rahmatm.samik-ibrahim.vlsm.org/
 * This program is free script/software. This program is distributed in the 
 * hope that it will be useful, but WITHOUT ANY WARRANTY; without even the 
 * implied warranty of MERCHANTABILITY or FITNESS FOR A PARTICULAR PURPOSE.
# INFO: TOP (Table of Processes)
 * REV11 Tue 30 Mar 18:25:50 WIB 2021
 * REV07 Fri 26 Mar 22:52:06 WIB 2021
 * REV04 Mon 12 Mar 17:33:30 WIB 2018
 * START Mon 03 Oct 09:26:51 WIB 2016
 */

#define TOKEN  "OS212W05"
#define MSTARTS 0x125E4
// #define MSTARTS 0x2BE5
// #define MSTARTS 0xFE4
// #define MSTARTS 0x3E4
// #define MSTARTS 0x1E4

#define MSIZE14 0x40609
#define MSIZE13 0x40609
#define MSIZE12 0x40608
#define MSIZE11 0x40608
#define MSIZE10 0x20FE8
#define MSIZE09 0x20FE8
#define MSIZE08 0x1F609

\end{lstlisting}
\end{frame}

% XXXXXXXXXXXXXXXXXXXXXXXXXXXXXXXXXXXXXXXXXXXXXXXXXXXXXXXXXXXXXXXXXXXXXXXXXX
\begin{frame}[fragile]
\frametitle{TOP: Table of Processes (12-memory.c) (02)}
% \begin{lstlisting}[basicstyle=\ttfamily\footnotesize]
\begin{lstlisting}[basicstyle=\ttfamily\tiny]

#define MSIZE07 0x1F609
#define MSIZE06 0x1F608
#define MSIZE05 0x1F608
#define MSIZE04 0x1E609
#define MSIZE03 0x1E609
#define MSIZE02 0x1E609
#define MSIZE01 0x1E608
#define MSIZE00 0x1E608
#define LINE   75
#define MAXSTR 80
#include <stdio.h>
#include <stdlib.h>
#include <unistd.h>
#include <string.h>
#include <sys/types.h>

typedef  unsigned char* uChrPtr;
void     chktoken (uChrPtr token);

void printLine(int line) {
   while(line-- > 0) putchar('x');
   putchar('\n');
   fflush(NULL);
}

uChrPtr GlobalChar[MSTARTS];

\end{lstlisting}
\end{frame}

% XXXXXXXXXXXXXXXXXXXXXXXXXXXXXXXXXXXXXXXXXXXXXXXXXXXXXXXXXXXXXXXXXXXXXXXXXX
\begin{frame}[fragile]
\frametitle{TOP: Table of Processes (12-memory.c) (03)}
% \begin{lstlisting}[basicstyle=\ttfamily\tiny]
% \begin{lstlisting}[basicstyle=\ttfamily\footnotesize]
\begin{lstlisting}[basicstyle=\ttfamily\tiny]

void main(void) {
   int   msize[] = {MSIZE00, MSIZE01, MSIZE02, MSIZE03, MSIZE04, 
                    MSIZE05, MSIZE06, MSIZE07, MSIZE08, MSIZE09,
                    MSIZE10, MSIZE11, MSIZE12, MSIZE13, MSIZE14};
   int   ii, jj;
   int   myPID    = (int) getpid();
   char  strSYS1[MAXSTR], strOUT[MAXSTR];
   char* chrPTR; 
   char* chrStr;

   printLine(LINE);
   printf("ZCZC chktoken\n");
   chktoken(TOKEN);
   printLine(LINE);

   sprintf(strSYS1, "top -b -n 1 -p%d | tail -5", myPID);
   system (strSYS1);
   sprintf(strSYS1, "top -b -n 1 -p%d | tail -1", myPID);
   printf("PART 1\n");
   printLine(LINE);
   for (ii=0; ii < (sizeof(msize)/sizeof(int)); ii++){
      chrStr = malloc(msize[ii]);
      FILE* filePtr=popen(strSYS1, "r");
      fgets(strOUT, sizeof(strOUT)-1, filePtr);
      pclose(filePtr);
      strOUT[(int) strlen(strOUT)-1]='\0';
      printf("%s [%X]\n", strOUT, msize[ii]);
      free(chrStr);
   }

\end{lstlisting}
\end{frame}

% XXXXXXXXXXXXXXXXXXXXXXXXXXXXXXXXXXXXXXXXXXXXXXXXXXXXXXXXXXXXXXXXXXXXXXXXXX
\begin{frame}[fragile]
\frametitle{TOP: Table of Processes (12-memory.c) (04)}
% \begin{lstlisting}[basicstyle=\ttfamily\tiny]
% \begin{lstlisting}[basicstyle=\ttfamily\footnotesize]
\begin{lstlisting}[basicstyle=\ttfamily\footnotesize]

   printf("\nPART 2\n");
   printLine(LINE);
   for (ii=0; ii < (sizeof(msize)/sizeof(int)); ii++){
      chrPTR = chrStr = malloc(msize[ii]);
      for (jj=0;jj<msize[ii];jj++)
         *chrPTR++='x';
      FILE* filePtr=popen(strSYS1, "r");
      fgets(strOUT, sizeof(strOUT)-1, filePtr);
      pclose(filePtr);
      strOUT[(int) strlen(strOUT)-1]='\0';
      printf("%s [%X]\n", strOUT, msize[ii]);
      free(chrStr);
   }
}

\end{lstlisting}
\end{frame}

% XXXXXXXXXXXXXXXXXXXXXXXXXXXXXXXXXXXXXXXXXXXXXXXXXXXXXXXXXXXXXXXXXXXXXXXXXX
\begin{frame}[fragile]
\frametitle{TOP: Table of Processes (13-chktoken.c) (05)}
% \begin{lstlisting}[basicstyle=\ttfamily\footnotesize]
\begin{lstlisting}[basicstyle=\ttfamily\tiny]
/*
 * Copyright (C) 2021 Rahmat M. Samik-Ibrahim
 * http://rahmatm.samik-ibrahim.vlsm.org/
 * This program is free script/software. This program is distributed in the 
 * hope that it will be useful, but WITHOUT ANY WARRANTY; without even the 
 * implied warranty of MERCHANTABILITY or FITNESS FOR A PARTICULAR PURPOSE.
 * REV05: Tue 30 Mar 14:55:36 WIB 2021
 * REV04: Tue 30 Mar 10:35:13 WIB 2021
 * REV03: Tue 30 Mar 08:36:56 WIB 2021
 * START: Mon 22 Mar 2021 16:14:36 WIB
 *
# INFO: chktoken(TOKEN) function
 */

#include <stdio.h>
#include <stdlib.h>
#include <string.h>
#include <time.h>

#define  MAXINPUT   256
#define  MAXCMD     MAXINPUT
#define  MAXOUTPUT  MAXINPUT
#define  RESULT     4

typedef           char  Chr;
typedef           char* ChrPtr;
typedef  unsigned char  uChr;
typedef  unsigned char* uChrPtr;

\end{lstlisting}
\end{frame}

% XXXXXXXXXXXXXXXXXXXXXXXXXXXXXXXXXXXXXXXXXXXXXXXXXXXXXXXXXXXXXXXXXXXXXXXXXX
\begin{frame}[fragile]
\frametitle{TOP: Table of Processes (13-chktoken.c) (05)}
% \begin{lstlisting}[basicstyle=\ttfamily\tiny]
% \begin{lstlisting}[basicstyle=\ttfamily\footnotesize]
\begin{lstlisting}[basicstyle=\ttfamily\tiny]

#define CMDSTRING "echo %s | sha1sum | cut -c1-4 | tr '[:lower:]' '[:upper:]' "
void mySHA1(uChrPtr input, uChrPtr output) {
    Chr  cmd[MAXCMD];
    sprintf(cmd, CMDSTRING, input);
    FILE* filePtr = popen(cmd, "r");
    fgets(output, RESULT+1, filePtr);
    output[RESULT]=0;
    pclose(filePtr);
}
void getTimeStamp(uChrPtr timeStamp) {
    time_t tt    =  time(NULL);
    struct tm tm = *localtime(&tt);
    sprintf(timeStamp, "%2.2d%2.2d", tm.tm_min, tm.tm_sec);
}
void     chktoken (uChrPtr token) {
    uChr    input [MAXINPUT];
    uChr    output[MAXOUTPUT];
    uChr    timeStamp[] = "MMSS";
    uChrPtr user        = getenv("USER");
    getTimeStamp(timeStamp);
    int     len   = strlen(timeStamp);
    strcpy(input,timeStamp);
    strcpy(input+len,user);
    len          += strlen(user);
    strcpy(input+len,token);
    len          += strlen(token);
    mySHA1(input,   output);
    printf("%s %s-%s\n", user, timeStamp, output);
}

\end{lstlisting}
\end{frame}

% XXXXXXXXXXXXXXXXXXXXXXXXXXXXXXXXXXXXXXXXXXXXXXXXXXXXXXXXXXXXXXXXXXXXXXXXXX
\begin{frame}[fragile]
\frametitle{TOP: Table of Processes (13-chktoken) (06)}
%\begin{lstlisting}[basicstyle=\ttfamily\footnotesize]
\begin{lstlisting}[basicstyle=\ttfamily\tiny]

xxxxxxxxxxxxxxxxxxxxxxxxxxxxxxxxxxxxxxxxxxxxxxxxxxxxxxxxxxxxxxxxxxxxxxxxxxx
ZCZC chktoken
cbkadal 5257-80A5
xxxxxxxxxxxxxxxxxxxxxxxxxxxxxxxxxxxxxxxxxxxxxxxxxxxxxxxxxxxxxxxxxxxxxxxxxxx
MiB Mem :    986.5 total,    157.1 free,    174.2 used,    655.2 buff/cache
MiB Swap:    488.0 total,    488.0 free,      0.0 used.    632.0 avail Mem 

  PID    VIRT    RES    SHR   SWAP   CODE    DATA   USED nDRT
  864    6000   1528   1240      0      8     948   1528    0
PART 1
xxxxxxxxxxxxxxxxxxxxxxxxxxxxxxxxxxxxxxxxxxxxxxxxxxxxxxxxxxxxxxxxxxxxxxxxxxx
  864    6000   1528   1240      0      8     948   1528    0 [1E608]
  864    6000   2620   2292      0      8     948   2620    0 [1E608]
  864    6132   2620   2292      0      8    1080   2620    0 [1E609]
  864    6004   2620   2292      0      8     952   2620    0 [1E609]
  864    6004   2620   2292      0      8     952   2620    0 [1E609]
  864    6004   2620   2292      0      8     952   2620    0 [1F608]
  864    6004   2620   2292      0      8     952   2620    0 [1F608]
  864    6136   2620   2292      0      8    1084   2620    0 [1F609]
  864    6136   2624   2292      0      8    1084   2624    0 [1F609]
  864    6136   2624   2292      0      8    1084   2624    0 [20FE8]
  864    6136   2624   2292      0      8    1084   2624    0 [20FE8]
  864    6136   2624   2292      0      8    1084   2624    0 [40608]
  864    6136   2624   2292      0      8    1084   2624    0 [40608]
  864    6268   2624   2292      0      8    1216   2624    0 [40609]
  864    6264   2624   2292      0      8    1212   2624    0 [40609]

\end{lstlisting}
\end{frame}

% XXXXXXXXXXXXXXXXXXXXXXXXXXXXXXXXXXXXXXXXXXXXXXXXXXXXXXXXXXXXXXXXXXXXXXXXXX
\begin{frame}[fragile]
\frametitle{TOP: Table of Processes (13-chktoken) (07)}
%\begin{lstlisting}[basicstyle=\ttfamily\footnotesize]
\begin{lstlisting}[basicstyle=\ttfamily\tiny]

PART 2
xxxxxxxxxxxxxxxxxxxxxxxxxxxxxxxxxxxxxxxxxxxxxxxxxxxxxxxxxxxxxxxxxxxxxxxxxxx
  864    6004   2624   2292      0      8     952   2624    0 [1E608]
  864    6004   2736   2292      0      8     952   2736    0 [1E608]
  864    6004   2736   2292      0      8     952   2736    0 [1E609]
  864    6004   2736   2292      0      8     952   2736    0 [1E609]
  864    6004   2736   2292      0      8     952   2736    0 [1E609]
  864    6004   2736   2292      0      8     952   2736    0 [1F608]
  864    6004   2736   2292      0      8     952   2736    0 [1F608]
  864    6136   2736   2292      0      8    1084   2736    0 [1F609]
  864    6136   2736   2292      0      8    1084   2736    0 [1F609]
  864    6136   2736   2292      0      8    1084   2736    0 [20FE8]
  864    6136   2744   2292      0      8    1084   2744    0 [20FE8]
  864    6136   2748   2292      0      8    1084   2748    0 [40608]
  864    6136   2868   2292      0      8    1084   2868    0 [40608]
  864    6268   2868   2292      0      8    1216   2868    0 [40609]
  864    6268   2868   2292      0      8    1216   2868    0 [40609

\end{lstlisting}
\begin{table}
\end{table}
\end{frame}

% XXXXXXXXXXXXXXXXXXXXXXXXXXXXXXXXXXXXXXXXXXXXXXXXXXXXXXXXXXXXXXXXXXXXXXXXXX

% \begin{lstlisting}[basicstyle=\ttfamily\tiny]         % 108
% \begin{lstlisting}[basicstyle=\ttfamily\footnotesize] %  72
% \begin{lstlisting}[basicstyle=\ttfamily\small]        %  65
% \begin{lstlisting}[basicstyle=\ttfamily\large]        %  54

% XXXXXXXXXXXXXXXXXXXXXXXXXXXXXXXXXXXXXXXXXXXXXXXXXXXXXXXXXXXXXXXXXXXXXXXXXX

\end{document}

