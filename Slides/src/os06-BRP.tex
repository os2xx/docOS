%%%%%%%%%%%%%%%%%%%%%%%%%%%%%%%
% REV412: Tue 22 Aug 2023 14:00
% REV406: Sat 05 Aug 2023 12:00
% REV211: Fri 01 Nov 2019 01:00
% REV154: Thu 23 Aug 2018 11:00
% START0: Thu 26 Jul 2018 20:00
%%%%%%%%%%%%%%%%%%%%%%%%%%%%%%%

\section{Week 06}
\begin{frame}[fragile]
\frametitle{Week 06 Concurrency:
Topics\footnote{Source: ACM IEEE CS Curricula}}

\begin{itemize}
\item States and state diagrams 
\item Structures (ready list, process control blocks, and so forth) 
\item Dispatching and context switching 
\item The role of interrupts 
\item Managing atomic access to OS objects 
\item Implementing synchronization primitives 
\item Multiprocessor issues (spin-locks, reentrancy)  
\end{itemize}
\end{frame}

\begin{frame}[fragile]
\frametitle{Week 06 Concurrency:
Learning Outcomes (1)\footnote{Source: ACM IEEE CS Curricula}}
\begin{itemize}
\item Describe the need for concurrency within the framework of an operating system. [Familiarity] 
\item Demonstrate the potential run-time problems arising from the concurrent operation of many separate tasks. [Usage] 
\item Summarize the range of mechanisms that can be employed at the operating system level to realize concurrent systems and describe the benefits of each. [Familiarity] 
\item Explain the different states that a task may pass through and the data structures needed to support the management of many tasks. [Familiarity] 
\end{itemize}
\end{frame}


\begin{frame}[fragile]
\frametitle{Week 06 Concurrency:
Learning Outcomes (2)\footnote{Source: ACM IEEE CS Curricula 2023 (beta)}}
\begin{itemize}
\item Summarize techniques for achieving synchronization in an operating system (e.g., describe how to implement a semaphore using OS primitives). [Familiarity] 
\item Describe reasons for using interrupts, dispatching, and context switching to support concurrency in an operating system. [Familiarity] 
\item Create state and transition diagrams for simple problem domains. [Usage] 
\end{itemize}
\end{frame}

