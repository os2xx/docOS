%%%%%%%%%%%%%%%%%%%%%%%%%%%%%%%%%%%%%%%%%%%%%%%%%%%%%%%%%%%%%%%%%%%%%%%%
% Beamer Presentation - LaTeX - Template Version 1.0 (10/11/12)
% This template has been downloaded from: http://www.LaTeXTemplates.com
% License: % CC BY-NC-SA 3.0 (http://creativecommons.org/)
% Modified by BinKadal Sdn. Bhd.

% REV023: Tue 30 Jan 2024 22:00
% REV022: Mon 13 Feb 2023 00:00
% REV003: Mon 14 Feb 2022 21:00
% REV001: Mon 07 Feb 2022 05:00
% STARTX: Wed 14 Sep 2016 10:00
%%%%%%%%%%%%%%%%%%%%%%%%%%%%%%%%%%%%%%%%%%%%%%%%%%%%%%%%%%%%%%%%%%%%%%%%%

% PACKAGES AND THEMES 
\documentclass[aspectratio=169, xcolor=table, notheorems, hyperref={pdfpagelabels=false}]{beamer}
%%%%%%%%%%%%%%%%%%%%%%%%%%%%%%%%%%%%%%%%%%%%%%%%%%%%%%%%%%%%%%%%%%%%%%%%
% Beamer Presentation - LaTeX - Template Version 1.0 (10/11/12)
% This template has been downloaded from: http://www.LaTeXTemplates.com
% License: % CC BY-NC-SA 3.0 (http://creativecommons.org/)
% Modified by Rahmat M. Samik-Ibrahim
% REV419: Wed 24 Jul 2024 17:00
% REV383: Tue 12 Jul 2022 10:00
% REV316: Wed 14 Jul 2021 13:00
% REV198: Wed 13 Mar 2019 16:00
% REV005: Mon  2 Oct 2017 14:00
% STARTX: Thu 25 Aug 2016 14:00
%%%%%%%%%%%%%%%%%%%%%%%%%%%%%%%%%%%%%%%%%%%%%%%%%%%%%%%%%%%%%%%%%%%%%%%%%

%% ZCZC NNNN
\newtheorem{example}{Example}

%%%%%%%%%%%%%%%%%%%%%%%%%%%%%%%%%%%%%%%%%%%%%%%%%%%%%%%%%%%%%%%%%%%%%%%%%

\let\Tiny=\tiny
\mode<presentation> {
% The Beamer class comes with a number of default slide themes
% which change the colors and layouts of slides. Below this is a list
% of all the themes, uncomment each in turn to see what they look like.
%\usetheme{Boadilla}
\usetheme{Madrid}
% ZCZC %%%%%%%%%%%%%%%%%%%%%%%%%%%%%%%%%%%%%%%%%%%%%%%%%%%%%%%%%%%%%%%%%%
% \usetheme{default} \usetheme{AnnArbor} \usetheme{Antibes} \usetheme{Bergen}
% \usetheme{Berkeley} \usetheme{Berlin} \usetheme{CambridgeUS} 
% \usetheme{Copenhagen} \usetheme{Darmstadt} \usetheme{Dresden}
% \usetheme{Frankfurt} \usetheme{Goettingen} \usetheme{Hannover}
% \usetheme{Ilmenau} \usetheme{JuanLesPins} \usetheme{Luebeck}
% \usetheme{Malmoe} \usetheme{Marburg} \usetheme{Montpellier}
% \usetheme{PaloAlto} \usetheme{Pittsburgh} \usetheme{Rochester}
% \usetheme{Singapore} \usetheme{Szeged} \usetheme{Warsaw}
% NNNN %%%%%%%%%%%%%%%%%%%%%%%%%%%%%%%%%%%%%%%%%%%%%%%%%%%%%%%%%%%%%%%%%%
% As well as themes, the Beamer class has a number of color themes
% for any slide theme. Uncomment each of these in turn to see how it
% changes the colors of your current slide theme.
%\usecolortheme{orchid}
%\usecolortheme{rose}
%\usecolortheme{seagull}
%\usecolortheme{seahorse}
\usecolortheme{whale}
% ZCZC %%%%%%%%%%%%%%%%%%%%%%%%%%%%%%%%%%%%%%%%%%%%%%%%%%%%%%%%%%%%%%%%%%
%\usecolortheme{albatross} \usecolortheme{beaver} \usecolortheme{beetle}
%\usecolortheme{crane} \usecolortheme{dolphin} \usecolortheme{dove}
%\usecolortheme{fly} \usecolortheme{lily} \usecolortheme{wolverine}
% NNNN %%%%%%%%%%%%%%%%%%%%%%%%%%%%%%%%%%%%%%%%%%%%%%%%%%%%%%%%%%%%%%%%%%
% To remove the footer line in all slides uncomment this line
%\setbeamertemplate{footline} 
% To replace the footer line in all slides uncomment this line
%\setbeamertemplate{footline}[page number] 
% To remove the navigation symbols from the bottom uncomment this line
\setbeamertemplate{navigation symbols}{} 
}

\usepackage{array}       % ZCZC
\usepackage{amssymb}     % ZCZC
\usepackage{bold-extra}  % ZCZC
\usepackage{booktabs}    % Allows \toprule, \midrule and \bottomrule in tables
\usepackage{caption}
\usepackage[T1]{fontenc} % ZCZC << >>
\usepackage{graphicx}    % Allows including images
\usepackage{listings}    % listing
\usepackage{lmodern}     % ZCZC
\usepackage{perpage}     % reset footnote per page
\usepackage{geometry}    % ZCZC
\usepackage{adjustbox}   % ZCZC
\usepackage{multicol}    % ZCZC
\usepackage{multirow}    % ZCZC
\usepackage{pgf-pie}     % ZCZC pie chart

% \definecolor{links}{HTML}{2A1B81}
\definecolor{links}{HTML}{0011FF}
\hypersetup{colorlinks,linkcolor=,urlcolor=links}

% \usepackage{xcolor}
% \usepackage[colorlinks = true,
%             linkcolor = blue,
%             urlcolor  = blue,
%             citecolor = blue,
%             anchorcolor = blue]{hyperref}

\captionsetup[table]{name=Tabel}
\makeatletter
\def\input@path{{src/}}
\makeatother
\graphicspath{{src/}}      % src directory
\MakePerPage{footnote}     % reset page

% NNNN %%%%%%%%%%%%%%%%%%%%%%%%%%%%%%%%%%%%%%%%%%%%%%%%%%%%%%%%%%%%%%%%%%

%% % XXXXXXXXXXXXXXXXXXXXXXXXXXXXXXXXXXXXXXXXXXXXXXXXXXXXXXXXXXXXXXXXXXXXXXXXXX
%% % The short title appears at the bottom of every slide, 
%% % the full title is only on the title page
%% \title[Judul Pendek]{Judul Panjang dan Lengkap} 
%% \author{Cecak bin Kadal}
%% \institute[UILA]
%% {
%% University of Indonesia at Lenteng Agung \\ 
%% \medskip
%% \textit{cecak@binKadal.com}
%% }
%% \date{REV00 24-Aug-2016}
%% % \date{\today}
%% 

%% % XXXXXXXXXXXXXXXXXXXXXXXXXXXXXXXXXXXXXXXXXXXXXXXXXXXXXXXXXXXXXXXXXXXXXXXXXX
%% \begin{document}
%% \section{Judul}
%% \begin{frame}
%% \titlepage
%% \end{frame}
%% 
%% % XXXXXXXXXXXXXXXXXXXXXXXXXXXXXXXXXXXXXXXXXXXXXXXXXXXXXXXXXXXXXXXXXXXXXXXXXX
%% \section{Agenda}
%% \begin{frame}
%% \frametitle{Agenda}
%% % Throughout your presentation, if you choose to use \section{} and 
%% % \subsection{} commands, these will automatically be printed on 
%% % this slide as an overview of your presentation
%% \tableofcontents 
%% \end{frame}
%% 
%% % XXXXXXXXXXXXXXXXXXXXXXXXXXXXXXXXXXXXXXXXXXXXXXXXXXXXXXXXXXXXXXXXXXXXXXXXXX
%% \section{UUD dan Pancasila}
%% \subsection{UUD}
%% \begin{frame}
%% \frametitle{Pembukaan}
%% Bahwa sesungguhnya kemerdekaan itu ialah hak segala bangsa dan oleh 
%% sebab itu, maka penjajahan diatas dunia harus dihapuskan karena 
%% tidak sesuai dengan perikemanusiaan dan perikeadilan.
%% \\~\\
%% Atas berkat rahmat Allah Yang Maha Kuasa dan dengan didorongkan oleh 
%% keinginan luhur, supaya berkehidupan kebangsaan yang bebas, maka 
%% rakyat Indonesia menyatakan dengan ini kemerdekaannya.
%% \end{frame}
%% 
%% % XXXXXXXXXXXXXXXXXXXXXXXXXXXXXXXXXXXXXXXXXXXXXXXXXXXXXXXXXXXXXXXXXXXXXXXXXX
%% \begin{frame}
%% \frametitle{Alenia Ketiga}
%% Kemudian daripada itu untuk membentuk suatu pemerintah negara Indonesia 
%% yang melindungi segenap bangsa Indonesia dan seluruh tumpah darah Indonesia 
%% dan untuk memajukan kesejahteraan umum, mencerdaskan kehidupan bangsa, dan 
%% ikut melaksanakan ketertiban dunia yang berdasarkan kemerdekaan, perdamaian 
%% abadi dan keadilan sosial, maka disusunlah kemerdekaan kebangsaan Indonesia 
%% itu dalam suatu Undang-Undang Dasar negara Indonesia, yang terbentuk dalam 
%% suatu susunan negara Republik Indonesia yang berkedaulatan rakyat dengan 
%% berdasar kepada:
%% \begin{itemize}
%% \item Ketuhanan Yang Maha Esa,
%% \item kemanusiaan yang adil dan beradab,
%% \item persatuan Indonesia,
%% \item dan kerakyatan yang dipimpin oleh hikmat kebijaksanaan 
%%       dalam permusyawaratan/ perwakilan,
%% \item serta dengan mewujudkan suatu keadilan sosial bagi seluruh rakyat 
%%       Indonesia.
%% \end{itemize}
%% \end{frame}
%% 
%% % XXXXXXXXXXXXXXXXXXXXXXXXXXXXXXXXXXXXXXXXXXXXXXXXXXXXXXXXXXXXXXXXXXXXXXXXXX
%% \subsection{Pancasila}
%% \begin{frame}
%% \frametitle{Tujuh Kunci Pokok}
%% \begin{block}{Pertama - Kedua - Ketiga}
%% Indonesia ialah negara berdasarkan hukum.
%% Sistem konstitusional.
%% Kekuasaan negara tertinggi di tangan MPR.
%% \end{block}
%% 
%% \begin{block}{Keempat - Kelima}
%% Presiden adalah penyelenggara pemerintahan tertinggi di bawah MPR.
%% Adanya pengawasan DPR.
%% \end{block}
%% 
%% \begin{block}{Keenam}
%% Menteri negara adalah pembantu presiden dan tidak bertanggung jawab 
%% kepada DPR.
%% \end{block}
%% 
%% \begin{block}{Ketujuh}
%% Kekuasaan kepala negara tidak tak tebatas.
%% \end{block}
%% 
%% \end{frame}
%% 
%% % XXXXXXXXXXXXXXXXXXXXXXXXXXXXXXXXXXXXXXXXXXXXXXXXXXXXXXXXXXXXXXXXXXXXXXXXXX
%% \section{Rupa-rupa}
%% \subsection{Kolom}
%% \begin{frame}
%% \frametitle{Kolom}
%% % The "c" option specifies centered vertical alignment 
%% % while the "t" option is used for top vertical alignment
%% \begin{columns}[c] 
%% % Left column and width
%% \column{.45\textwidth} 
%% \textbf{Heading}
%% \begin{enumerate}
%% \item Satu-satu
%% \item Dua-dua
%% \item Tiga-tiga
%% \item Satu-dua-tiga
%% \end{enumerate}
%% 
%% % Right column and width
%% \column{.5\textwidth}
%% Satu-satu~\dots{} aku sayang ibu!
%% Dua-dua~\ldots{} juga sayang ayah!
%% Tiga-tiga~\ldots{} sayang adik kakak!
%% Satu-dua-tiga~\ldots{} sayang semuanya!
%% 
%% \end{columns}
%% \end{frame}
%% 
%% % XXXXXXXXXXXXXXXXXXXXXXXXXXXXXXXXXXXXXXXXXXXXXXXXXXXXXXXXXXXXXXXXXXXXXXXXXX
%% \subsection{Tabel}
%% \begin{frame}
%% \frametitle{Tabel}
%% \begin{table}
%% \begin{tabular}{l l l}
%% \toprule
%% \textbf{Nama} & \textbf{NPM} & \textbf{Tanggal Lahir}\\
%% \midrule
%% Cecak bin Kadal & 1234567890 & 1 Jan 2015 \\
%% Aneh bin Ajaib  & 0987654321 & 31 Des 2014 \\
%% \bottomrule
%% \end{tabular}
%% \caption{Keterangan Tabel}
%% \end{table}
%% \end{frame}
%% 
%% % XXXXXXXXXXXXXXXXXXXXXXXXXXXXXXXXXXXXXXXXXXXXXXXXXXXXXXXXXXXXXXXXXXXXXXXXXX
%% \subsection{Teori}
%% \begin{frame}
%% \frametitle{Teori}
%% \begin{theorem}[Teori Satu Batu]
%% $E = mc^2$
%% \end{theorem}
%% \end{frame}
%% 
%% % XXXXXXXXXXXXXXXXXXXXXXXXXXXXXXXXXXXXXXXXXXXXXXXXXXXXXXXXXXXXXXXXXXXXXXXXXX
%% \subsection{Verbatim}
%% % Need to use the fragile option when verbatim is used in the slide
%% \begin{frame}[fragile] 
%% \frametitle{Verbatim}
%% \begin{example}[Teori Satu Batu]
%% \begin{verbatim}
%% \begin{theorem}[Teori Satu Batu]
%% $E = mc^2$
%% \end{theorem}
%% \end{verbatim}
%% \end{example}
%% \end{frame}
%% 
%% % XXXXXXXXXXXXXXXXXXXXXXXXXXXXXXXXXXXXXXXXXXXXXXXXXXXXXXXXXXXXXXXXXXXXXXXXXX
%% \subsection{Gambar}
%% \begin{frame}
%% \frametitle{Gambar}
%% \begin{figure}
%% \includegraphics[width=0.5\linewidth]{2}
%% \caption{Ini Gambar JPG}
%% \end{figure}
%% \end{frame}
%% 
%% % XXXXXXXXXXXXXXXXXXXXXXXXXXXXXXXXXXXXXXXXXXXXXXXXXXXXXXXXXXXXXXXXXXXXXXXXXX
%% \subsection{Rujukan}
%% % Need to use the fragile option when verbatim is used in the slide
%% \begin{frame}[fragile] 
%% \frametitle{Rujukan dan Kutipan}
%% Contoh penggunaan \verb|\cite| ketika mengutip\cite{p1}.
%% Perhatian: Beamer tidak mengerti \verb|\BibTeX|~\ldots
%% \footnotesize{
%%   \begin{thebibliography}{99} 
%%   \bibitem[Smith, 2012]{p1} John Smith (2012)
%%      \newblock Katak dalam Tempurung
%%      \newblock \emph{Jurnal Kelapa dan Amfibi} 12(3), 45 -- 678.
%%   \end{thebibliography}
%% }
%% \end{frame}
%% 
%% % XXXXXXXXXXXXXXXXXXXXXXXXXXXXXXXXXXXXXXXXXXXXXXXXXXXXXXXXXXXXXXXXXXXXXXXXXX
%% \subsection{Selesai}
%% \begin{frame}
%% \Huge{\centerline{Selesai}}
%% \end{frame}
%% 
%% % XXXXXXXXXXXXXXXXXXXXXXXXXXXXXXXXXXXXXXXXXXXXXXXXXXXXXXXXXXXXXXXXXXXXXXXXXX
%% \end{document}

\newcommand{\revision}{REV025: Wed 24 Jul 2024 19:00}
% w! tmptmp
% REV025: Wed 24 Jul 2024 19:00
% REV019: Thu 02 Feb 2023 19:00
% REV009: Mon 18 Apr 2022 06:00
% REV007: Thu 17 Mar 2022 10:00
% REV001: Mon 07 Feb 2022 18:00
% STARTS: Wed 24 Aug 2016 19:00
%%%%%%%%%%%%%%%%%%%%%%%%%%%%%%%
\newcommand{\kopikopi}{\textcopyright{}2016-2024 CBKadal + VauLSMorg}



% XXXXXXXXXXXXXXXXXXXXXXXXXXXXXXXXXXXXXXXXXXXXXXXXXXXXXXXXXXXXXXXXXXXXXXXXXX
% The short title appears at the bottom of every slide, 
% the full title is only on the title page
% \date{\today}
\title[\kopikopi]
{CSCE604227 System Programming \\
CSCE604227 Pemrograman Sistem \\
Week 01:
Linux Kernel and Programming Interface}
\author{C. BinKadal}
\institute[SDN]
{
Sendirian Berhad\\
\medskip
\url{https://docOS.vlsm.org/SPSlides/sp01.pdf}
\\ \texttt{Always check for the latest revision!}
}
\date{\revision}

% XXXXXXXXXXXXXXXXXXXXXXXXXXXXXXXXXXXXXXXXXXXXXXXXXXXXXXXXXXXXXXXXXXXXXXXXXX
\begin{document}

\lstset{
basicstyle=\ttfamily\tiny, % \tiny \small \footnotesize 
breakatwhitespace=true,
language=C,
columns=fullflexible,
keepspaces=true,
breaklines=true,
tabsize=3, 
showstringspaces=false,
extendedchars=true}

\section{Start}
\begin{frame}
\titlepage
\end{frame}

% XXXXXXXXXXXXXXXXXXXXXXXXXXXXXXXXXXXXXXXXXXXXXXXXXXXXXXXXXXXXXXXXXXXXXXXXXX

%%%%%%%%%%%%%%%%%%%%%%%%%%%%%%%
% REV024: Wed 31 Jan 2024 10:00
% REV023: Tue 30 Jan 2024 22:00
% REV019: Thu 02 Feb 2023 00:00
% REV003: Mon 14 Feb 2022 21:00
% REV001: Mon 07 Feb 2022 18:00
% START0: Wed 14 Sep 2016 10:00
%%%%%%%%%%%%%%%%%%%%%%%%%%%%%%%

\begin{frame}[fragile]
\section{Schedule}
\frametitle{SP241\footnote{%
) This information will be on \textbf{EVERY} page two (2) of this course material.}): 
System Progamming}

\vspace{5pt}

\scalebox{1.2}{%
\begin{tabular}{|c|c|l|}
\hline
\textbf{Week} & 
\textbf{Topic} \\ 
\hline
Week 00  & Overview \\
Week 01  & Linux Kernel and Programming Interface \\
Week 02  & Revisit Linux From Scratch \\
Week 03  & FUSE: Filesystem in Userspace  \\
Week 04  & GetOpt \\
Week 05  & Autoconf and Automake \\
\hline
Week 06  & Boxing/Unboxing \\
Week 07  & Sync, SETUID, and MMAP \\
Week 08  & Kernel Modules I   \\
Week 09  & Kernel Modules II  \\
Week 10  & Kernel Modules III \\
\hline
\end{tabular}
}
\end{frame}

\begin{frame}[fragile]
\frametitle{\textbf{STARTING POINT} --- 
{
\definecolor{links}{HTML}{FDEE00}
\hypersetup{colorlinks,linkcolor=,urlcolor=links}
\url{https://sp.vlsm.org/}
}
}
\begin{itemize}
\item[$\square$] \textbf{Text Book} ---
The Linux Programming Interface, 2010, No Starch Press, 
ISBN 978-1-59327-220-3 --- \url{https://man7.org/tlpi/}. 
\item[$\square$] \textbf{Resources}
\begin{itemize}
\item[$\square$] \href{https://scele.cs.ui.ac.id/course/view.php?id=3742}{\textbf{SCELE}} ---
\url{https://scele.cs.ui.ac.id/course/view.php?id=3742}.\\
The enrollment key is \textbf{XXX}.
\item[$\square$] \textbf{Download Slides and Demos from GitHub.com} \\
\url{https://github.com/os2xx/docOS/}:

                 {\scriptsize%
                 \href{https://docOS.vlsm.org/SPSlides/sp00.pdf}{\texttt{sp00.pdf} (W00)},
                 \href{https://docOS.vlsm.org/SPSlides/sp01.pdf}{\texttt{sp01.pdf} (W01)},
                 \href{https://docOS.vlsm.org/SPSlides/sp02.pdf}{\texttt{sp02.pdf} (W02)},
                 \href{https://docOS.vlsm.org/SPSlides/sp03.pdf}{\texttt{sp03.pdf} (W03)},

                 \href{https://docOS.vlsm.org/SPSlides/sp04.pdf}{\texttt{sp04.pdf} (W04)},
                 \href{https://docOS.vlsm.org/SPSlides/sp05.pdf}{\texttt{sp05.pdf} (W05)},
                 \href{https://docOS.vlsm.org/SPSlides/sp06.pdf}{\texttt{sp06.pdf} (W06)},
                 \href{https://docOS.vlsm.org/SPSlides/sp07.pdf}{\texttt{sp07.pdf} (W07)},

                 \href{https://docOS.vlsm.org/SPSlides/sp08.pdf}{\texttt{sp08.pdf} (W08)},
                 \href{https://docOS.vlsm.org/SPSlides/sp09.pdf}{\texttt{sp09.pdf} (W09)},
                 \href{https://docOS.vlsm.org/SPSlides/sp10.pdf}{\texttt{sp10.pdf} (W10)}.
                 }
\item[$\square$] \textbf{LFS} --- \url{http://www.linuxfromscratch.org/lfs/view/stable/}
\item[$\square$] \textbf{OSP4DISS} --- \url{https://osp4diss.vlsm.org/}
\item[$\square$] \textbf{This is How Me DO IT!} --- \url{https://doit.vlsm.org/}
\begin{itemize}
\item[$\square$] PS: ''Me'' rhymes better than ''I'' duh!
\end{itemize}
\end{itemize}
\end{itemize}
\end{frame}



% XXXXXXXXXXXXXXXXXXXXXXXXXXXXXXXXXXXXXXXXXXXXXXXXXXXXXXXXXXXXXXXXXXXXXXXXXX
% Throughout your presentation, if you choose to use \section{} and 
% \subsection{} commands, these will automatically be printed on 
% this slide as an overview of your presentation
\section{Agenda}
\begin{frame}{Outline}
  \frametitle{Agenda}
  \tableofcontents[sections={1-}]
\end{frame}
% \begin{frame}
%    \frametitle{Agenda (2)}
%    \tableofcontents[sections={12-}]
% \end{frame}

% XXXXXXXXXXXXXXXXXXXXXXXXXXXXXXXXXXXXXXXXXXXXXXXXXXXXXXXXXXXXXXXXXXXXXXX
\section{The Linux Kernel Archives}
\begin{frame}[fragile]
\frametitle{The Linux Kernel Archives}
\begin{itemize}
\item URL: \url{https://kernel.org/}
\begin{itemize}
\item HTTP: \url{https://www.kernel.org/pub/}
\item Kernel Source: \url{https://www.kernel.org/pub/linux/kernel/}
\end{itemize}
\item 30-Jan-2024 --- 6.7.2 --- \url{https://www.kernel.org/finger_banner}
% \begin{lstlisting}[basicstyle=\ttfamily\tiny]
\begin{lstlisting}[basicstyle=\ttfamily\small]
The latest stable version of the Linux kernel is:             6.7.2
The latest mainline version of the Linux kernel is:           6.8-rc2
The latest stable 6.7 version of the Linux kernel is:         6.7.2
The latest longterm 6.6 version of the Linux kernel is:       6.6.14
The latest longterm 6.1 version of the Linux kernel is:       6.1.75
[...]
The latest longterm 5.4 version of the Linux kernel is:       5.4.268
The latest longterm 4.19 version of the Linux kernel is:      4.19.306
The latest longterm 4.14 version of the Linux kernel is:      4.14.336 (EOL)
The latest linux-next version of the Linux kernel is:         next-20240130
\end{lstlisting}
\item How to compile Linux Kernel on Debian VirtualBox Guest --- \url{https://doit.vlsm.org/007.html}.
\end{itemize}
\end{frame}

% XXXXXXXXXXXXXXXXXXXXXXXXXXXXXXXXXXXXXXXXXXXXXXXXXXXXXXXXXXXXXXXXXXXXXXX
\section{The Linux Programming Interface}
\begin{frame}[fragile]
\frametitle{The Linux Programming Interface (1)}
\begin{itemize}
\item URL: \url{https://man7.org/tlpi/}
\item Purchasing: \url{https://man7.org/tlpi/purchase.html}
\item Source Code: \url{https://man7.org/tlpi/code/}
\begin{itemize}
\item Lastest Distribution: \url{https://man7.org/tlpi/code/download/tlpi-240109-dist.tar.gz}
\item Download Full List: \url{https://man7.org/tlpi/code/download/}
\end{itemize}
\item README: \url{https://man7.org/tlpi/code/README.html}
\item BUILDING notes: \url{https://man7.org/tlpi/code/BUILDING.html}
\item FAQ: \url{https://man7.org/tlpi/code/faq.html}
\item Debian Packages List: \url{https://doit.vlsm.org/026.html}.
\end{itemize}
\end{frame}

% XXXXXXXXXXXXXXXXXXXXXXXXXXXXXXXXXXXXXXXXXXXXXXXXXXXXXXXXXXXXXXXXXXXXXXX
\begin{frame}[fragile]
\frametitle{The Linux Programming Interface (2)}
\begin{itemize}
\item Test: \texttt{time/calendar\_time}

% \begin{lstlisting}[basicstyle=\ttfamily\tiny]
\begin{lstlisting}[basicstyle=\ttfamily\small]

cbkadal@cbkadal:~/src/tlpi-dist$ time/calendar_time 
Seconds since the Epoch (1 Jan 1970): 1644855271 (about 52.123 years)
  gettimeofday() returned 1644855271 secs, 35730 microsecs
Broken down by gmtime():
  year=122 mon=1 mday=14 hour=16 min=14 sec=31 wday=1 yday=44 isdst=0
Broken down by localtime():
  year=122 mon=1 mday=14 hour=23 min=14 sec=31 wday=1 yday=44 isdst=0

asctime() formats the gmtime() value as: Mon Feb 14 16:14:31 2022
ctime() formats the time() value as:     Mon Feb 14 23:14:31 2022
mktime() of gmtime() value:    1644830071 secs
mktime() of localtime() value: 1644855271 secs

cbkadal@cbkadal:~/src/tlpi-dist$

\end{lstlisting}
\end{itemize}
\end{frame}

% XXXXXXXXXXXXXXXXXXXXXXXXXXXXXXXXXXXXXXXXXXXXXXXXXXXXXXXXXXXXXXXXXXXXXXX
\begin{frame}[fragile]
\frametitle{The Linux Programming Interface (3)}
\begin{itemize}
\item Test: \texttt{getopt/t\_getopt} (TLPI Appendix B)
% \begin{lstlisting}[basicstyle=\ttfamily\small]
\begin{lstlisting}[basicstyle=\ttfamily\tiny]
cbkadal@cbkadal:~/src/tlpi-dist$ getopt/t_getopt -x -p hello world
opt =120 (x); optind = 2
opt =112 (p); optind = 4
-x was specified (count=1)
-p was specified with the value "hello"
First nonoption argument is "world" at argv[4]

cbkadal@cbkadal:~/src/tlpi-dist$ getopt/t_getopt -p
opt = 58 (:); optind = 2; optopt =112 (p)
Missing argument (-p)
Usage: getopt/t_getopt [-p arg] [-x]

cbkadal@cbkadal:~/src/tlpi-dist$ getopt/t_getopt -a
opt = 63 (?); optind = 2; optopt = 97 (a)
Unrecognized option (-a)
Usage: getopt/t_getopt [-p arg] [-x]

cbkadal@cbkadal:~/src/tlpi-dist$ getopt/t_getopt -p str -- -x
opt =112 (p); optind = 3
-p was specified with the value "str"
First nonoption argument is "-x" at argv[4]

cbkadal@cbkadal:~/src/tlpi-dist$ getopt/t_getopt -p -x
opt =112 (p); optind = 3
-p was specified with the value "-x"

\end{lstlisting}
\end{itemize}
\end{frame}

\end{document}

